%nag warns a lot about tikzposter.
%\RequirePackage[l2tabu, orthodox]{nag}
\documentclass[blockverticalspace=3cm]{tikzposter}
\input{preamble/packages}
%\input{preamble/redac}
\input{preamble/math_basics}
%Decision Theory (MCDA and SC)
\newcommand{\allalts}{\mathscr{A}}
\newcommand{\allcrits}{\mathscr{C}}
\newcommand{\alts}{A}
\newcommand{\dm}{i}
\newcommand{\allF}{\mathscr{F}}
\newcommand{\allvoters}{\mathscr{N}}
\newcommand{\voters}{N}
\newcommand{\allprofs}{\boldsymbol{\mathcal{R}}}
\newcommand{\prof}{\boldsymbol{R}}
\newcommand{\linors}{\mathscr{L}(\allalts)}
%Thanks to https://tex.stackexchange.com/q/154549
	%\makeatletter
	%\def\@myRgood@#1#2{\mathrel{R^X_{#2}}}
	%\def\myRgood{\@ifnextchar_{\@myRgood@}{\mathrel{R^X}}}
	%\makeatother

%Deliberated Judgment
\newcommand{\prop}{\textcolor{red}{t}}
\newcommand{\allargs}{S^*}
\newcommand{\args}{S}
\newcommand{\ar}[1][]{%
	\textcolor{blue}{%
		\ifx\\#1\\%
			s
		\else
			s_#1
		\fi%
	}%
}
\newcommand{\ileadsto}{\textcolor{brown}{⇝}}
\newcommand{\ibeatse}{\textcolor{brown}{⊳_\exists}}
\newcommand{\nibeatse}{⋫_\exists}
\newcommand{\ibeatsst}{⊳_\forall}
\newcommand{\nibeatsst}{⋫_\forall}
\newcommand{\mleadsto}[1][\eta]{⇝_{#1}}
\newcommand{\mbeats}[1][\eta]{\textcolor{violet}{⊳_{#1}}}
\newcommand{\ibeatseinv}{⊳_\exists^{-1}}

%Logic
\newcommand{\ltru}{\texttt{T}}
\newcommand{\lfal}{\texttt{F}}


\input{preamble/draw}
\usepackage{tabularx}
\tikzposterlatexaffectionproofoff
\usepackage{rsc}
\renewcommand{\bibsection}{}
%\usepackage[backend=bibtex,style=authoryear]{biblatex}
%\addbibresource{biblio.bib}

\listfiles

\definecolor{myorange}{RGB}{250, 95, 0} 
\definecolor{myburgundy}{RGB}{150, 0, 15} 

\colorlet{backgroundcolor}{white}
\colorlet{framecolor}{orange}
\colorlet{blocktitlebgcolor}{myorange}

\newcommand{\profile}{\bm{v}}%(complete) profile
\newcommand{\pprofile}{{\bm{p}}}%partial profile
\newcommand{\w}{\bm{w}}
\newcommand{\W}{\mathcal{W}}
\newcommand{\Co}{\mathcal{C}}
\newcommand{\pw}{W}%our knowledge about the weights
\newcommand{\strat}[1]{\emph{#1}}
\newcommand{\ppref}{\succ^\text{p}}%partial pref
\newcommand{\pprefeq}{\succeq^\text{p}}%partial pref
\newcommand{\pref}{\succ}% pref
\DeclareMathOperator{\Regret}{Regret}
\DeclareMathOperator{\SCORE}{Score}
\DeclareMathOperator{\PMR}{PMR}
\DeclareMathOperator{\MR}{MR}
\DeclareMathOperator{\MMR}{MMR}



\makeatletter
\def\title#1{\gdef\@title{\scalebox{\TP@titletextscale}{%
			\begin{minipage}[t]{\linewidth}
				\centering
				#1
				\par
				\vspace{0.5em}
			\end{minipage}%
		}}}
		\makeatother
%I find these settings useful in draft mode. Should be removed for final versions.
	%Which line breaks are chosen: accept worse lines, therefore reducing risk of overfull lines. Default = 200.
%		\tolerance=2000
	%Accept overfull hbox up to...
%		\hfuzz=2cm
	%Reduces verbosity about the bad line breaks.
%		\hbadness 5000
	%Reduces verbosity about the underful vboxes.
%		\vbadness=1300

\title{Simultaneous Elicitation of Committee and \\ Voters' Preferences}
\institute{$^1$ LAMSADE, Université Paris-Dauphine, Paris, France \\ $^2$ LIP6, Sorbonne Universit\'e, Paris, France}
\author{B. Napolitano$^1$, O. Cailloux$^1$ and P. Viappiani$^2$}


\begin{document}
	
\maketitle[titletotopverticalspace=10cm]

\begin{columns}
	\column{0.5}
		\block{Scenario}{
				\begin{tikzpicture}[remember picture,overlay]
					\path (current page.north west) ++(1.5cm, -1cm) node[anchor=north west, inner sep=0] (first) {
						\includegraphics[height=6.5cm]{dauphine_psl2018.png}
					}; 
					\path (current page.north east) ++(-1.5cm, -1cm) node[anchor=north east, inner sep=0] {
						\includegraphics[height=7cm]{LAMSADE95.jpg}
					};
				\end{tikzpicture}
		%
		%
%			\begin{tikzpicture}[remember picture,overlay]
%				\path (current page.south west) ++(1.5cm, 1.5cm) node[anchor=south west, text width=27cm] {
%					Olivier Cailloux and Yves Meinard. \emph{A formal framework for deliberated judgment}. Under revision at Theory and Decision. \href{https://arxiv.org/abs/1801.05644}{arXiv:1801.05644 [cs.AI]}.
%				};
%			\end{tikzpicture}%
		%
			\textbf{Incompletely specified profile and positional scoring rule \\}
			\begin{tikzfigure}
				\includegraphics[scale=0.975]{setting.png}
				%		\caption{.}
				%		\label{fig:b1}
			\end{tikzfigure}

			\textbf{Goal}
			\begin{itemize}
				\item[] Development of query strategies interleaving questions to the committee and to the voters in order to simultaneously elicit preferences and voting rule
			\end{itemize}
		}
		\block{Framework}{
		
			\begin{align*}
				\color{blue}{|N|=n, |A|=m} \quad &\text{voters, alternatives} \\
				\color{blue}{v_j =  {\succ_j}} \quad &\text{real preference order of the voter $j \in N$ }\\ 
				%	& \text{(connex,transitive, asymmetric relation)} \\
				\color{blue}{V=\{\profile \mid \profile=(v_1, \dots, v_n)\}} \quad & \text{set of complete preference profiles} \\
				\color{blue}{W=\{\w \mid \w=(w_1, \dots, w_m)\}} \quad & \text{set of scoring vectors} \\
				\color{blue}{s^{\profile, \w}(x) = \sum_{j \in N} w_{v_j(x)}} \quad & \text{score of alternative \textit{x} under the profile $\profile$ and weights $\w$} \\
				\color{blue}{\ppref_j} \quad & \text{partial preference order of the voter $j \in N$} \\
				\color{blue}{C(\ppref_j)=\set{{\succ} \mid {\ppref_j} \subseteq {\succ}}} \quad & \text{set of possible completions of $\ppref_j$} \\
				\color{blue}{\Co_W} \quad& \text{set of linear constraints given by the committee about $\w$}
%				 \\ &\text{the scoring vector $\w$}
			\end{align*}
%			\begin{align*}
%			\Regret^{\profile,\w}(x) & = \max_{y \in A} s^{\profile,\w}(y) - s^{\profile, \w}(x) \\
%			\PMR^{\pprofile,W}(x,y) & = \max_{\w \in W} \max_{\profile \in C(\pprofile)}s^{\profile,\w}(y) - s^{\profile, \w}(x) \\
%			\MR^{\pprofile,W}(x) & = \max_{y \in A} \PMR^{\pprofile,W}(x,y)\\
%			%& = \max_{\w \in W} \max_{\profile \in C(\profile)} \Regret(x, \profile, \w) \\
%			\MMR(\pprofile,W) & = \min_{x \in A} \MR^{\pprofile,W}(x) \\
%			x^{*}_{\pprofile,W} \in A^*_{\pprofile, W} & = \argmin_{x \in A}\MR^{\pprofile,W}(x)
%			\end{align*}
		}
		\block{Minimax Regret}{
			
%			\begin{align*}
%			|N|=n, |A|=m \quad &\text{voters, alternatives} \\
%			v_j = \succ_j \quad &\text{real preference order of the voter $j \in N$ }\\ 
%			& \text{(connex,transitive, asymmetric relation)} \\
%			V=\{\profile|\profile=(v_1, \dots, v_n)\} \quad & \text{set of complete preference profiles} \\
%			W=\{\w|\w=(w_1, \dots, w_m)\} \quad & \text{set of scoring vectors} \\
%			s^{\profile, \w}(x) = \sum_{j \in N} w_{v_j(x)} \quad & \text{score of alternative \textit{x} under the profile $\profile$ and weights $\w$} \\
%			\ppref_j \quad & \text{partial preference order of the voter $j \in N$} \\
%			C(\ppref_j)=\set{{\succ} \suchthat {\ppref_j} \subseteq {\succ}} \quad & \text{set of possible completions of $\ppref_j$} \\
%			\Co_W \quad& \text{set of linear constraints given by the committee about} \\ &\text{the scoring vector $\w$}
%			\end{align*}
			\begin{align*}
				\color{red}{\Regret^{\profile,\w}(x)} & = \max_{y \in A} s^{\profile,\w}(y) - s^{\profile, \w}(x) \\
				\text{is the regret of selecting $x$ as a winner} & \text{ instead of the optimal alternative under $\profile$ and $\w$}\\ 
%				\text{is the regret of selecting $x$ as a winner} & \text{ instead of choosing the optimal alternative}\\ \text{considering all possible} & \text{ realizations of $\profile$ and $\w$}\\
				\color{red}{\PMR^{\pprofile,W}(x,y)} & = \max_{\w \in W} \max_{\profile \in C(\pprofile)}s^{\profile,\w}(y) - s^{\profile, \w}(x) \\
				\text{is the worst-case } &	\text{loss of choosing $x$ instead of $y$} \\
%				\MR^{\pprofile,W}(x) & = \max_{y \in A} \PMR^{\pprofile,W}(x,y)\\
%%				\text{is the worst-case } &	\text{loss of $x$} \\
%				%& = \max_{\w \in W} \max_{\profile \in C(\profile)} \Regret(x, \profile, \w) \\
%				\MMR(\pprofile,W) & = \min_{x \in A} \MR^{\pprofile,W}(x) \\
			%	x^{*}_{\pprofile,W} \in A^*_{\pprofile, W} & = \argmin_{x \in A}\MR^{\pprofile,W}(x)
			\end{align*}
			\centering Max regret {\color{red}{$\MR^{\pprofile,W}(x)$}} is the worst-case loss of $x$ and the minimal max regret {\color{red}{$\MMR^{\pprofile,W}$}} is the regret associated to an optimal alternative.
		}
		
		\block{Elicitation strategies}{
			A function that, given our partial knowledge so far, returns a question that should be asked. 
			\begin{itemize}
				\item \textbf{Random}: it decides, with $1/2$ probability, whether to ask a question to the voters or to the committee, then it equiprobably draws one among the set of the possible questions;
			
				\item \textbf{Extreme completions}: it asks a question to the committee or to the voters depending on which uncertainty contributes the most to the regret;
			
				\item \textbf{Pessimistic}: it selects the question that leads to minimal regret in the worst case considering, and aggregating, both possible answers to each question; 
			
				\item \textbf{Two phase}: it asks a predefined, non adaptive sequence of $m-2$ questions to the committee and then it only asks questions about the voters.
			\end{itemize}
		}
	\column{0.5}
		\block{Motivation and approach}{
				\textbf{Who?}
				\begin{itemize}
					\item Imagine to be an \emph{external observer} helping with the voting procedure
				\end{itemize}
				\textbf{Why?}
				\begin{itemize}
					\item Voters: difficult or costly to order \emph{all} alternatives
%					Requiring voters to express \emph{full preference} orderings can be prohibitively \emph{costly}, especially for decisions with lots of alternatives
					\item Committee: difficult to specify a voting rule \emph{precisely} and abstractly
%					\emph{Difficult} for non-expert users \emph{to formalize} a voting rule on the basis of some generic preferences over a desired aggregation method
				\end{itemize}
				\textbf{How?}
				\begin{itemize}
					\item \emph{Minimax regret}: given the current knowledge, the alternatives with the lowest worst-case regret are selected as tied winners
				\end{itemize}		
				\textbf{Assumptions}
				\begin{itemize}
					\item Voters and committee have true preferences in mind
					\item The voting rule is a Positional Scoring Rule where the scoring vector $(w_1, \dots , w_m)$ is a convex sequence of weights and $w_1=1$, $w_m=0$ 
				\end{itemize}
		}
		\block{Question Types}{	
			\textbf{Questions to the voters}
			\begin{itemize}
				\item[] Comparison queries that ask a particular voter to compare two alternatives
				\[a \pref_j b \quad ?\]
			\end{itemize}
			\textbf{Questions to the committee}
			\begin{itemize}
				\item[] Queries relating the difference between the importance of consecutive ranks $r$ and $r+1$
				\[ w_{r} - w_{r+1} \geq \lambda (w_{r+1} - w_{r+2}) \quad ? \] 
			\end{itemize}
		}
		\block{Pairwise Max Regret Computation}{
			The computation of $\PMR^{\pprofile,W}(x,y)$ can be seen as a game in which an adversary can both:
			\begin{itemize}
				\item \textbf{complete the partial profile}\\
					\begin{center}
						\includegraphics[scale=1.015]{completion.png}
					\end{center}
					
				\item \textbf{choose a feasible weight vector}\\
				\centerline{\color{red}$\mathbf{(1,0,0)}$}
			\end{itemize}
			}
			\block{References}{
				\small{
					\bibliography{biblio}
					\bibliographystyle{abbrvnat}
				}
			}
		
\end{columns}

\end{document}

