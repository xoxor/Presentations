%nag warns a lot about tikzposter.
%\RequirePackage[l2tabu, orthodox]{nag}
\documentclass[blockverticalspace=3cm]{tikzposter}
\input{preamble/packages}
%\input{preamble/redac}
\input{preamble/math_basics}
%Decision Theory (MCDA and SC)
\newcommand{\allalts}{\mathscr{A}}
\newcommand{\allcrits}{\mathscr{C}}
\newcommand{\alts}{A}
\newcommand{\dm}{i}
\newcommand{\allF}{\mathscr{F}}
\newcommand{\allvoters}{\mathscr{N}}
\newcommand{\voters}{N}
\newcommand{\allprofs}{\boldsymbol{\mathcal{R}}}
\newcommand{\prof}{\boldsymbol{R}}
\newcommand{\linors}{\mathscr{L}(\allalts)}
%Thanks to https://tex.stackexchange.com/q/154549
	%\makeatletter
	%\def\@myRgood@#1#2{\mathrel{R^X_{#2}}}
	%\def\myRgood{\@ifnextchar_{\@myRgood@}{\mathrel{R^X}}}
	%\makeatother

%Deliberated Judgment
\newcommand{\prop}{\textcolor{red}{t}}
\newcommand{\allargs}{S^*}
\newcommand{\args}{S}
\newcommand{\ar}[1][]{%
	\textcolor{blue}{%
		\ifx\\#1\\%
			s
		\else
			s_#1
		\fi%
	}%
}
\newcommand{\ileadsto}{\textcolor{brown}{⇝}}
\newcommand{\ibeatse}{\textcolor{brown}{⊳_\exists}}
\newcommand{\nibeatse}{⋫_\exists}
\newcommand{\ibeatsst}{⊳_\forall}
\newcommand{\nibeatsst}{⋫_\forall}
\newcommand{\mleadsto}[1][\eta]{⇝_{#1}}
\newcommand{\mbeats}[1][\eta]{\textcolor{violet}{⊳_{#1}}}
\newcommand{\ibeatseinv}{⊳_\exists^{-1}}

%Logic
\newcommand{\ltru}{\texttt{T}}
\newcommand{\lfal}{\texttt{F}}


\input{preamble/draw}
\usepackage{tabularx}
\tikzposterlatexaffectionproofoff
\usepackage{rsc}
\renewcommand{\bibsection}{}
\usepackage{pdfpages}
\usepackage{graphics}
%\usepackage[backend=bibtex,style=authoryear]{biblatex}
%\addbibresource{biblio.bib}

\listfiles

\definecolor{myorange}{RGB}{250, 95, 0} 
\definecolor{myorange2}{RGB}{240, 100, 30} 
\definecolor{myburgundy}{RGB}{150, 0, 15} 

\colorlet{backgroundcolor}{white}
\colorlet{framecolor}{orange}
\colorlet{blocktitlebgcolor}{myorange2}

\newcommand{\profile}{\bm{v}}%(complete) profile
\newcommand{\pprofile}{{\bm{p}}}%partial profile
\newcommand{\w}{\bm{w}}
\newcommand{\W}{\mathcal{W}}
\newcommand{\Co}{\mathcal{C}}
\newcommand{\pw}{W}%our knowledge about the weights
\newcommand{\strat}[1]{\emph{#1}}
\newcommand{\ppref}{\succ^\text{p}}%partial pref
\newcommand{\pprefeq}{\succeq^\text{p}}%partial pref
\newcommand{\pref}{\succ}% pref
\DeclareMathOperator{\Regret}{Regret}
\DeclareMathOperator{\SCORE}{Score}
\DeclareMathOperator{\PMR}{PMR}
\DeclareMathOperator{\MR}{MR}
\DeclareMathOperator{\MMR}{MMR}

\newcommand*{\icimg}[1]{%
	\raisebox{-.3\baselineskip}{%
		\includegraphics[
		height=\baselineskip,
		width=\baselineskip,
		keepaspectratio,
		]{#1}%
	}%
}

\newcommand*{\icarr}[1]{%
	\raisebox{-0.4\baselineskip}{%
		\includegraphics[
		height=2.5\baselineskip,
		width=3\baselineskip,
		keepaspectratio,
		]{#1}%
	}%
}

\makeatletter
\def\title#1{\gdef\@title{\scalebox{\TP@titletextscale}{%
			\begin{minipage}[t]{\linewidth}
				\centering
				#1
				\par
				\vspace{0.5em}
			\end{minipage}%
		}}}
		\makeatother
%I find these settings useful in draft mode. Should be removed for final versions.
	%Which line breaks are chosen: accept worse lines, therefore reducing risk of overfull lines. Default = 200.
%		\tolerance=2000
	%Accept overfull hbox up to...
%		\hfuzz=2cm
	%Reduces verbosity about the bad line breaks.
%		\hbadness 5000
	%Reduces verbosity about the underful vboxes.
%		\vbadness=1300

\title{Simultaneous Elicitation of Committee and \\ Voters' Preferences}
\institute{$^1$ LAMSADE, Université Paris-Dauphine, Paris, France \\ $^2$ LIP6, Sorbonne Universit\'e, Paris, France}
\author{B. Napolitano$^1$, O. Cailloux$^1$ and P. Viappiani$^2$}


\begin{document}
%\includepdf[pages=-]{masque.pdf}
\maketitle[titletotopverticalspace=13cm]

\begin{columns}
	\column{0.5}
		\block{Scenario}{
				\begin{tikzpicture}[remember picture,overlay]
				\path (current page.north west) ++(1.5cm, -1cm) node[anchor=north west, inner sep=0] (first) {
					\includegraphics{masque.pdf}
				}; 
					\path (current page.south east) ++(-15cm, 1.5cm) node[anchor=south east, inner sep=0] (first) {
						\includegraphics[height=3.5cm]{dauphine_psl2018.png}
					}; 
					\path (current page.south east) ++(-2.5cm, 1.5cm) node[anchor=south east, inner sep=0] {
						\includegraphics[height=4cm]{LAMSADE95.jpg}
					};
				\end{tikzpicture}
				
		
%				\begin{tikzpicture}[remember picture,overlay]
%					\path (current page.south west) ++(1.5cm, 3cm) node[anchor=south west, text width=35cm] {
%						Beatrice Napolitano, Olivier Cailloux and Paolo Viappiani. \emph{Simultaneous Elicitation of Committee and Voters' Preferences}. Proceedings of RJCIA 2019.
%					};
%				\end{tikzpicture}%
	
			\textbf{Incompletely specified profile and positional scoring rule \\}
			\begin{tikzfigure}
				\includegraphics[scale=1.095]{setting3.png}
				%		\caption{.}
				%		\label{fig:b1}
			\end{tikzfigure}

			\textbf{Goal}
			\begin{itemize}
				\item[] Development of query strategies interleaving questions to the committee and to the voters in order to simultaneously elicit preferences and voting rule
			\end{itemize}
		}
%		\block{Framework}{
%			\begin{align*}
%				\color{blue}{|N|=n, |A|=m} \quad &\text{voters, alternatives} \\
%				\color{blue}{\ppref_j} \quad & \text{partial preference order of the voter $j \in N$} \\
%				\color{blue}{\Co_W} \quad& \text{set of linear  constraints given by the committee about $\w$} \\
%			\end{align*}
%			Given complete voters preferences $\profile$, a specific positional scoring rule, defined by a scoring vector $\w$, attributes a score {\color{blue} $s^{\profile, \w}$} to each alternative.
%			\bigskip \medskip
%		}
		\block{Questions}{	
			\textbf{Questions to the voters}
			\begin{itemize}
				\item[] Comparison queries that ask a particular voter to compare two alternatives $a$ and $b$
				\[a \pref_j b \quad ?\]
			\end{itemize}
			\textbf{Questions to the committee}
			\begin{itemize}
				\item[] Queries relating the difference between the importance of consecutive ranks $r$ and $r+1$
				\[ w_{r} - w_{r+1} \geq \lambda (w_{r+1} - w_{r+2}) \quad ? \] 
			\end{itemize}
		}
			\block{Elicitation strategies}{
				Given our knowledge so far, what is the next question that should be asked?
				\begin{itemize}
					\item \textbf{Random}: it decides, with $1/2$ probability, whether to ask a question to the voters or to the committee, then it equiprobably draws one among the set of the possible questions;
					
					\item \textbf{Extreme completions}: it asks a question to the committee or to the voters depending on which uncertainty contributes the most to the regret;
					
					\item \textbf{Pessimistic}: it selects the question that leads to minimal regret in the worst case considering, and aggregating, both possible answers to each question; 
					
					\item \textbf{Two phase}: it asks a predefined, non adaptive sequence of $m-2$ questions to the committee and then it only asks questions about the voters.
				\end{itemize}
			}
			 
			
			\block{References}{
				\small{
					\bibliography{biblio}
					\bibliographystyle{abbrvnat}
				}
			}
			
%			\begin{tikzpicture}[remember picture,overlay]
%				\path (current page.south west) ++(1.5cm, 1.5cm) node[anchor=south west, text width=27cm] {
%					Olivier Cailloux and Yves Meinard. \emph{A formal framework for deliberated judgment}. Under revision at Theory and Decision. \href{https://arxiv.org/abs/1801.05644}{arXiv:1801.05644 [cs.AI]}.
%				};
%			\end{tikzpicture}
		
	\column{0.5}
		\block{Motivation and approach}{
				\textbf{Who?}
				\begin{itemize}
					\item Imagine to be an \emph{external observer} helping with the voting procedure
				\end{itemize}
				\textbf{Why?}
				\begin{itemize}
					\item Voters: difficult or costly to order \emph{all} alternatives
					\item Committee: difficult to \emph{specify} a voting rule precisely and abstractly
				\end{itemize}
				\textbf{How?}
				\begin{itemize}
					\item \emph{Minimax regret}: given the current knowledge, the alternatives with the lowest worst-case regret are selected as tied winners
				\end{itemize}		
				\textbf{Context}
				\begin{itemize}
					\item $A$ Alternatives: $|A|=m$
					\item $N$ Voters: have a complete preferences profile $P = (\pref_j, \ j \in N ), \ P \in \mathcal{P} $ 
					\item Committee: has a (convex) scoring vector in mind $W=(\w_k,\ 1\leq k \leq m), \ W \in \mathcal{W}$ 
					\item $W$ defines a Positional Scoring Rule $f_W(P)\subseteq A$ using scores $s^{W,P}(a), \ \forall \ a \in A$
				\end{itemize}
		}
		\block{Our Knowledge}{
			The answers to these questions define $C_P$ and $C_W$ that is our knowledge about P and W 
			\medskip
			\begin{itemize}
				\item $C_P \subseteq \mathcal{P}$ constraints on the profile given by the voters
				\item $C_W \subseteq \mathcal{W}$ constraints on the voting rule given by the committee
			\end{itemize}
		}
	\block{Minimax Regret}{
		Given $C_P$ and $C_W$:
		\bigskip
		\coloredbox{the pairwise max regret $\color{red}{\PMR^{C_P,C_W}(a,b)}= \max_{P\in C_P, W \in C_W} s^{P,W}(b)-s^{P,W}(a) $ is the maximum difference of score between $a$ and $b$ under all possible realizations of the full profile {\em and} weights}
		\medskip
		We care about the worst case loss: \emph{maximal regret} between a chosen alternative $a$ and best real alternative $b$. \bigskip
		
		\centerline{\textbf{We select the alternative which \emph{minimizes} the maximal regret}}
	}	
		\block{Pairwise Max Regret Computation}{
			The computation of $\PMR^{C_P,C_W}( \icimg{salad.png},\icimg{aubergine.png})$ can be seen as a game in which an adversary
			\begin{itemize}
				\item \textbf{chooses a complete profile $\mathbf{P \in \mathcal{P}}$}\\
				\begin{center}
					\includegraphics[scale=0.765]{completion4.png}
				\end{center}
				\medskip
				\item \textbf{chooses a feasible weight vector $\mathbf{W \in \mathcal{W}}$}\\
				\centerline{\color{red}$\mathbf{(1,?,0)}$ \icarr{arrow.png} \color{red}$\mathbf{(1,0,0)}$}
			\end{itemize}
			\medskip
			in order to maximize the difference of scores. 
			\medskip
		}
			
		
\end{columns}

\end{document}

