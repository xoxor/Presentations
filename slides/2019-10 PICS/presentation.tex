\documentclass{beamer}
\usetheme{CambridgeUS}
\usecolortheme{beaver}
\usepackage{amsmath,amssymb,enumerate,amsthm}
\usepackage{bm}

\newcommand*{\icimg}[1]{%
	\raisebox{-.3\baselineskip}{%
		\includegraphics[
		height=\baselineskip,
		width=\baselineskip,
		keepaspectratio,
		]{#1}%
	}%
}


\makeatletter
\defbeamertemplate*{title page}{mydefault}[1][]
{
	\vbox{}
	\vfill
	\begin{centering}

%{\usebeamercolor[fg]{titlegraphic}\inserttitlegraphic\par}
		\begin{beamercolorbox}{titlegraphic}
				\usebeamerfont{titlegraphic}\inserttitlegraphic
		\end{beamercolorbox}%
			\vskip1em\par	
		\begin{beamercolorbox}[rounded=true, center, shadow=true, sep=8pt,#1]{title}
			\usebeamerfont{title}\inserttitle\par%
			\ifx\insertsubtitle\@empty%
			\else%
			\vskip0.5em%
			{\usebeamerfont{subtitle}\usebeamercolor[fg]{subtitle}\insertsubtitle\par}%
			\fi%     
		\end{beamercolorbox}%
		\vskip1em\par
		\begin{beamercolorbox}[sep=8pt,center,#1]{author}
			\usebeamerfont{author}\insertauthor
		\end{beamercolorbox}
		\begin{beamercolorbox}[sep=8pt,center,#1]{institute}
			\usebeamerfont{institute}\insertinstitute
		\end{beamercolorbox}
		\begin{beamercolorbox}[sep=8pt,center,#1]{date}
			\usebeamerfont{date}\insertdate
		\end{beamercolorbox}\vskip0.5em
		\begin{beamercolorbox}[sep=8pt,center,#1]{logo}
			\usebeamerfont{titlegraphic}\insertlogo
		\end{beamercolorbox}%
	\end{centering}
	\vfill
}
\setbeamertemplate{title page}[mydefault]
\makeatother



\titlegraphic{\includegraphics[width=50mm]{logo_dauphine} \hspace*{5.5cm} \includegraphics[width=7mm]{cnrs}}
\title[Ex-Ante vs Ex-Post Compromise]{Ex-Ante versus Ex-Post Compromise}
\institute[]{LAMSADE, Université Paris-Dauphine, Paris, France}
\author[O. Cailloux, B. Napolitano, R. Sanver]{O. Cailloux, B. Napolitano and R. Sanver}
\date[31 Oct 2019]{{\small $9th$ Murat Sertel Workshop on Economic Design} \\ \includegraphics[width=35mm]{LOGO_LAMSADE} }

\usepackage{tikz}
\usepackage{amsmath}
\usepackage{graphicx}

\definecolor{darkred}{rgb}{0.8,0,0}

\begin{document}

\beamertemplatenavigationsymbolsempty

\begin{frame}[plain]
\maketitle
\end{frame}

\addtocounter{framenumber}{-1}


\section{Introduction}
\subsection{Problem}

\begin{frame}
\frametitle{Introducing the problem}
\textbf{Setting}: Several voters express their preferences over a set of alternatives. 
\textbf{Goal}: Find a procedure determining a collective choice that promote a notion of compromise.
\end{frame}

\subsection{Related Works}
\begin{frame}
	\frametitle{Compromise rules}
	\begin{itemize}
		\item \textbf{plurality}:selects the alternatives considered as best by the highest number of voters. 
		%	In other words, it insists on a support of first and highest quality, disregarding the quantity of support this may lead to.
		\item \textbf{MVR}:MVR picks all alternatives receiving a majority support at the highest
		possible quality
		\item \textbf{MC}: picks alternatives receiving a majority support at the highest possible quality while ties are broken according to the quantity of support these receive
		%gives up from the quality of support, in order to ensure a majority support behind the selected alternatives.
		\item \textbf{FB}: bargainers fall back, in lockstep, to less and	less preferred alternatives until they reach a unanimous agreement. 
		\item \textbf{q-approval FB}:picks the alternatives which	receive the support of q voters at the highest possible quality – breaking ties according to the quantity of support
		%	Note that MC and FB winners are	particular cases of q-approval compromises, for q being respectively equal to majority and unanimity. Moreover for q = 1, q-approval compromises	coincide with the plurality rule (PR) winners
	\end{itemize}
\end{frame}

\subsection{Ex-Ante vs Ex-Post Perspective}
\begin{frame}
	\frametitle{Ex-Ante versus Ex-Post Perspective}
	\begin{block}{ex-ante compromise}
		impose over individuals a willingness to compromise but they do not ensure an outcome where everyone has effectively compromised. 
	\end{block}
	\begin{block}{ex-post compromise}
		favoring an outcome	where every voter gives up her most preferred positions if this
		increases equality. 
	\end{block}
\end{frame}

\subsection{Examples}
\begin{frame}
	\frametitle{Example 1}
	\begin{center}
		$
		\begin{array}{cccc}
		\mathbf{51} \quad &a&b&c\\
		\mathbf{49} \quad &c&b&a\\
		\end{array}
		$
	\end{center}
	\begin{itemize}
		\item plurality
		\item MVR
		\item MC
		\item FB
		\item q-approval FB 	$q\in \left\{ 1,..., \frac{n}{2} +1\right\} $
	\end{itemize}
\end{frame}

\begin{frame}
	\frametitle{Example 2}
	\begin{center}
		$
		\begin{array}{ccccc}
		\mathbf{26} \quad &a&b&c&d\\
		\mathbf{25} \quad &c&b&a&d\\
		\mathbf{z-51} \quad &d&b&a&c\\
		\mathbf{100-z} \quad &d&a&c&b\\
		\end{array}
		$
	\end{center}
	\begin{itemize}
		\item plurality
		\item MVR
		\item MC
		\item FB
		\item q-approval FB $q\in \left\{ \frac{n}{2},..., z\right\}$
	\end{itemize}
\end{frame}

\begin{frame}
	\frametitle{Idea}
	A social planner must choose between a world $x$ where individuals may sell their organs, and a world $y$ where they do not.
	\begin{center}
		$
		\begin{array}{ccc}
		& u_1 & u_2 \\
		\mathbf{x} \ & 1 & 100 \\
		\mathbf{y} \ & 0 & 0 \\
		\end{array}
		$
	\end{center}
	Even though $y$ is Pareto dominated in this example, the social planner might prefer $y$ to $x$. 
\end{frame}


\addtocounter{framenumber}{-1}
\begin{frame}[plain]
	\centering \color{darkred}\LARGE Thank You!
\end{frame}

%\addtocounter{framenumber}{-1}


\bibliographystyle{plain}
\scriptsize{\bibliography{biblio} }


\end{document}