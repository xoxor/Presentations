\documentclass{beamer}
\usetheme{CambridgeUS}
\usecolortheme{beaver}
\usepackage{amsmath,amssymb,enumerate,amsthm}
\usepackage{bm}

\newcommand{\R}{\mathbb{R}}
\newcommand{\paretopt}{\mathit{PO}}
\newcommand{\SPPd}{\Sigma^\text{PPd}}
\newcommand{\SAll}{\Sigma^\text{All}}
\newcommand{\SThreshold}{\Sigma_\text{threshold}}
\newcommand{\vpr}{\mathbf{v}}
\DeclareMathOperator*{\argmin}{arg\,min}
\newcommand{\suchthat}{\;\ifnum\currentgrouptype=16 \middle\fi|\;}

\newcommand*{\icimg}[1]{%
	\raisebox{-.3\baselineskip}{%
		\includegraphics[
		height=\baselineskip,
		width=\baselineskip,
		keepaspectratio,
		]{#1}%
	}%
}


\makeatletter
\defbeamertemplate*{title page}{mydefault}[1][]
{
	\vbox{}
	\vfill
	\begin{centering}

%{\usebeamercolor[fg]{titlegraphic}\inserttitlegraphic\par}
		\begin{beamercolorbox}{titlegraphic}
				\usebeamerfont{titlegraphic}\inserttitlegraphic
		\end{beamercolorbox}%
			\vskip1em\par	
		\begin{beamercolorbox}[rounded=true, center, shadow=true, sep=8pt,#1]{title}
			\usebeamerfont{title}\inserttitle\par%
			\ifx\insertsubtitle\@empty%
			\else%
			\vskip0.5em%
			{\usebeamerfont{subtitle}\usebeamercolor[fg]{subtitle}\insertsubtitle\par}%
			\fi%     
		\end{beamercolorbox}%
		\vskip1em\par
		\begin{beamercolorbox}[sep=8pt,center,#1]{author}
			\usebeamerfont{author}\insertauthor
		\end{beamercolorbox}
		\begin{beamercolorbox}[sep=8pt,center,#1]{institute}
			\usebeamerfont{institute}\insertinstitute
		\end{beamercolorbox}
		\begin{beamercolorbox}[sep=8pt,center,#1]{date}
			\usebeamerfont{date}\insertdate
		\end{beamercolorbox}\vskip0.5em
		\begin{beamercolorbox}[sep=8pt,center,#1]{logo}
			\usebeamerfont{titlegraphic}\insertlogo
		\end{beamercolorbox}%
	\end{centering}
	\vfill
}
\setbeamertemplate{title page}[mydefault]
\makeatother



\titlegraphic{\includegraphics[width=50mm]{logo_dauphine} \hspace*{5.5cm} \includegraphics[width=7mm]{cnrs}}
\title[Ex-Ante vs Ex-Post Compromise]{Ex-Ante versus Ex-Post Compromise}
\institute[]{LAMSADE, Université Paris-Dauphine, Paris, France}
\author[O. Cailloux, B. Napolitano, R. Sanver]{O. Cailloux, B. Napolitano and R. Sanver}
\date[31 Oct 2019]{{\small $9th$ Murat Sertel Workshop on Economic Design} \\ \includegraphics[width=35mm]{LOGO_LAMSADE} }

\usepackage{tikz}
\usepackage{amsmath}
\usepackage{graphicx}

\definecolor{darkred}{rgb}{0.8,0,0}

\begin{document}

\beamertemplatenavigationsymbolsempty

\begin{frame}[plain]
\maketitle
\end{frame}

\addtocounter{framenumber}{-1}


\section{Introduction}
\subsection{Problem}

\begin{frame}
\frametitle{Introducing the problem}
\textbf{Setting}: Several voters express their preferences over a set of alternatives \vspace{6mm}

\onslide<2->{\textbf{Goal}: Find a procedure determining a collective choice that promote a notion of compromise}
\end{frame}

\subsection{Related Works}
\begin{frame}
	\frametitle{Compromise rules}
	\begin{itemize}
		\item<1-> \textbf{Plurality}: selects the alternatives considered as best by the highest number of voters 
		%	In other words, it insists on a support of first and highest quality, disregarding the quantity of support this may lead to.
		\item<2-> \textbf{Median Voting Rule}: picks all alternatives receiving a majority of support at the highest possible quality
		\item<3-> \textbf{Majoritarian Compromise}: MVR and ties are broken according to the quantity of support these receive
		%gives up from the quality of support, in order to ensure a majority support behind the selected alternatives.
		\item<4-> \textbf{Fallback Bargaining}: bargainers fall back to less and less preferred alternatives until they reach a unanimous agreement 
		\item<5-> \textbf{q-approval FB}: picks the alternatives which receive the support of q voters at the highest possible quality, breaking ties according to the quantity of support
		%	Note that MC and FB winners are	particular cases of q-approval compromises, for q being respectively equal to majority and unanimity. Moreover for q = 1, q-approval compromises	coincide with the plurality rule (PR) winners
	\end{itemize}
\end{frame}

\subsection{Examples}
\begin{frame}
	\frametitle{Example 1}
	$|N|=100, A=\{a,b,c\}$
	\begin{center}
		$
		\begin{array}{cccc}
		\mathbf{51} \quad &a&b&c\\
		\mathbf{49} \quad &c&b&a\\
		\end{array}
		$
	\end{center}
	\begin{itemize}
		\item<1-> Plurality: $\{a\}$
		\item<2-> MVR: $\{a\}$
		\item<3-> MC: $\{a\}$
		\item<4-> FB: $\{b\}$
		\item<5-> q-approval FB 	$q\in \left\{ 1,..., \frac{n}{2} +1\right\} $: $\{a\}$
	\end{itemize}
\end{frame}

\begin{frame}
	\frametitle{Example 2}
	$|N|=100, A=\{a,b,c,d\}, z\in \left\{ 64,..., 99\right\}$
	\begin{center}
		$
		\begin{array}{ccccc}
		\mathbf{26} \quad &a&b&c&d\\
		\mathbf{25} \quad &c&b&a&d\\
		\mathbf{z-51} \quad &d&b&a&c\\
		\mathbf{100-z} \quad &d&a&c&b\\
		\end{array}
		$
	\end{center}
	\begin{itemize}
		\item<1-> Plurality: $\{d\}$
		\item<2-> MVR: for $z<76$ $\{a,b\}$ 
		\only<3-3>{	\begin{center}
				$
				\begin{array}{ccc}
				&1^\circ&2^\circ \\
				\mathbf{a} \quad &26&126-z\\
				\mathbf{b} \quad &0&z\\
				\mathbf{c} \quad &25&25\\
				\mathbf{d} \quad &49&49\\
				\end{array}
				$
			\end{center} }
		\only<4-4>{	\begin{center}
			$
			\begin{array}{ccc}
			&1^\circ&2^\circ \\
			\mathbf{a} \quad &26&51\\
			\mathbf{b} \quad &0&75\\
			\mathbf{c} \quad &25&25\\
			\mathbf{d} \quad &49&49\\
			\end{array}
			$
		\end{center} }
		\onslide<5->, for $z\geq76$ $\{b\}$ 
		\only<5-5>{	\begin{center}
				$
				\begin{array}{ccc}
				&1^\circ&2^\circ \\
				\mathbf{a} \quad &26&50\\
				\mathbf{b} \quad &0&76\\
				\mathbf{c} \quad &25&25\\
				\mathbf{d} \quad &49&49\\
				\end{array}
				$
			\end{center} }	
		\item<6-> MC: $\{b\}$
		\item<7-> FB: $\{a\}$
		\item<8-> q-approval FB $q\in \left\{ \frac{n}{2},..., z\right\}$: $\{b\}$
	\end{itemize}
\end{frame}

\begin{frame}
 \frametitle{Motivation}
 $|N|=2, |A|=2k+2$
 \begin{center}
 	$
 	\begin{array}{cc}
 	\mathbf{i_1}& \mathbf{i_2} \\
 	x &b_1\\
 	a_1 &\cdot\\
 	\cdot &\cdot\\
 	\cdot &b_{k-1}\\
 	\cdot &y\\
 	a_k &x\\
 	y &b_k\\
 	b_1 &a_1\\
 	\cdot &\cdot\\
 	\cdot &\cdot\\
 	b_k &a_k\\
 	\end{array}
 	$
 	\end{center}
\end{frame}

\subsection{Ex-Ante vs Ex-Post Perspective}
\begin{frame}
	\frametitle{Ex-Ante versus Ex-Post Perspective}
	\begin{block}{ex-ante compromise}
		imposes over individuals a willingness to compromise but it does not ensure an outcome where everyone has effectively compromised 
	\end{block}
	\begin{block}{ex-post compromise}
		favors an outcome where every voter gives up her most preferred positions if this	increases equality 
	\end{block}
\end{frame}

\section{Ex-Post Compromise}
\subsection{Cardinal Compromise}
\begin{frame}
	\frametitle{Cardinal Compromise}
	\framesubtitle{Setting}
	\begin{description}[$u_i \in U(A) \subseteq \R^A$]
		\item[$A$] alternatives
		\item[$N$] voters
		\item[$u_i \in \R^A$] utility function depending on the rank of voter $i$ 
	\end{description}
	\onslide<2-> \begin{block}{}
		{\color{blue}$\lambda_{i}^u(x) = \max_{a \in A} u_i(a) - u_{i}(x)$} represents the loss of utility for the voter $i$ if the alternative $x$ is elected instead of her favorite one; and {\color{blue}$\lambda ^{u}(x)$} represents the vector of these losses
	\end{block}
\end{frame}

\begin{frame}
\frametitle{Spread Measure}
$\sigma : \R_{+}^{N}\longrightarrow \R_{+}$
\vspace{5mm}
\onslide<2-> \begin{block}{Pure Equality Recognition}
	\[ r_i=r_j \ \forall i,j \in N \Rightarrow \sigma(r)=0 \quad r \in \R_{+}^{N}\]
\end{block}
\onslide<3-> \begin{block}{Pairwise Pareto Dominance}
	\[\left[\left\vert r_{i}-r_{j}\right\vert \leq \left\vert s_{i}-s_{j}\right\vert \ \forall i, j\in N\right] \Rightarrow \sigma (r)\leq \sigma (s) \quad r,s\in \R_{+}^{N}\] 
\end{block}

\end{frame}

\begin{frame}
	\frametitle{Spread Measure}
	\framesubtitle{Example}
	\[\bar{r}=\frac{\sum_{i=1}^{n}r_i}{n}\]
	\[\sigma_{avg}(r)= \sum_{i=1}^{n}|\bar{r}-r_i| \]
	\onslide<2-> Examples:
	\begin{align*}
	s=(3,3,3,3) \qquad & \sigma_{avg}(s)= {\scriptstyle\sum_{i=1}^{4}(3-3)}=0 \\
	t=(1,2,3,4) \qquad & \sigma_{avg}(t)= {\scriptstyle |2.5-1|+|2.5-2|+|2.5-3|+|2.5-4|}=4 \\
%	t=(1,1,2,2) \qquad & \sigma_{avg}(t)= {\scriptstyle |1.5-1|+|1.5-1|+|1.5-2|+|1.5-2|}=2 \\
%	w=(2,2,3,3) \qquad & \sigma_{avg}(w)= |2.5-2|+|2.5-2|+|2.5-3|+|2.5-3|=2 \\
	w=(1,3,5,7) \qquad & \sigma_{avg}(w)= 
	{\scriptstyle|4-1|+|4-3|+|4-5|+|4-7|}=8
	\end{align*}
\end{frame}


%\begin{frame}
%	\frametitle{Spread Measure}
%	\framesubtitle{Gini Coefficient}
%	\[\sigma_{avg}(r)= \frac{\sum_{i=1}^{n}\sum_{j=1}^{n}|r_i-r_j|}{2n\sum_{i=1}^{n}|r_{max}-r_i|}\]
%	Examples:
%	\begin{align*}
%		s=(3,3,3,3) \qquad & \sigma_{G}(s)= \frac{0}{2 \cdot 4 \cdot 12}=0 \\
%		t=(1,1,2,2) \qquad & \sigma_{G}(t)= \frac{(1+1)\cdot 4}{2 \cdot 4 \cdot 6}=\frac{1}{6} \\
%		w=(2,2,3,3) \qquad & \sigma_{G}(w)= \frac{(1+1)\cdot 4}{2 \cdot 4 \cdot 10}=\frac{1}{10} \\
%		w=(1,3,5,7) \qquad & \sigma_{G}(w)= \frac{{\scriptscriptstyle(2+4+6)+(2+2+4)+(4+2+2)+(6+4+2)}}{2 \cdot 4 \cdot 16}=\frac{40}{128}=0.31
%	\end{align*}
%\end{frame}
\begin{frame}
	\frametitle{Cardinal Compromise}
	\begin{description}[$PO(u) \qquad $]
		\item[$\mathcal{U}$] set of injective utility functions defined	over A
		\item[$PO(u)$] set of Pareto optimal alternatives at {\color{blue}$u \in \mathcal{U}$}
		\item[$\lambda ^{u}(x)$] losses vector when electing the alternative $x$
		\item[$\sigma$] spread measure
	\end{description}
	\onslide<2-> \begin{block}{}
		\[ C^{\sigma }(u) = \{ x\in \paretopt(u):\sigma (\lambda ^{u}(x))\leq \sigma (\lambda ^{u}(y)),  \forall y\in A\} \]
	\end{block}
\end{frame}

\begin{frame}
	\frametitle{Cardinal Compromise}
	\framesubtitle{Example}
	$|N|=100, A=\{a,b,c\}$
	\begin{center}
		$
		\begin{array}{cccc}
		\mathbf{51} \quad &a&b&c\\
		\mathbf{49} \quad &c&b&a\\
		\end{array}
		$
	\end{center}
	\only<2-2>{\begin{center}
		$
		\begin{array}{ccccccc}
		& i_{1}& \dots& i_{51}& i_{52}&\dots&i_{100} \\
		\mathbf{a} \quad &2&\dots&2&0&\dots&0\\
		\mathbf{b} \quad &1&\dots&1&1&\dots&1\\
		\mathbf{c} \quad &0&\dots&0&2&\dots&2
		\end{array}
		$
	\end{center}}
	\onslide<3->{\begin{center}
			$
			\begin{array}{ccccccc}
			& i_{1}& \dots& i_{51}& i_{52}&\dots&i_{100} \\
			\lambda(a) = (&0,&\dots&0,&2,&\dots&2)\\
			\lambda(b) = (&1,&\dots&1,&1,&\dots&1)\\
			\lambda(c) = (&2,&\dots&2,&0,&\dots&0)
			\end{array}
			$
	\end{center}}
	\onslide<4->{ $\sigma_{avg}(\lambda(a))=99.96, \quad \sigma_{avg}(\lambda(b))=0, \quad \sigma_{avg}(\lambda(c))=100.04$}
	\onslide<5->{ \[ C^{\sigma_{avg} }(u) = b \]}
\end{frame}

\subsection{Ordinal Compromise}
\begin{frame}
	\frametitle{Ordinal Compromise}
	\begin{description}[$\mathbf{v}: L(A)^N \rightarrow U(A)^N$]
		\item[$P_i \in L(A)$] linear order over $A$ which represents the preference of $i\in N$
		\item[$v:\left\{ 1,..., m\right\} \rightarrow \R$] utility assignment
		\item[$v_{P_{i}} \in \R^A $] utility function for $P_{i}\in L(A)$ induced by $v$
		\item[$\mathbf{v}: L(A)^N \rightarrow \mathcal{U}$] function mapping a profile of ordinal preferences to a utility profile 
	\end{description}
	\onslide<2-> \begin{block}{}
		\[(C^{\sigma }\circ \vpr)(\{P_i, i \in N\}) = C^{\sigma}(\{v_{P_i}, i \in N\})\]
	\end{block}
\end{frame}

\begin{frame}
	\frametitle{Ordinal Compromise}
	$\SAll = \R_+^{\R_+^N}$ the set of all spread measures
	\vspace{5mm}
	\begin{block}{UA-independence}
		A class of spread measure $\Sigma \subseteq \SAll$ is UA-independent iff, given any $\sigma \in \Sigma $ and any two UAs $v$ and $v^{\prime }$, there exists a $\sigma^\prime\in \Sigma $ such that $C^{\sigma}\circ \vpr = C^{\sigma ^{\prime}} \circ \vpr^\prime$
	\end{block}
\end{frame}

\begin{frame}
	\frametitle{Ordinal Compromise}
	\framesubtitle{UA-independence}
	$\SPPd \subseteq \SAll$ the class of spread measures that satisfy PPd
	\begin{block}{Proposition 1:}
		$\SPPd$ is not UA-independent
	\end{block}
	\only<2-3>{\begin{center}
			\vspace{5mm}
		$
		\begin{array}{ccccccccc}
		\mathbf{i_1} \quad &x&a_1&a_2&a_3&y&b_1&b_2&b_3\\
		\mathbf{i_2} \quad &b_1&b_2&y&x&b_3&a_1&a_2&a_3\\
		\end{array}
		$
	\end{center}}
	\only<3-3>{\begin{center}
			\vspace{5mm}
		$
		\begin{array}{cccccccccc}
		&k=& 1&2&3&4&5&6&7&8 \vspace{1.5mm}\\
		v(k) & &7&6&5&4&3&2&1&0\\
		v^\prime(k) & &1000&999&998&997&3&2&1&0\\
		\end{array}
		$
	\end{center}}
	\only<4-4>{\begin{center}
		$
		\begin{array}{ccccc}
		& v_{P_1}(\cdot) & v_{P_2}(\cdot) & & \lambda^P(\cdot) \\
		x \quad &7&4 & \quad & (0,3)\\
		y \quad &3&5 & \quad & \color{red}(4,2)\\
		a_1 \quad &6&2 & \quad & (1,5)\\
		a_2 \quad &5&1 & \quad & (2,6)\\
		a_3 \quad &4&0 & \quad & (3,7)\\
		b_1 \quad &2&7 & \quad & (5,0)\\
		b_2 \quad &1&6 & \quad & (6,1)\\
		b_3 \quad &0&3 & \quad & (7,4)\\
		\end{array}
		$
	\end{center}
	$C^{\sigma}(v)\in \{y\}$}
	\onslide<5->{\begin{center}
		$
		\begin{array}{ccccc}
		& v^\prime_{P_1}(\cdot) & v^\prime_{P_2}(\cdot) & & \lambda^{\prime P}(\cdot) \\
		x \quad &1000&997 & \quad & \color{red}(0,3)\\
		y \quad &3&998 & \quad & (997,2)\\
		a_1 \quad &999&2 & \quad & (1,998)\\
		a_2 \quad &998&1 & \quad & (2,999)\\
		a_3 \quad &997&0 & \quad & (3,1000)\\
		b_1 \quad &2&1000 & \quad & (998,0)\\
		b_2 \quad &1&999 & \quad & (999,1)\\
		b_3 \quad &0&3 & \quad & \color{red}(1000,997)\\
		\end{array}
		$
	\end{center}
	$C^{\sigma}(v^\prime)\in \{x,b_3\}$}
\end{frame}

\begin{frame}
	\frametitle{Ordinal Compromise}
	\framesubtitle{UA-independence}
	$\SThreshold= \{ \sigma^k , k \in \R\}$
	where $\sigma^k (\lambda)= \# \{ i \in N \suchthat \lambda_i \geq k\}$
	\begin{block}{Proposition 2:}
		$\SThreshold$ is UA-independent
	\end{block}
\end{frame}

\begin{frame}
	\frametitle{Ordinal Compromise}
	\framesubtitle{UA-independence}
	\begin{block}{Proposition 3:}
		$\sigma^k \in \SThreshold$ fails Pure Equality Recognition and Pairwise Pareto Dominance
	\end{block}
	\only<2-2>{
		\vspace{5mm}
		$\lambda(x)=(4,4,4,4) \qquad \sigma^3(\lambda(x))=4$}
	\onslide<3->{
	\begin{center}
		$
		\begin{array}{ccccc}
		\mathbf{i_1} \quad &a&d&b&c\\
		\mathbf{i_2} \quad &b&c&d&a\\
		\end{array}
		$
	\end{center}
	}
	\onslide<4->{
			$v$ assigns the utility values $10,2,1,0$ respectively to the ranks $1,2,3,4$}
	\onslide<5->{
		\begin{center}
			$
			\begin{array}{ccccccc}
			&&&&&& \sigma^1(\lambda^P(\cdot))\\
			\lambda^P(a) \ =  &(&0&10&) & \quad & 1\\
			\lambda^P(b) \ =  &(&9&0&) &  & 1\\
			\lambda^P(c) \ =  &(&10&8&) &  & 2\\
			\lambda^P(d) \ =  &(&8&9&) &  & 2\\
			\end{array}
			$
		\end{center}}
	
\end{frame}

\section{Conclusions}
\begin{frame}
	\frametitle{Questions}
	\begin{itemize}
		\item <1-> Which properties define a "good" spread measure?
		\item <2-> How to characterize those class of social choice rules? 
		\item <3-> What are the relationships between them and already existing rules?
		\item <4-> What are the consequences of allowing the utility assignments to vary among individuals?
		\item <5-> Is it reasonable to drop the Pareto Optimality constraint?
		\item <6-> $\dots$
	\end{itemize}
\end{frame}

\appendix
\begin{frame}
	\frametitle{Appendix I \\ Minimal Liberty}
	A social planner must choose between a world $x$ where individuals may sell their organs, and a world $y$ where they do not
	\begin{center}
		$
		\begin{array}{ccc}
		& u_1 & u_2 \\
		\mathbf{x} \ & 1 & 100 \\
		\mathbf{y} \ & 0 & 0 \\
		\end{array}
		$
	\end{center}
	Even though $y$ is Pareto dominated, the social planner might prefer $y$ to $x$ 
\end{frame}


\addtocounter{framenumber}{-1}
\begin{frame}[plain]
	\centering \color{darkred}\LARGE Thank You!
\end{frame}

%\addtocounter{framenumber}{-1}


\bibliographystyle{plain}
\scriptsize{\bibliography{biblio} }


\end{document}