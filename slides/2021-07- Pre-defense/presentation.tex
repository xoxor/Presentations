\documentclass{beamer}
\usetheme{CambridgeUS}
\usecolortheme{beaver}

\newcommand\hmmax{0}
\newcommand\bmmax{0}

\usepackage{stmaryrd}

\usepackage{amsmath,amssymb,enumerate,amsthm}
\usepackage{bm}
\usepackage{graphicx}

\usepackage{algorithm, algpseudocode}
\usepackage[strict]{siunitx}
\usepackage{hyperref}
\usepackage{mathrsfs}

\usepackage{bm}
\usepackage{empheq}
\usepackage{cleveref}
\usepackage{xcolor}
\usepackage{textcomp}
\usepackage{siunitx}
\usepackage{booktabs}
\usepackage{caption}
\usepackage{tikz}
\usetikzlibrary{decorations.pathreplacing,calligraphy}
\usepackage{pgfplots}
\input{preamble/math_basics}
%Decision Theory (MCDA and SC)
\newcommand{\allalts}{\mathscr{A}}
\newcommand{\allcrits}{\mathscr{C}}
\newcommand{\alts}{A}
\newcommand{\dm}{i}
\newcommand{\allF}{\mathscr{F}}
\newcommand{\allvoters}{\mathscr{N}}
\newcommand{\voters}{N}
\newcommand{\allprofs}{\boldsymbol{\mathcal{R}}}
\newcommand{\prof}{\boldsymbol{R}}
\newcommand{\linors}{\mathscr{L}(\allalts)}
%Thanks to https://tex.stackexchange.com/q/154549
	%\makeatletter
	%\def\@myRgood@#1#2{\mathrel{R^X_{#2}}}
	%\def\myRgood{\@ifnextchar_{\@myRgood@}{\mathrel{R^X}}}
	%\makeatother

%Deliberated Judgment
\newcommand{\prop}{\textcolor{red}{t}}
\newcommand{\allargs}{S^*}
\newcommand{\args}{S}
\newcommand{\ar}[1][]{%
	\textcolor{blue}{%
		\ifx\\#1\\%
			s
		\else
			s_#1
		\fi%
	}%
}
\newcommand{\ileadsto}{\textcolor{brown}{⇝}}
\newcommand{\ibeatse}{\textcolor{brown}{⊳_\exists}}
\newcommand{\nibeatse}{⋫_\exists}
\newcommand{\ibeatsst}{⊳_\forall}
\newcommand{\nibeatsst}{⋫_\forall}
\newcommand{\mleadsto}[1][\eta]{⇝_{#1}}
\newcommand{\mbeats}[1][\eta]{\textcolor{violet}{⊳_{#1}}}
\newcommand{\ibeatseinv}{⊳_\exists^{-1}}

%Logic
\newcommand{\ltru}{\texttt{T}}
\newcommand{\lfal}{\texttt{F}}



\newcommand{\profile}{\bm{v}}%(complete) profile
\newcommand{\pprofile}{{\bm{p}}}%partial profile
\newcommand{\w}{\bm{w}}
\newcommand{\W}{\mathcal{W}}
\newcommand{\Co}{\mathcal{C}}
\newcommand{\pw}{W}%our knowledge about the weights
\newcommand{\strat}[1]{\emph{#1}}
\newcommand{\ppref}{\succ^\text{p}}%partial pref
\newcommand{\pprefeq}{\succeq^\text{p}}%partial pref
\DeclareMathOperator{\Regret}{Regret}
\DeclareMathOperator{\SCORE}{Score}
\DeclareMathOperator{\PMR}{PMR}
\DeclareMathOperator{\MR}{MR}
\DeclareMathOperator{\MMR}{MMR}

\newcommand*{\icimg}[1]{%
	\raisebox{-.3\baselineskip}{%
		\includegraphics[
		height=\baselineskip,
		width=\baselineskip,
		keepaspectratio,
		]{#1}%
	}%
}

\newcommand*{\icarr}[1]{%
	\raisebox{-0.4\baselineskip}{%
		\includegraphics[
		height=2.5\baselineskip,
		width=3\baselineskip,
		keepaspectratio,
		]{#1}%
	}%
}

\makeatletter
\defbeamertemplate*{title page}{mydefault}[1][]
{
	\vbox{}
	\vfill
	\begin{centering}

%{\usebeamercolor[fg]{titlegraphic}\inserttitlegraphic\par}
		\begin{beamercolorbox}{titlegraphic}
				\usebeamerfont{titlegraphic}\inserttitlegraphic
		\end{beamercolorbox}%
			\vskip1em\par	
		\begin{beamercolorbox}[rounded=true, center, shadow=true, sep=8pt,#1]{title}
			\usebeamerfont{title}\inserttitle\par%
			\ifx\insertsubtitle\@empty%
			\else%
			\vskip0.5em%
			{\usebeamerfont{subtitle}\usebeamercolor[fg]{subtitle}\insertsubtitle\par}%
			\fi%     
		\end{beamercolorbox}%
		\vskip1em\par
		\begin{beamercolorbox}[sep=8pt,center,#1]{author}
			\usebeamerfont{author}\insertauthor
		\end{beamercolorbox}
		\begin{beamercolorbox}[sep=8pt,center,#1]{institute}
			\usebeamerfont{institute}\insertinstitute
		\end{beamercolorbox}
		\begin{beamercolorbox}[sep=8pt,center,#1]{date}
			\usebeamerfont{date}\insertdate
		\end{beamercolorbox}\vskip0.5em
		\begin{beamercolorbox}[sep=8pt,center,#1]{logo}
			\usebeamerfont{titlegraphic}\insertlogo
		\end{beamercolorbox}%
	\end{centering}
	\vfill
}
\setbeamertemplate{title page}[mydefault]
\makeatother



\titlegraphic{\includegraphics[width=50mm]{logo_dauphine} \hspace*{5.5cm} \includegraphics[width=7mm]{cnrs}}
\title[Elicitation and explanation for voting rules]{Elicitation and explanation for voting rules}
%\subtitle{Proposal: ``Elicitation and Explanation for Voting Rules''}
\author[Beatrice Napolitano]{\textbf{Beatrice Napolitano} \\
	Supervisors: Remzi Sanver, Olivier Cailloux}
\date[06 July 2021]{ Pré-soutenance de thèse \\ 06 July 2021 \\ \includegraphics[width=35mm]{LOGO_LAMSADE} }

\usepackage{tikz}
\usepackage{amsmath}
\usepackage{graphicx}

\definecolor{darkred}{rgb}{0.8,0,0}

\begin{document}

\beamertemplatenavigationsymbolsempty

\begin{frame}[plain]
\maketitle
\end{frame}

\addtocounter{framenumber}{-1}


\begin{frame}
	\frametitle{Outline}
	\tableofcontents[hideallsubsections, sectionstyle=shaded/show]
\end{frame}

\AtBeginSection{
	\begin{frame}
		\frametitle{Outline}
		\tableofcontents[currentsection, hideallsubsections]
	\end{frame}
}

\section{Notation}
\subsection{Context}
\begin{frame}
	\frametitle{Context}	
	\begin{description}[$\prof=(\succ_{1},\dots,\succ_{n}) \in \linors^\voters$]
		\item [$\allalts$] set of alternatives, $|\allalts|=m$
		\item [$\voters$] set of voters, $|\voters|=n$
		\item [$\linors$] set of all linear orderings given $\allalts$
		\item [${\prefi} \in \linors$] preference ranking of voter $i \in \voters$
		\item [$\prof=(\succ_{1},\dots,\succ_{n}) \in \linors^\voters$] a profile
		\item [$\powersetz{\allalts}$] possible winners (the non-empty subsets of $\allalts$)
		\item [$f: \linors^\voters \rightarrow \powersetz{\allalts}$] an SCR
	\end{description}
\end{frame}

\section{Compromising as an equal loss principle}
\subsection{Context}
\begin{frame}
	\frametitle{Context}
	\framesubtitle{Introducing the problem}
	\textbf{Setting}: Several voters express their preferences over a set of alternatives \vspace{6mm}
	
	\onslide<2->{\textbf{Goal}: Find a procedure determining a collective choice that promotes a notion of compromise}
\end{frame}

\subsection{Related Works}
\begin{frame}
	\frametitle{Context}
	\framesubtitle{Related Works}
	\begin{itemize}
		\item<1-> \textbf{Plurality}: selects the alternatives considered as best by the highest number of voters 
		%	In other words, it insists on a support of first and highest quality, disregarding the quantity of support this may lead to.
		\item<2-> \textbf{Median Voting Rule}: picks all alternatives receiving a majority of support at the highest possible quality (Bassett and Persky, 1999 \cite{Bassett1999})
		\item<3-> \textbf{Majoritarian Compromise}: MVR and ties are broken according to the quantity of support received (Sertel, 1986 \cite{Sertel1986})
		%gives up from the quality of support, in order to ensure a majority support behind the selected alternatives.
		\item<4-> \textbf{Fallback Bargaining}: bargainers fall back to less and less preferred alternatives until they reach a unanimous agreement (Brams and Kilgour, 2001 \cite{Brams2001})
		\item<5-> \textbf{q-approval FB}: picks the alternatives which receive the support of q voters at the highest possible quality, breaking ties according to the quantity of support
		%	Note that MC and FB winners are	particular cases of q-approval compromises, for q being respectively equal to majority and unanimity. Moreover for q = 1, q-approval compromises	coincide with the plurality rule (PR) winners
	\end{itemize}
\end{frame}

\begin{frame}
	\frametitle{Context}
	\framesubtitle{Motivation: A simple example}
	$n=100, \allalts=\{a,b,c\}$
	\begin{center}
		$
		\begin{array}{cccccc}
			\mathbf{51} \quad &a&\succ & b & \succ&c\\
			\mathbf{49} \quad &c&\succ & b & \succ&a\\
		\end{array}
		$
	\end{center}
	\begin{itemize}
		\item<2-> Plurality: $\{a\}$
		\item<3-> MVR: $\{a\}$
		\item<4-> MC: $\{a\}$
		\item<5-> FB: $\{b\}$
		\item<6-> FB$_q$, $q\in \left\{ 1,..., \frac{n}{2} +1\right\} $: $\{a\}$
	\end{itemize}
	\vspace{0.5cm}
	\visible<7->{\centering Does $b$ seem a better compromise?}
\end{frame}

\subsection{Notation}

\begin{frame}
	\frametitle{Notation}
	\framesubtitle{Losses}
	%Our compromises attempt to equalize “losses”
	\begin{description}[$\lambda_{\prof}: \allalts \rightarrow \intvl{0, m - 1}^\voters$]
		\item<1-> [$\lambda_{\prof}: \allalts \rightarrow \intvl{0, m - 1}^\voters$] a loss vector
	\end{description}
	\vspace{4cm} 
	%Thus, the spread of $l$ gets its lowest value $0$ in case of perfect equality and only in this case.
\end{frame}

\begin{frame}
	\frametitle{Notation}
	\framesubtitle{Losses}
		\begin{center}
		\begin{tikzpicture}
			\path node (P) {$\prof$};
			\path (P.south) node[anchor=north] (profile) {$%
				\begin{array}{r@{\hspace{1mm} : \hspace{1mm}}l}
					v_1 & a \pref b \pref c\\%
					v_2 & c \pref b \pref a\\%
				\end{array}%
				$};
			\path (P.center) ++ (3cm, 0) node (L) {$\lambda_{\prof}$};
			\path (L.south) node[anchor=north] (losses) {$%
				\begin{array}{r@{\hspace{1mm} : \hspace{1mm}}l}
					a & (0, 2)\\%
					b & (1, 1)\\%
					c & (2, 0)\\%
				\end{array}%
				$};
		\end{tikzpicture}
		\end{center}
	\bigskip
	
	Given $\prof = (\prefi)_{i \in N}$:
	\begin{itemize}
		\item $\lambda_{\prefi}(x) = \card{\set{y \in \allalts \suchthat y \prefi x}} \in \intvl{0, m - 1}$ the loss of $i$ when choosing $x \in \allalts$ instead of her favorite alternative
		\item $\lambda_{\prof}(x)$ associates to each voter her loss when choosing $x$
	\end{itemize}
\end{frame}

\begin{frame}
	\frametitle{Notation}
	\framesubtitle{Losses}
	%Our compromises attempt to equalize “losses”
	\begin{description}[$\lambda_{\prof}: \allalts \rightarrow \intvl{0, m - 1}^\voters$]
		\item<1-> [$\lambda_{\prof}: \allalts \rightarrow \intvl{0, m - 1}^\voters$] a loss vector
		\item<2-> [$\sigma: \intvl{0, m - 1}^N \rightarrow \R^+$] a spread measure
	\end{description}
	\bigskip
	\onslide<3-> \begin{block}{}
		$\Sigma$ is the set of spread measures $\sigma$ such that 
		\[ \sigma(l)=0 \iff l_{i}=l_{j}, \ \forall i,j\in N, \quad \forall l\in\alllosses \].
	\end{block}
	%Thus, the spread of $l$ gets its lowest value $0$ in case of perfect equality and only in this case.
\end{frame}

\begin{frame}
	\frametitle{Notation}
	\framesubtitle{Minimizing losses}
	\begin{block}{}
		Given $X \subseteq \allalts$
		\[ \argmin_{X}(\sigma~\circ~\lambda_P)=\set{x \in X \suchthat \forall y \in X: \sigma(\lambda_P(x)) \leq \sigma(\lambda_P(y))}\]
	\end{block}
	\vspace{0.5cm}
	$\argmin_{X}(\sigma\circ\lambda_{P})$ denotes the alternatives in X whose loss vectors are the most equally distributed according to $\sigma$

\end{frame}

\begin{frame}
	\frametitle{Notation}
	\framesubtitle{Minimizing losses}
	\begin{center}
		\begin{tikzpicture}
			\path node (P) {$\prof$};
			\path (P.south) node[anchor=north] (profile) {$%
				\begin{array}{r@{\hspace{1mm} : \hspace{1mm}}l}
					v_1 & a \pref b \pref c\\%
					v_2 & c \pref b \pref a\\%
				\end{array}%
				$};
			\path (P.center) ++ (3cm, 0) node (L) {$\lambda_{\prof}$};
			\path (L.south) node[anchor=north] (losses) {$%
				\begin{array}{r@{\hspace{1mm} : \hspace{1mm}}l}
					a & (0, 2)\\%
					b & (1, 1)\\%
					c & (2, 0)\\%
				\end{array}%
				$};
		\end{tikzpicture}
	\end{center}
	\bigskip
	
	\[\argmin_{X}(\sigma~\circ~\lambda_P)=\{b\} \quad \forall \sigma \in \Sigma\]
\end{frame}

\subsection{Egalitarian Compromise}

\begin{frame}
	\frametitle{Egalitarian compromises}
	An SCR is Egalitarian Compromise Compatible iff at each profile, it selects some “less unequal” alternatives
	\begin{block}{Egalitarian compromise compatibility}
		An SCR $f$ is ECC iff 
		\[
			\exists \sigma \in \Sigma \suchthat \forall \prof \in \linors^N: f(\prof) \cap \argmin_\allalts (\sigma \circ \lambda_{\prof}) \neq \emptyset
		\]
	\end{block}
\end{frame}

\begin{frame}
	\frametitle{Egalitarian compromises and Pareto dominance}
	ECC rules are \emph{very} egalitarian
	\begin{center}
		\begin{tikzpicture}
			\path node (P) {$\prof$};
			\path (P.south) node[anchor=north] (profile) {$%
				\begin{array}{r@{\hspace{1mm} : \hspace{1mm}}l}
					v_1 & a \pref c \pref b\\%
					v_2 & c \pref a \pref b\\%
				\end{array}%
				$};
				\path (P.center) ++ (3cm, 0) node (L) {$\lambda_{\prof}$};
				\path (L.south) node[anchor=north] (losses) {$%
					\begin{array}{r@{\hspace{1mm} : \hspace{1mm}}l}
						a & (0, 1)\\%
						b & (2, 2)\\%
						c & (1, 0)\\%
					\end{array}%
					$};
		\end{tikzpicture}
	\end{center}
	\[\argmin_{\allalts} (\sigma \circ \lambda_{\prof}) = \{b\} \quad \forall \sigma \in \Sigma\]
%	\onslide<3>{Egalitarian compromises favor equality over Pareto efficiency}
	\onslide<3->\begin{theorem}
		ECC $\cap$ Paretian = $\emptyset$ \hfill {\small (for $n, m \geq 2$)}
	\end{theorem}
	\onslide<4-> $f \in \text{ECC} \Rightarrow b \in f(\prof)$, $f \in \text{Paretian} \Rightarrow b \notin f(\prof)$
\end{frame}

%\begin{frame}
%	\frametitle{ECC $\cap$ Paretian = $\emptyset$}
%	\begin{theorem}
%		No Paretian SCR is ECC \hfill {\small (for $n, m \geq 2$)}
%	\end{theorem}
%	\begin{proof}[Proof (for $m$ = 4; adaptations for any $m \geq 2$ are straightforward)]
%		$\begin{array}{r@{}l@{\hspace{1mm} \mapsto \hspace{1mm}}l}
%			1 &\text{ voter} & a_1 \pref a_2 \pref a_3 \pref x\\%
%			n - 1 &\text{ voters} & a_2 \pref a_3 \pref a_1 \pref x\\%
%		\end{array}$%
%		\begin{itemize}
%			\item $\forall \sigma \in \Sigma: \argmin_\allalts (\sigma \circ \lambda_{\prof}) = \set{x}$
%			\item $f \in \text{ECC} \Rightarrow x \in f(\prof)$
%			\item $f \in \text{Paretian} \Rightarrow x \notin f(\prof)$ \qedhere
%		\end{itemize}
%	\end{proof}
%\end{frame}

\subsection{Paretian compromises}
\begin{frame}
	\frametitle{Paretian compromises}
	An SCR is Paretian Compromise Compatible iff at each profile, it selects some “less unequal” alternatives \emph{among the Pareto optimal ones}
	\begin{block}{Paretian compromise compatibility}
		An SCR $f$ is PCC iff 
		\[\exists \sigma \in \Sigma \suchthat \forall \prof \in \linors^N: f(\prof) \cap \argmin_{\PO(\prof)} (\sigma \circ \lambda_{\prof}) \neq \emptyset \]
	\end{block}
%	{\small Recall:
%	\begin{block}{Egalitarian compromise compatibility}
%		An SCR $f$ is ECC iff 
%		\[\exists \sigma \in \Sigma \suchthat \forall \prof \in \linors^N: f(\prof) \cap \argmin_\allalts (\sigma \circ \lambda_{\prof}) \neq \emptyset\]
%	\end{block}}
\end{frame}

\begin{frame}
	\frametitle{FB and AP are PCC}
	\begin{theorem}
		FB and Antiplurality are PCC. \hfill {\small (for $n, m \geq 3$)}
	\end{theorem}
	\begin{proof}[Proof sketch]
		Define $\sigma^\text{discrete}(l) = 1 \iff l$ is not constant.
		
		If some $a \in \PO(\prof)$ has a constant loss vector, e.\ g.\ 
		\\$\begin{array}{l}
			a_1 \pref a_2 \pref a_3 \pref a_4\\
			a_3 \pref a_2 \pref a_1 \pref a_4 \\
		\end{array}$
		\\there is exactly one such alternative, $FB(\prof) = \set{a}$ and it is never last so $a \in AP(\prof)$.
		
		Otherwise, $\sigma$ does not discriminate among $\PO(\prof)$, thus Paretianism suffices.
	\end{proof}
\end{frame}

\subsection{Restricting \texorpdfstring{$\Sigma$}{Sigma}}
\begin{frame}
	\frametitle{Restricting $\Sigma$}
	\begin{definition}[Condition $C_{m, n}$]
		Given $m \geq 4, n \geq \max\set{4, m - 1}$, $\sigma$ satisfies condition $C_{m, n}$ iff $\sigma(m - 3, m - 1, m - 2, …, m - 2) < \sigma(m - 2, m - 3, …, 1, 0, …, 0)$.
	\end{definition}
	\begin{center}
		$\begin{array}{r@{\hspace{1mm} : \hspace{1mm}}lllll}
			v_1 &	&	&x	&y	&a_1\\%
			v_2 &	&	&y	&	&x\\%
			v_3 &	&y	&	&x	&a_2\\%
			v_4 &y	&	&	&x	&a_3\\%
		\end{array}$%
	\end{center}
	Requires that:
	\[(\sigma \circ \lambda_{\prof})(x) < (\sigma \circ \lambda_{\prof})(y)\] 
	
\end{frame}

\begin{frame}
	\frametitle{Restricting $\Sigma$}
	\begin{theorem}
		Under condition $C_{m, n}$, AP and FB are not PCC.
	\end{theorem}
	\begin{proof}[Proof for $m = 5, n = 4$]
		$\begin{array}{r@{\hspace{1mm} : \hspace{1mm}}lllll}
			v_1 &	&	&x	&y	&a_1\\%
			v_2 &	&	&y	&	&x\\%
			v_3 &	&y	&	&x	&a_2\\%
			v_4 &y	&	&	&x	&a_3\\%
		\end{array}$%
		\begin{itemize}
			\item $y$ is the only alternative never last, thus for both rules: $f(\prof) = \set{y}$
			\item $(\sigma \circ \lambda_{\prof})(x) < (\sigma \circ \lambda_{\prof})(y)$
			\item and $x \in \PO(\prof)$, thus $y \notin \argmin_{\PO(\prof)}(\sigma \circ \lambda_{\prof})$ \qedhere
		\end{itemize}
	\end{proof}
\end{frame}

\subsection{Other results}
\begin{frame}
	\frametitle{Other results}
	\begin{theorem}
		Condorcet consistent rules are neither ECC nor PCC \hfill {\small (for $m, n \geq 3$)}
	\end{theorem}
	\begin{theorem}
		Scoring rules,except AP, are neither ECC nor PCC \hfill {\small (for $m \geq 3$ and large enough $n$)}
	\end{theorem}
	\begin{theorem}
		FB$_q$ rules with $q \in \intvl{1, n - 1}$ are neither ECC nor PCC \hfill {\small (for $m, n \geq 3$)}
	\end{theorem}
\end{frame}

\section[Simultaneous Elicitation of PSR and Agent Preferences]{Simultaneous Elicitation of Scoring Rule and Agent Preferences for Robust Winner Determination}

\subsection{Context}

%\begin{frame}
%	\frametitle{Classical Scenario}
%	\textbf{Setting}: Voters specify preferences over alternatives and a committee defines the social choice rule to aggregate them
%	\begin{figure}
%		\includegraphics[scale=0.35]{classset.png}
%		%		\caption{.}
%		%		\label{fig:b1}
%	\end{figure}
%	\onslide<2-> \textbf{Goal}: Find a consensus choice 
%\end{frame}

\begin{frame}
\frametitle{Introducing the problem}
%\framesubtitle{Robust Winner Determination}
%	\textbf{Setting}: Two kind of players
\textbf{Setting}: Incompletely specified preferences and social choice rule \bigskip
\onslide<2-> \begin{figure}
	\includegraphics[scale=0.35]{ourset.png}
	%		\caption{.}
	%		\label{fig:b1}
\end{figure}
\onslide<3-> \textbf{Goal}: Develop an incremental elicitation strategy to quickly acquire the most relevant information 
\end{frame}
%\addtocounter{framenumber}{-1}
%\begin{frame}
%	\frametitle{Scenario}
%	\textbf{Setting}: Incompletely specified profile and positional scoring rule
%	\begin{figure}
%		\includegraphics[scale=0.35]{set2.png}
%%		\caption{.}
%%		\label{fig:b1}
%	\end{figure}
%	\textbf{Goal}: Development of an incremental elicitation protocol based on minimax regret 
%\end{frame}

\begin{frame}
	\frametitle{Motivation and approach}
	\textbf{Who?}
	\begin{itemize}
		\item Imagine to be an \emph{external observer} helping with the voting procedure
	\end{itemize}
	\onslide<2-> \textbf{Why?}
	\begin{itemize}
		\item Voters: difficult or costly to order \emph{all} alternatives
		\item Committee: difficult to \emph{specify} a voting rule precisely
	\end{itemize}
	\onslide<3-> \textbf{What?}
	\begin{itemize}
		\item We want to reduce uncertainty, inferring (\textit{eliciting}) the true preferences of voters and committee, in order to quickly converge to an optimal
		or a near-optimal alternative
	\end{itemize}		
\end{frame}

\begin{frame}
	\frametitle{Motivation and approach}
	\onslide<1-> \textbf{Approach}
	\begin{itemize}
		\item Develop query strategies that interleave questions to the committee and questions to the voters
		\item Use \emph{Minimax regret} to measure the quality of those strategies
	\end{itemize}
	\onslide<2-> \textbf{Assumptions}
	\begin{itemize}
		\item We consider \textit{positional scoring rules}, which attach weights to positions according to a scoring vector $w$
		\item We assume $w$ to be \textit{convex}
		\[ w_r - w_{r+1} \geq w_{r+1}-w_{r+2} \qquad \forall r\]
		and that $w_1=1$ and $w_m=0$
	\end{itemize}	
\end{frame}

\subsection{Related Works}
\begin{frame}
	\frametitle{Related Works}
	\textbf{Incomplete profile}  
	\begin{itemize}
		\item and known weights: Minimax regret to produce a robust winner approximation (\textit{Lu and Boutilier 2011}, \cite{Lu2011}; \textit{Boutilier et al. 2006}, \cite{Boutilier2006})
	\end{itemize}~\\
	\textbf{Uncertain weights} 
	\begin{itemize}
		\item and complete profile: dominance relations derived to eliminate alternatives always less preferred than others (\textit{Stein et al. 1994}, \cite{Stein1994})
		\item in positional scoring rules (\textit{Viappiani 2018}, \cite{Viappiani2018})
	\end{itemize}
\end{frame}

\subsection{Notation}
\begin{frame}
	\frametitle{Notation}

		\begin{description}[$W=(\w_k,\ 1\leq k \leq m), \ W \in \mathcal{W}$]
			\item [$P \in \mathcal{P}$] complete preferences profile 
			\item [$W=(\w_r,\ 1\leq r \leq m), \ W \in \mathcal{W}$] (convex) scoring vector that the committee has in mind
		\end{description}
		\bigskip
		\onslide<2-> \begin{block}{}
			$W$ defines a Positional Scoring Rule $f_W(P)\subseteq \allalts$ using scores \color{blue}$s^{W,P}(a), \ \forall \ a \in \allalts$
		\end{block}
		\onslide<3-> \begin{block}{}
			$P$ and $W$ exist in the minds of voters and committee but unknown to us
		\end{block}
		
\end{frame}

\subsection{Questions}
\begin{frame}
	\frametitle{Questions}
	Two types of questions:\\
	\vspace{0.5cm}
	\onslide<2-> \textbf{Questions to the voters}
	\begin{itemize}
		\item[] Comparison queries that ask a particular voter to compare two alternatives $a, b \in \allalts$
		\color{blue}\[a \pref_j b \quad ?\]
	\end{itemize}
	\onslide<3->  \textbf{Questions to the committee}
	\begin{itemize}
		\item[] Queries relating the difference between the importance of consecutive ranks from $r$ to $r+2$
		\color{blue} \[ w_{r} - w_{r+1} \geq \lambda (w_{r+1} - w_{r+2}) \quad ? \] 
	\end{itemize}
\end{frame}

\subsection{Current Knowledge}
\begin{frame}
	\frametitle{Current Knowledge}
	The answers to these questions define $C_P$ and $C_W$ that is our knowledge about P and W 
	\medskip
	\begin{itemize}
		\onslide<2-> \item $C_P \subseteq \mathcal{P}$ constraints on the profile given by the voters
		\onslide<3-> \item $C_W \subseteq \mathcal{W}$ constraints on the voting rule given by the committee
	\end{itemize}

	\def\Pcircle{(0,0) circle (0.6cm)}
	\def\CPcircle{(0,0) circle (0.2cm)}
	\def\Wcircle{(0,-1.6) circle (0.6cm)}
	\def\CWcircle{(0,-1.6) circle (0.2cm)}
	\def\Allcircle{(3,-0.8) circle (0.6cm)}
	\def\Currcircle{(3,-0.8) circle (0.2cm)}
	\def\wcircle{(3,-0.8) circle (0.05cm)}
	\begin{center}
		\begin{tikzpicture}
			\onslide<2->\draw \Pcircle node[left] at (0,0.7) {\footnotesize$\mathcal{P}$};
			\onslide<2->\filldraw[fill=gray, draw=black] \CPcircle node[above]  at (0,0.07) {\footnotesize$C_P$};
			\onslide<3->\draw \Wcircle node[left] at (0,-0.9) {\footnotesize$\mathcal{W}$};
			\onslide<3->\filldraw[fill=gray, draw=black] \CWcircle node[above]  at (0,-1.5) {\footnotesize$C_W$};
			\onslide<4->\draw [decorate, decoration = {brace, raise=40pt, amplitude=5pt}] (0,0.7) --  (0,-2.3);
			\onslide<5->\draw \Allcircle node[above] at (3,-0.2) {\footnotesize$\powersetz{\allalts}$};
			\onslide<5->\filldraw[fill=gray, draw=black] \Currcircle node[above]  at (3,-0.7){\footnotesize$C_{\mathscr{P}}$};
			\onslide<6->\filldraw[fill=black, draw=black] \wcircle;
		\end{tikzpicture}
	\end{center}
\end{frame}

\subsection{Minimax Regret}
\begin{frame}
	\frametitle{Minimax Regret}
	Given $C_P \subseteq \mathcal{P}$ and $C_W \subseteq \mathcal{W}$:
	
	\begin{block}{}
		\[\color{red}{\PMR^{C_P,C_W}(a,b)}= \max_{P\in C_P, W \in C_W} s^{P,W}(b)-s^{P,W}(a) \]
		is the maximum difference of score between $a$ and $b$ under all possible realizations of the full profile {\em and} weights
	\end{block}
	
	\onslide<2->  We care about the worst case loss: \emph{maximal regret} between a chosen alternative $a$ and best real alternative $b$
	\[\MR^{C_P,C_W}(a)= \max_{b\in \allalts} \PMR^{C_P,C_W}(a,b) \]
	 	
	\onslide<3-> \centerline{\textbf{We select the alternative which \emph{minimizes} the maximal regret}}
	\[\MMR^{C_P,C_W}= \min_{a\in \allalts} \MR^{C_P,C_W}(a)\]
\end{frame}

\subsection{Pairwise Max Regret Computation}
\begin{frame}
	\frametitle{Pairwise Max Regret Computation}
	The computation of $\PMR^{C_P,C_W}( \icimg{salad.png},\icimg{aubergine.png})$ can be seen as a game in which an adversary both:
	\begin{itemize}
		\onslide<2-> \item \textbf{chooses a complete profile $\mathbf{P \in \mathcal{P}}$}\\
		\medskip
		\begin{center}
			\includegraphics[scale=0.35]{completion4.png}
		\end{center}
		
		\onslide<3-> \item \textbf{chooses a feasible weight vector $\mathbf{W \in \mathcal{W}}$}\\
		\medskip
		\centerline{\color{red}$\mathbf{(1,?,0)}$ \icarr{arrow.png} \color{red}$\mathbf{(1,0,0)}$}
	\end{itemize}
	\medskip
	in order to maximize the difference of scores
\end{frame}

%\subsection{Computing Minimax Regret}
%\begin{frame}[t]
%	\frametitle{Computing Minimax Regret: Example}
%	\textbf{Profile completion}\\
%	Consider the following partial profile
%	\begin{center}
%		\includegraphics[scale=0.33]{compl.png}
%	\end{center}
%\end{frame}
%\begin{frame}[t]
%	\frametitle{Computing Minimax Regret: Example}
%	\textbf{Weight selection} \\ \bigskip
%	Consider the following constraint on the scoring vector given by the committee
%	\[w_1 \geq 2 \cdot w_2\]
%	and the convex assumption
%	\[w_1 - w_2 \geq w_2 - w_3 \]
%
%\end{frame}
%\begin{frame}[t]
%	\frametitle{Computing Minimax Regret: Example}
%	\textbf{Minimax computing}\\
%	\begin{center}
%		\includegraphics[scale=0.35]{minmax.png}
%	\end{center}
%\end{frame}

	
	\subsection{Elicitation strategies}
	\begin{frame}
		\frametitle{Elicitation strategies}
		At each step, the strategy selects a question to ask either to one of the voters about her preferences or to the committee about the voting rule \\ \bigskip
		\onslide<2-> The termination condition could be:
		\begin{itemize}
			\item <3-> when the minimax regret is lower than a threshold
			\item <4-> when the minimax regret is zero
		\end{itemize}
		\bigskip
	\end{frame}
	
	\begin{frame}[t]
		\frametitle{Elicitation strategies}
		\framesubtitle{Pessimistic Strategy}
		\onslide<1-> Assume that a question leads to the possible new knowledge states $(C_P^1, C_W^1)$ and $(C_P^2, C_W^2)$ depending on the answer, then the badness of the question in the worst case is:
		\[\max_{i=1,2} \MMR(C_P^i, C_W^i) \]
		\onslide<2-> The pessimistic strategy selects the question that leads to minimal regret in the worst case from a set of $n+1$ candidate questions \\
		\bigskip
		{\small \onslide<3-> \begin{block}{Note:}
			if the maximal MMR of two questions are equal, then prefers the one with the lowest MMR values associated to the opposite answer
		\end{block}}
	\end{frame}
	
	\begin{frame}
	\frametitle{Elicitation strategies}
	\framesubtitle{Pessimistic Strategy: Candidate questions}
	Let $(a^{*}, \bar{b}, \bar{P}, \bar{W})$ be the current solution of the minimax regret \\ \vspace{0.5em}
	We select $n + 1$	candidate questions:
	\begin{itemize}
		\item \textbf{One question per voter:} For each voter $i$, either:  
			\begin{itemize}
				\item $a^* \pref^{\bar{P}}_j \bar{b}$ : we ask about an incomparable alternative that can be placed above $a^*$ by the adversary to increase PMR($a^*$,$\bar{b}$)
				\item $\bar{b} \pref^{\bar{P}}_j a^*$: we ask about an incomparable alternative that can be placed between $a^*$ and $\bar{b}$ by the adversary to increase PMR($a^*$,$\bar{b}$) 
				\item $a^*$ and $\bar{b}$ are incomparable: we ask to compare them
		\end{itemize}
		\item \textbf{One question to the committee:} Consider $W_\tau$ the weight vector that minimize the PMR in the worst case. \\ We ask about the position
		$r = \argmax\limits_{i= [\![1, m-1]\!] } |\bar{W}(i)-W_\tau(i)|$
		
	\end{itemize}
	\end{frame}
	
	\subsection{Empirical Evaluation}
	\begin{frame}
		\frametitle{Empirical Evaluation}
		\framesubtitle{Pessimistic for different datasets}
		\begin{figure}
			\centering
			\caption{Average MMR (normalized by $n$) after $k$ questions with Pessimistic strategy for different datasets.}
			\label{fig:linearity}
			\begin{tikzpicture}
				\pgfplotsset{
					every axis legend/.append style={
						at={(0.5,1.1)},
						anchor=south
					},
				}
				\begin{axis}[
					y=80,
					legend columns=3,
					xlabel=Number of Questions,
					ylabel=MMR/n,
					ytick={0,0.5,1},
					xtick distance=100,
					xtick pos=left,
					ymajorgrids=true,
					enlarge x limits=-1, %hack to plot on the full x-axis scale
					width=10cm, %set bigger width
					ytick style={draw=none},
					ymin=0,
					ymax=1,
					xmin=0,
					xmax=1000,
					yticklabels={0,0.5,1},
					legend style={font=\footnotesize}]
					
					\addlegendimage{mark=halfsquare right*,brown,mark size=2}
					\addlegendimage{mark=diamond*,red,mark size=2}
					\addlegendimage{mark=pentagon*,cyan,mark size=2}
					\addlegendimage{mark=halfcircle*,violet,mark size=2}
					\addlegendimage{mark=*,pink,mark size=2}
					\addlegendimage{mark=triangle*,green,mark size=2}
					\addlegendimage{mark=halfsquare left*,blue,mark size=2}
					\addlegendimage{mark=square*,teal,mark size=2}
					\addlegendimage{mark=halfsquare*,magenta,mark size=2}
					
					
					\addplot[thick, mark=halfsquare right*, mark size = {2}, mark indices = {120}, brown] table [x=k, y=5.20]{data/linearity.dat};
					\addlegendentry{m=5, n=20}
					\addplot[thick, mark=diamond*, mark size = {2}, mark indices = {150}, red] table [x=k, y=10.20]{data/linearity.dat};
					\addlegendentry{m=10, n=20}
					\addplot[thick, mark=pentagon*, mark size = {2}, mark indices = {240}, cyan] table [x=k, y=11.30]{data/linearity.dat};
					\addlegendentry{m=11, n=30}
					\addplot[thick, mark=halfcircle*, mark size = {2}, mark indices = {400}, violet] table [x=k, y=tshirts]{data/linearity.dat};
					\addlegendentry{tshirts m11n30}
					\addplot[thick, mark=*, mark size = {2}, mark indices = {400}, pink] table [x=k, y=courses]{data/linearity.dat};
					\addlegendentry{courses m9n146}
					\addplot[thick, mark=triangle*, mark size = {2}, mark indices = {400}, green] table [x=k, y=9.146]{data/linearity.dat};
					\addlegendentry{m=9, n=146}
					\addplot[thick, mark=halfsquare left*, mark size = {2}, mark indices = {200}, blue] table [x=k, y=14.9]{data/linearity.dat};
					\addlegendentry{m=14, n=9}
					\addplot[thick, mark=square*, mark size = {2}, mark indices = {60}, teal] table [x=k, y=skate]{data/linearity.dat};
					\addlegendentry{skate m14n9}
					\addplot[thick, mark=halfsquare*, mark size = {2}, mark indices = {400}, magenta] table [x=k, y=15.30]{data/linearity.dat};
					\addlegendentry{m=15, n=30}
				\end{axis}
			\end{tikzpicture}
		\end{figure}
	\end{frame}
	
	\begin{frame}
		\frametitle{Empirical Evaluation}
		\framesubtitle{Pessimistic reaching "low enough" regret}
		\sisetup{table-number-alignment = center, table-figures-integer=2, table-figures-decimal=1, table-auto-round}
		\begin{minipage}{\textwidth}
			\captionof{table}{Questions asked by Pessimistic strategy on several datasets to reach $\frac{n}{10}$ regret, columns 4 and 5, and zero regret, last two columns.}
			\label{tab:questions}
			\scalebox{0.8}{
			\begin{tabular}{cccS[table-number-alignment = center, table-figures-integer=2] S[table-figures-integer=3, table-figures-decimal=1]@{ | }S[table-figures-integer=2, table-figures-decimal=1]@{ | }S[table-figures-integer=2, table-figures-decimal=1]@{ ]} S[table-number-alignment = center, table-figures-integer=2]S[table-figures-integer=3, table-figures-decimal=1]@{ | }S[table-figures-integer=2, table-figures-decimal=1]@{ | }S[table-figures-integer=2, table-figures-decimal=1]@{ ]}}
				\toprule
				{dataset} & m & n &{$q_{c}^{\scriptscriptstyle{MMR} \leq n/10}$} & \multicolumn{3}{c}{$q_{a}^{\scriptscriptstyle{MMR} \leq n/10}$} & {$q_{c}^{ \scriptscriptstyle{MMR} = 0}$} & \multicolumn{3}{c}{$q_{a}^{ \scriptscriptstyle{MMR} = 0}$} \\
				\midrule
				m5n20 & 5&20&0.0&[4.3 &4.95 & 5.84 &5.25&[ \ 5.36 & 6.15 & 7.21\\
				m10n20&10&20&0.0&[13.85 & 16.1& 18.41&31.95&[19.66 & 21.78 & 24.7\\
				m11n30&11&30&0.0&[16.55&19.0&22.26&45.15&[23.07&25.7&28.89\\
				tshirts&11&30&0.0&[13.08&16.6&19.58& 43.15 &[28.22&31.98 &35.62\\
				courses&9&146&0.0&[6.03 &7.0 &7.0&0.0&[6.81 & 7.0 &7.0\\
				%m9n146&9&146&0&0&[1.94 &8 &9.25&999.3&0.47&0&0&[1.94&8&9.25&999.3&0.47\\
				m14n9&14&9&5.4&[30.3&33.45&36.65&64.05&[37.55&40.5&44.3\\
				skate&14&9&0.0&[11.35&11.6&12.3&0.0&[11.5&11.8&12.75 \\
				m15n30&15&30&0.0&[24.95&29.5&33.68 \\
				\bottomrule
			\end{tabular}}
		\end{minipage}
	\end{frame}


\begin{frame}
	\frametitle{Empirical Evaluation}
	\framesubtitle{Pessimistic committee first and then voters (and vice-versa)}
	\sisetup{table-number-alignment = center, table-figures-integer=2, table-figures-decimal=1, table-auto-round}
	\begin{minipage}{\textwidth}
		\centering
		\captionof{table}{Average MMR in problems of size $(10, 20)$ after $500$ questions, among which $q_c$ to the chair.}
		\label{tab:twoP500}
		\scalebox{0.85}{
			\begin{tabular}{S[table-figures-integer=3, table-figures-decimal=0]S[table-number-alignment = right]@{ ± }S[table-number-alignment = left, table-figures-integer=1]S[table-number-alignment = right]@{ ± }S[table-number-alignment = left, table-figures-integer=1]}
				\toprule
				{$q_c$} & {2 ph.\ ca} & {sd} & {2 ph.\ ac} & {sd} \\
				\midrule		
				
				0	&	0.62	&	0.52	&	0.62	&	0.52	\\
				15	&	0.515	&	0.48	&	0.54	&	0.46	\\
				30	&	0.345	&	0.47	&	0.325	&	0.425	\\
				50	&	0.045	&	0.09	&	0.03	&	0.065	\\
				100	&	0.14	&	0.23	&	0.075	&	0.135	\\
				200	&	2.305	&	1.36	&	2.145	&	1.845	\\
				300	&	5.15	&	2.38	&	6.83	&	0.625	\\
				400	&	10.905	&	0.89	&	12.245	&	0.99	\\
				500	&	20.0	&	0.0	&	20.0	&	0.0	\\
				
				\bottomrule
			\end{tabular}
		}
	\end{minipage}
\end{frame}


\section{Preference Elicitation under Majority Judgment}
\subsection{Context}
\begin{frame}
	\frametitle{Context}
	\framesubtitle{Introducing the problem}
	\textbf{Setting}: Voters judges a random subset of alternatives and the preferences are aggregated with the Majority Judgment rule \vspace{6mm}
	
	\onslide<2->{\textbf{Goal}: Analyse the impact of the randomness in the result and find a more efficient elicitation procedure}
\end{frame}

\begin{frame}
	\frametitle{Context}
	\framesubtitle{Majority Judgment}
		Voters judges candidates assigning grades from an ordinal scale. The winner is the candidate with the highest median of the grades received. \vspace{1cm} \\
	\includegraphics[width=\textwidth]{vector}
\end{frame}

\begin{frame}
	\frametitle{\textbf{Current Work:} Preference Elicitation under Majority Judgment}
	\framesubtitle{Introducing the problem}
	\onslide<1-> In the last few years MJ has being adopted by a progressively larger number of french political parties including: Le Parti Pirate, Génération(s), LaPrimaire.org, France Insoumise and La République en Marche. \vspace{1cm}
	
	\onslide<2-> LaPrimaire.org is a french political initiative whose goal is to select an independent candidate for the french presidential election using MJ as voting rule.
\end{frame}

\begin{frame}
	\frametitle{\textbf{Current Work:} Preference Elicitation under Majority Judgment}
	\framesubtitle{LaPrimaire.org}
	The procedure consists of two rounds:
	\begin{itemize}
		\item[1:]<2-> each voter expresses her judgment on five random candidates. The five ones with the highest medians qualify for the second round. 
		\item[2:]<3-> each voter expresses her judgment on all the five finalists. The one with the best median is the winner.
	\end{itemize}
\end{frame}

\subsection{Research Questions}
\begin{frame}
	\frametitle{\textbf{Current Work:} Research Questions}
	\begin{itemize}
		\item<1-> Does expressing judgment on random candidates influence the result? 
		\item<2-> Does the number of questions influence the result? 
		\item<3-> What is the best trade-off between communication cost and optimal result?
		\item<4-> What is the voting rule applied on the resulting incomplete profile? What are its properties?
		\item<5-> The random selection of questions is fair in terms of probability of being asked about a certain candidate $i$, but is it fair in terms of $i$ being elected?
		\item<6-> Can we select the next question using a minimax regret notion instead of randomly selecting a candidate?
		\item<7-> Suppose that the fraction of candidates that each voter judges is variable, how this rule differ from the previous one? Can a voter manipulate the result by judging only certain candidates?
	\end{itemize}
\end{frame}


\addtocounter{framenumber}{-1}
\begin{frame}[plain]
	\centering \color{darkred}\LARGE Thank You!
\end{frame}

\begin{frame}
	\frametitle{Plan of the thesis and questions}
	\begin{itemize}
		\item Final dissertation by October 2021, defense by December 2021
		\item Status of the works:
		\begin{itemize}
			\item Compromise: Rejected from Social Choice and Welfare; under submission to Review of Economic Design;
			\item Elicitation PSR: Rejected from IJCAI20, AAMAS21 and IJCAI21; under revision at ADT21;
			\item Elicitation MJ: ongoing work, plan to have a final draft before the defense.
		\end{itemize}
		\item Given the current status of my works, is the plan feasible?
		\item Any suggestions on the dissertation structure ?
	\end{itemize}
\end{frame}


\bibliographystyle{plain}
\bibliography{biblio} 
%given a combination of axioms we want to find an outcome that doesn't satisfy them, and we would do that for several reasons:
%-querying the user, depending of her answer we might infer her preferences over the set of axioms;
%-proving that a set of axioms is not valid giving a counter-example;


%A method for automatically proving impossibility theorems in the area of ranking sets of objects has already been implemented (Geist \& Endriss, 2011). It:



\end{document}