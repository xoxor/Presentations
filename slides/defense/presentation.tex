%\documentclass[aspectratio=169]{beamer}
\documentclass{beamer}
\usetheme{CambridgeUS}
\usecolortheme{beaver}

\newcommand\hmmax{0}
\newcommand\bmmax{0}

\usepackage{stmaryrd}

\usepackage{amsmath,amssymb,enumerate,amsthm}
\usepackage{bm}
\usepackage{graphicx}

\usepackage{algorithm, algpseudocode}
\usepackage[strict]{siunitx}
\usepackage{hyperref}
\usepackage{mathrsfs}

\usepackage{bm}
\usepackage{empheq}
%\usepackage{cleveref}
\usepackage{xcolor}
\usepackage{textcomp}
\usepackage{siunitx}
\usepackage{booktabs}
\usepackage{caption}
\usepackage{tikz}
\usetikzlibrary{decorations.pathreplacing,calligraphy}
\usepackage{pgfplots}
\pgfplotsset{compat=1.17}
\usepackage{subfig}
\usepackage{lmodern}
\input{preamble/math_basics}
%Decision Theory (MCDA and SC)
\newcommand{\allalts}{\mathscr{A}}
\newcommand{\allcrits}{\mathscr{C}}
\newcommand{\alts}{A}
\newcommand{\dm}{i}
\newcommand{\allF}{\mathscr{F}}
\newcommand{\allvoters}{\mathscr{N}}
\newcommand{\voters}{N}
\newcommand{\allprofs}{\boldsymbol{\mathcal{R}}}
\newcommand{\prof}{\boldsymbol{R}}
\newcommand{\linors}{\mathscr{L}(\allalts)}
%Thanks to https://tex.stackexchange.com/q/154549
	%\makeatletter
	%\def\@myRgood@#1#2{\mathrel{R^X_{#2}}}
	%\def\myRgood{\@ifnextchar_{\@myRgood@}{\mathrel{R^X}}}
	%\makeatother

%Deliberated Judgment
\newcommand{\prop}{\textcolor{red}{t}}
\newcommand{\allargs}{S^*}
\newcommand{\args}{S}
\newcommand{\ar}[1][]{%
	\textcolor{blue}{%
		\ifx\\#1\\%
			s
		\else
			s_#1
		\fi%
	}%
}
\newcommand{\ileadsto}{\textcolor{brown}{⇝}}
\newcommand{\ibeatse}{\textcolor{brown}{⊳_\exists}}
\newcommand{\nibeatse}{⋫_\exists}
\newcommand{\ibeatsst}{⊳_\forall}
\newcommand{\nibeatsst}{⋫_\forall}
\newcommand{\mleadsto}[1][\eta]{⇝_{#1}}
\newcommand{\mbeats}[1][\eta]{\textcolor{violet}{⊳_{#1}}}
\newcommand{\ibeatseinv}{⊳_\exists^{-1}}

%Logic
\newcommand{\ltru}{\texttt{T}}
\newcommand{\lfal}{\texttt{F}}


\newcommand{\profile}{\bm{v}}%(complete) profile
\newcommand{\pprofile}{{\bm{p}}}%partial profile
\newcommand{\w}{\bm{w}}
\newcommand{\W}{\mathcal{W}}
\newcommand{\Co}{\mathcal{C}}
\newcommand{\pw}{W}%our knowledge about the weights
\newcommand{\strat}[1]{\emph{#1}}
\newcommand{\ppref}{\succ^\text{p}}%partial pref
\newcommand{\pprefeq}{\succeq^\text{p}}%partial pref
\DeclareMathOperator{\Regret}{Regret}
\DeclareMathOperator{\SCORE}{Score}
\DeclareMathOperator{\PMR}{PMR}
\DeclareMathOperator{\MR}{MR}
\DeclareMathOperator{\MMR}{MMR}

\newcommand*{\icimg}[1]{%
	\raisebox{-.3\baselineskip}{%
		\includegraphics[
		height=\baselineskip,
		width=\baselineskip,
		keepaspectratio,
		]{#1}%
	}%
}

\newcommand*{\icarr}[1]{%
	\raisebox{-0.4\baselineskip}{%
		\includegraphics[
		height=2.5\baselineskip,
		width=3\baselineskip,
		keepaspectratio,
		]{#1}%
	}%
}

%%Title page
\makeatletter
\defbeamertemplate*{title page}{mytitle}[1][]
{
	\vbox{}
	\vfill
	\begin{centering}

%{\usebeamercolor[fg]{titlegraphic}\inserttitlegraphic\par}
		\begin{beamercolorbox}{titlegraphic}
				\usebeamerfont{titlegraphic}\inserttitlegraphic
		\end{beamercolorbox}%
			\vskip1em\par	
		\begin{beamercolorbox}[rounded=true, center, shadow=true, sep=8pt,#1]{title}
			\usebeamerfont{title}\inserttitle\par%
			\ifx\insertsubtitle\@empty%
			\else%
			\vskip0.5em%
			{\usebeamerfont{subtitle}\usebeamercolor[fg]{subtitle}\insertsubtitle\par}%
			\fi%     
		\end{beamercolorbox}%
		\vskip1em\par
		\begin{beamercolorbox}[sep=8pt,center,#1]{author}
			\usebeamerfont{author}\insertauthor
		\end{beamercolorbox}
		\begin{beamercolorbox}[sep=8pt,center,#1]{institute}
			\usebeamerfont{institute}\insertinstitute
		\end{beamercolorbox}
		\begin{beamercolorbox}[sep=8pt,center,#1]{date}
			\usebeamerfont{date}\insertdate
		\end{beamercolorbox}\vskip0.5em
		\begin{beamercolorbox}[sep=8pt,center,#1]{logo}
			\insertlogo
		\end{beamercolorbox}%
	\end{centering}
	
}
\setbeamertemplate{title page}[mytitle]
\makeatother

%Footline
\makeatletter
\setbeamercolor{progress bar progress}{use=progress bar,bg=progress bar.fg}

\def\progressbar@progressbar{} % the progress bar
\newcount\progressbar@tmpcounta% auxiliary counter
\newcount\progressbar@tmpcountb% auxiliary counter
\newdimen\progressbar@pbht %progressbar height
\newdimen\progressbar@pbwd %progressbar width
\newdimen\progressbar@tmpdim % auxiliary dimension
\progressbar@pbwd=\paperwidth
\progressbar@pbht=0.5ex
% the progress bar

\defbeamertemplate{footline}{progress bar}{
	\progressbar@tmpcounta= \insertframenumber % max = ?
	\progressbar@tmpcountb=\inserttotalframenumber      
	\progressbar@tmpdim=.5\progressbar@pbwd
	\multiply\progressbar@tmpdim by \progressbar@tmpcounta
	\divide\progressbar@tmpdim by \progressbar@tmpcountb
	\progressbar@tmpdim=2\progressbar@tmpdim

	\leavevmode%
	\begin{beamercolorbox}[wd=\paperwidth,ht=2ex,dp=1ex,right]{bar}
		\insertframenumber{} / \inserttotalframenumber\hspace*{2ex} 
	\end{beamercolorbox}
	\newline
	\begin{beamercolorbox}[wd=\paperwidth,ht=1ex,dp=1ex]{progress bar}
		\begin{beamercolorbox}[wd=\progressbar@tmpdim,ht=1ex,dp=1ex]{progress bar progress}
		\end{beamercolorbox}%
	\end{beamercolorbox}%
	
	
}
\setbeamertemplate{footline}[progress bar]
\setbeamercolor{progress bar}{fg=red!65!black,bg=white!85!black}
\makeatother

%Headline empty
\setbeamertemplate{headline}{}


\titlegraphic{\includegraphics[width=50mm]{logo_dauphine}\hspace*{5.5cm} \includegraphics[width=7mm]{cnrs}}

\title[]{Compromise and elicitation in social choice}
%
\subtitle{A study of egalitarianism and incomplete information in voting}
%\institute[]{Université Paris-Dauphine, Université PSL, CNRS, LAMSADE}
\author{Beatrice Napolitano}
\date{\small{Ph.D. Thesis Defense, 09 December 2022} \\ \includegraphics[width=35mm]{LOGO_LAMSADE} }

\usepackage{tikz}
\usepackage{amsmath}
\usepackage{graphicx}

\usepackage{bibentry}

\definecolor{darkred}{rgb}{0.8,0,0}
\definecolor{amber}{rgb}{1.0, 0.75, 0.0}
\definecolor{mygreen}{rgb}{0.0, 0.5, 0.0}
\setbeamerfont{framesubtitle}{size=\large}

\begin{document}

\beamertemplatenavigationsymbolsempty

\begin{frame}[plain]
	\bibliographystyle{plain}
	\nobibliography*
	\maketitle
\end{frame}

\addtocounter{framenumber}{-1}


\begin{frame}
	\frametitle{Classical setting}
%	\vspace{-2em}
	\footnotesize{Agents} $=\{$ \includegraphics[scale=0.05]{assets/v1.png}, \includegraphics[scale=0.05]{assets/v2.png}, \includegraphics[scale=0.05]{assets/v3.png} $\}$, $\ $ \footnotesize{Altern.} $=\{$ \includegraphics[scale=0.025]{assets/d1.png}, \includegraphics[scale=0.025]{assets/d2.png}, \includegraphics[scale=0.025]{assets/d3.png} $\}$, $\ $ \footnotesize{Chair} $=$ \includegraphics[scale=0.02]{assets/committee.png} $\Rightarrow$ \footnotesize{Voting Rule}
	\vspace{2em}
	\begin{columns}
		\begin{column}{0.45\textwidth}
			\begin{figure}
				\begin{tabular}{ccc}
					\visible<2->{\subfloat{\includegraphics[scale=0.12]{assets/v1.png}}} &
					\visible<2->{\subfloat{\includegraphics[scale=0.12]{assets/v2.png}}} &
					\visible<2->{\subfloat{\includegraphics[scale=0.12]{assets/v3.png}}} \\
					\visible<3->{\subfloat{\includegraphics[scale=0.05]{assets/d1.png}}} &
					\visible<4->{\subfloat{\includegraphics[scale=0.05]{assets/d1.png}}} &
					\visible<5->{\subfloat{\includegraphics[scale=0.05]{assets/d3.png}}} \\
					\visible<3->{\subfloat{\includegraphics[scale=0.05]{assets/d2.png}}} &
					\visible<4->{\subfloat{\includegraphics[scale=0.05]{assets/d3.png}}} &
					\visible<5->{\subfloat{\includegraphics[scale=0.05]{assets/d2.png}}} \\
					\visible<3->{\subfloat{\includegraphics[scale=0.05]{assets/d3.png}}} &
					\visible<4->{\subfloat{\includegraphics[scale=0.05]{assets/d2.png}}} &
					\visible<5->{\subfloat{\includegraphics[scale=0.05]{assets/d1.png}}} \\
				\end{tabular}
			\end{figure}
		
		\end{column}
		\begin{column}{0.45\textwidth}
			\begin{center}
				\visible<6->{\includegraphics[width=0.5\textwidth]{assets/committee.png}}\\
				\vspace{2em}
				\visible<7->{Borda}
			\end{center}
			
		\end{column}
	\end{columns}
	\begin{center}
		\visible<8->{winner: \includegraphics[scale=0.05]{assets/d1.png}}
	\end{center}
	
\end{frame}



\begin{frame}
	\frametitle{Incomplete knowledge about profile}
	\vspace{-0.8em}
	\footnotesize{Agents} $=\{$ \includegraphics[scale=0.05]{assets/v1.png}, \includegraphics[scale=0.05]{assets/v2.png}, \includegraphics[scale=0.05]{assets/v3.png} $\}$, $\ $ \footnotesize{Altern.} $=\{$ \includegraphics[scale=0.025]{assets/d1.png}, \includegraphics[scale=0.025]{assets/d2.png}, \includegraphics[scale=0.025]{assets/d3.png} $\}$, $\ $ \footnotesize{Chair} $=$ \includegraphics[scale=0.02]{assets/committee.png} $\Rightarrow$ \footnotesize{Voting Rule}
	\vspace{2em}
	\begin{columns}
		\begin{column}{0.45\textwidth}
			\begin{figure}
				\begin{tabular}{ccc}
					\visible<1->{\subfloat{\includegraphics[scale=0.12]{assets/v1.png}}} &
					\visible<1->{\subfloat{\includegraphics[scale=0.12]{assets/v2.png}}} &
					\visible<1->{\subfloat{\includegraphics[scale=0.12]{assets/v3.png}}} \\
					\visible<1->{\subfloat{\includegraphics[scale=0.05]{assets/d1.png}}} &
					\visible<1->{\subfloat{\includegraphics[scale=0.05]{assets/d1.png}}} & \\
					\visible<1->{\subfloat{\includegraphics[scale=0.05]{assets/d2.png}}} &
					\visible<1->{\subfloat{\includegraphics[scale=0.05]{assets/d2.png}}} & \\
					\visible<1->{\subfloat{\includegraphics[scale=0.05]{assets/d3.png}}} & & \\
				\end{tabular}
			\end{figure}
		\end{column}
		\begin{column}{0.45\textwidth}
			\begin{center}
				\visible<1->{\includegraphics[width=0.5\textwidth]{assets/committee.png}}\\
				\vspace{2em}
				\visible<1->{Borda}
				%\visible<6->{weight$(1$\textsuperscript{st}$)$ $\geq$ 2 $\cdot$ weight$(2$\textsuperscript{nd}$)$}
			\end{center}
			
		\end{column}
	\end{columns}
	\begin{center}
		\visible<1->{winner: ?}
	\end{center}
\end{frame}

%\begin{frame}
%	\frametitle{Related Works}
%	\textbf{Incomplete profile}  
%	\begin{itemize}
%		\item and known rule: Minimax regret to produce a robust winner approximation (\textit{Lu and Boutilier 2011}, \cite{Lu2011}; \textit{Boutilier et al. 2006}, \cite{Boutilier2006})
%	\end{itemize}~\\
%\vspace{6em}
%%	\textbf{Uncertain weights} 
%%	\begin{itemize}
%%		\item and complete profile: dominance relations derived to eliminate alternatives always less preferred than others (\textit{Stein et al. 1994}, \cite{Stein1994})
%%		\item in positional scoring rules (\textit{Viappiani 2018}, \cite{Viappiani2018})
%%	\end{itemize}
%\end{frame}

\begin{frame}
	\frametitle{Incomplete knowledge about voting rule}
	\vspace{-0.8em}
	\footnotesize{Agents} $=\{$ \includegraphics[scale=0.05]{assets/v1.png}, \includegraphics[scale=0.05]{assets/v2.png}, \includegraphics[scale=0.05]{assets/v3.png} $\}$, $\ $ \footnotesize{Altern.} $=\{$ \includegraphics[scale=0.025]{assets/d1.png}, \includegraphics[scale=0.025]{assets/d2.png}, \includegraphics[scale=0.025]{assets/d3.png} $\}$, $\ $ \footnotesize{Chair} $=$ \includegraphics[scale=0.02]{assets/committee.png} $\Rightarrow$ \footnotesize{Voting Rule}
	\vspace{2em}
	\begin{columns}
		\begin{column}{0.45\textwidth}
			\begin{figure}
				\begin{tabular}{ccc}
					\visible<1->{\subfloat{\includegraphics[scale=0.12]{assets/v1.png}}} &
					\visible<1->{\subfloat{\includegraphics[scale=0.12]{assets/v2.png}}} &
					\visible<1->{\subfloat{\includegraphics[scale=0.12]{assets/v3.png}}} \\
					\visible<1->{\subfloat{\includegraphics[scale=0.05]{assets/d1.png}}} &
					\visible<1->{\subfloat{\includegraphics[scale=0.05]{assets/d1.png}}} &
					\visible<1->{\subfloat{\includegraphics[scale=0.05]{assets/d3.png}}} \\
					\visible<1->{\subfloat{\includegraphics[scale=0.05]{assets/d2.png}}} &
					\visible<1->{\subfloat{\includegraphics[scale=0.05]{assets/d3.png}}} &
					\visible<1->{\subfloat{\includegraphics[scale=0.05]{assets/d2.png}}} \\
					\visible<1->{\subfloat{\includegraphics[scale=0.05]{assets/d3.png}}} &
					\visible<1->{\subfloat{\includegraphics[scale=0.05]{assets/d2.png}}} &
					\visible<1->{\subfloat{\includegraphics[scale=0.05]{assets/d1.png}}} \\
				\end{tabular}
			\end{figure}
		\end{column}
		\begin{column}{0.45\textwidth}
			\begin{center}
				\visible<1->{\includegraphics[width=0.5\textwidth]{assets/committee.png}}\\
				\vspace{2em}
				\visible<1->{?}
				%\visible<6->{weight$(1$\textsuperscript{st}$)$ $\geq$ 2 $\cdot$ weight$(2$\textsuperscript{nd}$)$}
			\end{center}
			
		\end{column}
	\end{columns}
	\begin{center}
		\visible<1->{winner: ?}
	\end{center}
\end{frame}

\begin{frame}
	\frametitle{Research Question I:}
	\framesubtitle{Incomplete knowledge about profile and voting rule}
	\vspace{-0.8em}
	\footnotesize{Agents} $=\{$ \includegraphics[scale=0.05]{assets/v1.png}, \includegraphics[scale=0.05]{assets/v2.png}, \includegraphics[scale=0.05]{assets/v3.png} $\}$, $\ $ \footnotesize{Altern.} $=\{$ \includegraphics[scale=0.025]{assets/d1.png}, \includegraphics[scale=0.025]{assets/d2.png}, \includegraphics[scale=0.025]{assets/d3.png} $\}$, $\ $ \footnotesize{Chair} $=$ \includegraphics[scale=0.02]{assets/committee.png} $\Rightarrow$ \footnotesize{Voting Rule}
	\vspace{2em}
	\begin{columns}
		\begin{column}{0.45\textwidth}
			\begin{figure}
				\begin{tabular}{ccc}
					\visible<1->{\subfloat{\includegraphics[scale=0.12]{assets/v1.png}}} &
					\visible<1->{\subfloat{\includegraphics[scale=0.12]{assets/v2.png}}} &
					\visible<1->{\subfloat{\includegraphics[scale=0.12]{assets/v3.png}}} \\
					\visible<1->{\subfloat{\includegraphics[scale=0.05]{assets/d1.png}}} &
					\visible<1->{\subfloat{\includegraphics[scale=0.05]{assets/d1.png}}} & \\
					\visible<1->{\subfloat{\includegraphics[scale=0.05]{assets/d2.png}}} &
					\visible<1->{\subfloat{\includegraphics[scale=0.05]{assets/d2.png}}} & \\
					\visible<1->{\subfloat{\includegraphics[scale=0.05]{assets/d3.png}}} & & \\
				\end{tabular}
			\end{figure}
		\end{column}
		\begin{column}{0.45\textwidth}
			\begin{center}
				\visible<1->{\includegraphics[width=0.5\textwidth]{assets/committee.png}}\\
				\vspace{2em}
				\visible<1->{?}
				%\visible<6->{weight$(1$\textsuperscript{st}$)$ $\geq$ 2 $\cdot$ weight$(2$\textsuperscript{nd}$)$}
			\end{center}
			
		\end{column}
	\end{columns}
	\begin{center}
		\visible<1->{winner: ?}
	\end{center}
\end{frame}

\begin{frame}
	\frametitle{Research Question II:}
	\framesubtitle{Incomplete knowledge under Majority Judgment}
	\vspace{-0.8em}
	\footnotesize{Agents} $=\{$ \includegraphics[scale=0.05]{assets/v1.png}, \includegraphics[scale=0.05]{assets/v2.png}, \includegraphics[scale=0.05]{assets/v3.png} $\}$, $\ $ \footnotesize{Altern.} $=\{$ \includegraphics[scale=0.025]{assets/d1.png}, \includegraphics[scale=0.025]{assets/d2.png}, \includegraphics[scale=0.025]{assets/d3.png} $\}$, $\ $ \footnotesize{Chair} $=$ \includegraphics[scale=0.02]{assets/committee.png} $\Rightarrow$ \footnotesize{Voting Rule}
	\vspace{2em}
	\begin{columns}
		\begin{column}{0.45\textwidth}
			\begin{figure}
				\begin{tabular}{ccc}
					\visible<1->{\subfloat{\includegraphics[scale=0.12]{assets/v1.png}}} &
					\visible<1->{\subfloat{\includegraphics[scale=0.12]{assets/v2.png}}} &
					\visible<1->{\subfloat{\includegraphics[scale=0.12]{assets/v3.png}}} \\
					\visible<1->{\subfloat{\includegraphics[scale=0.05]{assets/d1.png}}, \color{mygreen}Good} &
					\visible<1->{\subfloat{\includegraphics[scale=0.05]{assets/d1.png}}, \color{mygreen}Excell.} & \\
					\visible<1->{\subfloat{\includegraphics[scale=0.05]{assets/d2.png}}, \color{amber}Aver.} &
					\visible<1->{\subfloat{\includegraphics[scale=0.05]{assets/d2.png}}, \color{red}To rej.} & \\
					\visible<1->{\subfloat{\includegraphics[scale=0.05]{assets/d3.png}}, \color{orange}Med.} & & \\
				\end{tabular}
			\end{figure}
		\end{column}
		\begin{column}{0.45\textwidth}
			\begin{center}
				\visible<1->{\includegraphics[width=0.5\textwidth]{assets/committee.png}}\\
				\vspace{2em}
				\visible<1->{Majority Judgment}
				%\visible<6->{weight$(1$\textsuperscript{st}$)$ $\geq$ 2 $\cdot$ weight$(2$\textsuperscript{nd}$)$}
			\end{center}
			
		\end{column}
	\end{columns}
	\begin{center}
		\visible<1->{winner: ?}
	\end{center}
\end{frame}

\begin{frame}
	\frametitle{Research Question III:}
	\framesubtitle{Compromise from an equal-loss perspective}
	\vspace{-0.8em}
	\footnotesize{Agents} $=\{$ \includegraphics[scale=0.05]{assets/v1.png}, \includegraphics[scale=0.05]{assets/v2.png}, \includegraphics[scale=0.05]{assets/v3.png} $\}$, $\ $ \footnotesize{Altern.} $=\{$ \includegraphics[scale=0.025]{assets/d1.png}, \includegraphics[scale=0.025]{assets/d2.png}, \includegraphics[scale=0.025]{assets/d3.png} $\}$, $\ $ \footnotesize{Chair} $=$ \includegraphics[scale=0.02]{assets/committee.png} $\Rightarrow$ \footnotesize{Voting Rule}
	\vspace{2em}
	\begin{columns}
		\begin{column}{0.45\textwidth}
			\begin{figure}
				\begin{tabular}{ccc}
					\visible<1->{\subfloat{\includegraphics[scale=0.12]{assets/v1.png}}} &
					\visible<1->{\subfloat{\includegraphics[scale=0.12]{assets/v2.png}}} &
					\visible<1->{\subfloat{\includegraphics[scale=0.12]{assets/v3.png}}} \\
					\visible<1->{\subfloat{\includegraphics[scale=0.05]{assets/d1.png}}} &
					\visible<1->{\subfloat{\includegraphics[scale=0.05]{assets/d1.png}}} &
					\visible<1->{\subfloat{\includegraphics[scale=0.05]{assets/d3.png}}} \\
					\visible<1->{\subfloat{\includegraphics[scale=0.05]{assets/d2.png}}} &
					\visible<1->{\subfloat{\includegraphics[scale=0.05]{assets/d2.png}}} &
					\visible<1->{\subfloat{\includegraphics[scale=0.05]{assets/d2.png}}} \\
					\visible<1->{\subfloat{\includegraphics[scale=0.05]{assets/d3.png}}} &
					\visible<1->{\subfloat{\includegraphics[scale=0.05]{assets/d3.png}}} &
					\visible<1->{\subfloat{\includegraphics[scale=0.05]{assets/d1.png}}} \\
				\end{tabular}
			\end{figure}
		\end{column}
		\begin{column}{0.45\textwidth}
			\begin{center}
				\visible<1->{\includegraphics[width=0.5\textwidth]{assets/committee.png}}\\
				\vspace{2em}
				\visible<1->{What is a compromise?}
				%\visible<6->{weight$(1$\textsuperscript{st}$)$ $\geq$ 2 $\cdot$ weight$(2$\textsuperscript{nd}$)$}
			\end{center}
			
		\end{column}
	\end{columns}
	\begin{center}
		\visible<2->{maybe \includegraphics[scale=0.05]{assets/d2.png} ?}
	\end{center}
\end{frame}

\begin{frame}
	\frametitle{Outline}
	\tableofcontents[hideallsubsections, sectionstyle=shaded/show]
\end{frame}

\AtBeginSection{
	\begin{frame}
		\frametitle{Outline}
		\tableofcontents[currentsection, hideallsubsections]
	\end{frame}
}

\section{Notation}
\begin{frame}
	\frametitle{Notation}	
	\begin{description}[$\prof=(\succ_{1},\dots,\succ_{n}) \in \linors^\voters$]
		\item [$\allalts$] set of alternatives, $|\allalts|=m$
		\item [$\voters$] set of voters, $|\voters|=n$
		\item [$\linors$] set of all linear orderings given $\allalts$
		\item [${\prefi} \in \linors$] preference ranking of voter $i \in \voters$
		\item [$\prof=(\succ_{1},\dots,\succ_{n}) \in \linors^\voters$] a profile
		\item [$\powersetz{\allalts}$] possible winners (non-empty subsets of $\allalts$)
		\item [$f: \linors^\voters \rightarrow \powersetz{\allalts}$] a Social Choice Rule
	\end{description}
\end{frame}

\section[Simultaneous Elicitation of PSR and Agent Preferences]{Simultaneous Elicitation of Scoring Rule and Agent Preferences for Robust Winner Determination}

\begin{frame}
	\frametitle{Simultaneous Elicitation of Scoring Rule and Agent Preferences for Robust Winner Determination}
	
	\textbf{Setting}: Incompletely specified preferences and social choice rule \\ 
	
	\vspace{1em}
%	\onslide<2->{
%		\begin{itemize}
%			\item Agents: difficult or costly to order \emph{all} alternatives
%			\item Chair: difficult to \emph{specify} a voting rule precisely
%		\end{itemize}
%	}
	\onslide<2->\textbf{Goal}: Reduce uncertainty, inferring (\textit{eliciting}) incrementally and simultaneously the true preferences of agents and chair to quickly converge to an optimal or a near-optimal alternative \\
	\vspace{1em}
	\onslide<3-> \textbf{Approach}:
	\vspace{-0.5em}
%	\newline
%	\setbeamertemplate{bibliography item}[article]
%	\usebeamertemplate{bibliography item}
%	\bibentry{Napolitano2021}
	\begin{block}{}
		\includegraphics[scale=0.035]{assets/article.png} {\footnotesize \bibentry{Napolitano2021}}
	\end{block}
	\begin{itemize}
		\item Develop query strategies that interleave questions to the chair and to the agents
		\item Use \emph{Minimax regret} to measure the quality of those strategies
	\end{itemize}
\end{frame}

\begin{frame}
	\frametitle{Related Works}
	\textbf{Incomplete profile}  
	\begin{itemize}
		\item and known rule: Minimax regret to produce a robust winner approximation (\textit{Lu and Boutilier 2011}, \cite{Lu2011}; \textit{Boutilier et al. 2006}, \cite{Boutilier2006})
	\end{itemize}~\\
	\textbf{Uncertain rule} 
	\begin{itemize}
		\item and complete profile: dominance relations derived to eliminate alternatives always less preferred than others (\textit{Stein et al. 1994}, \cite{Stein1994})
		\item considering positional scoring rules (\textit{Viappiani 2018}, \cite{Viappiani2018})
	\end{itemize}
\end{frame}

%\begin{frame}
%	\frametitle{Assumptions}
%	\begin{itemize}
%		\item We consider \textit{Positional Scoring Rules}, which attach weights to positions according to a scoring vector $W$
%		\item We assume $W$ to be \textit{convex}
%		\[ W_r - W_{r+1} \geq W_{r+1}-W_{r+2}\]
%		for all positions $r$, and that $W_1=1$ and $W_m=0$
%	\end{itemize}	
%\end{frame}

%\begin{frame}
%	\frametitle{Assumptions}
%	\vspace{-2em}
%	\small{Agents} $=\{$ \includegraphics[scale=0.05]{assets/v1.png}, \includegraphics[scale=0.05]{assets/v2.png}, \includegraphics[scale=0.05]{assets/v3.png} $\}$, $\ $ \small{Alternatives} $=\{$ \includegraphics[scale=0.025]{assets/d1.png}, \includegraphics[scale=0.025]{assets/d2.png}, \includegraphics[scale=0.025]{assets/d3.png} $\}$, $\ $ \small{Chair} $=$ \includegraphics[scale=0.02]{assets/committee.png} $\Rightarrow$ \small{Positional Scoring Rule}
%	\vspace{2em}
%	\begin{columns}
%		\begin{column}{0.45\textwidth}
%			\begin{figure}
%				\begin{tabular}{ccc}
%					\visible<1->{\subfloat{\includegraphics[scale=0.12]{assets/v1.png}}} &
%					\visible<1->{\subfloat{\includegraphics[scale=0.12]{assets/v2.png}}} &
%					\visible<1->{\subfloat{\includegraphics[scale=0.12]{assets/v3.png}}} \\
%					\visible<1->{\subfloat{\includegraphics[scale=0.05]{assets/d1.png}}} &
%					\visible<1->{\subfloat{\includegraphics[scale=0.05]{assets/d1.png}}} & \\
%					\visible<1->{\subfloat{\includegraphics[scale=0.05]{assets/d2.png}}} &
%					\visible<1->{\subfloat{\includegraphics[scale=0.05]{assets/d2.png}}} & \\
%					\visible<1->{\subfloat{\includegraphics[scale=0.05]{assets/d3.png}}} & & \\
%				\end{tabular}
%			\end{figure}
%		\end{column}
%		\begin{column}{0.45\textwidth}
%			\begin{center}
%				\visible<1->{\includegraphics[width=0.5\textwidth]{assets/committee.png}}\\
%				\vspace{2em}
%				\visible<1->{weight$(1$\textsuperscript{st}$)$ $\geq$ 2 $\cdot$ weight$(2$\textsuperscript{nd}$)$}
%			\end{center}
%			
%		\end{column}
%	\end{columns}
%	
%\end{frame}

\begin{frame}
	\frametitle{Context}	
	\begin{description}[$W=(\w_k,\ 1\leq k \leq m), \ W \in \mathcal{W}$]
%		\item [$A$] alternatives, $|A|=m$
%		\item [$N$] agents (\textit{voters})
		\item [$P = (\pref_i, \ i \in N ), \ P \in \mathcal{P} $] complete preferences profile unknown to us
		\item [$W=(W_r,\ 1\leq r \leq m), \ W \in \mathcal{W}$] \textbf{convex} scoring vector that the chair has in mind
	\end{description}
	\bigskip
	\onslide<2-> \begin{block}{}
		$W$ defines a \textbf{Positional Scoring Rule} $f_W(P)\subseteq A$ 
		%using scores \usebeamercolor*[fg]{description item}$s^{W,P}(a), \ \forall \ a \in A$
	\end{block}
	\onslide<3-> \begin{block}{}
		$P$ and $W$ exist in the minds of agents and chair but unknown to us
	\end{block}	
	
\end{frame}


%%%
%\begin{frame}
%	\frametitle{Outline}
%	\tableofcontents[hideallsubsections, sectionstyle=shaded/show]
%\end{frame}
%
%\AtBeginSection{
%	\begin{frame}
%		\frametitle{Outline}
%		\tableofcontents[currentsection, hideallsubsections]
%	\end{frame}
%}

\begin{frame}
	\frametitle{Questions}
	Two types of questions:\\
	\vspace{1.5em}
	\onslide<2-> \textbf{Questions to the agents}
	\begin{itemize}
		\item[] Comparison queries that ask a particular agent $i$ to compare two alternatives $a, b \in \allalts$
%		\usebeamercolor*[fg]{description item}\[a \pref_j b \quad ?\]
	\end{itemize}
	\vspace{2em}
	\onslide<3->  \textbf{Questions to the chair}
	\begin{itemize}
		\item[] Queries relating the difference between the importance of consecutive ranks from $r$ to $r+2$
%		\usebeamercolor*[fg]{description item} \[ W_{r} - W_{r+1} \geq \lambda (W_{r+1} - W_{r+2}) \quad ? \] 
	\end{itemize}
	\vspace{1em}
	\onslide<4->{The answers to these questions define $\mathbf{C_P}$ and $\mathbf{C_W}$ that is our knowledge about P and W }
\end{frame}

%\begin{frame}
%	\frametitle{Current Knowledge}
%	The answers to these questions define $C_P$ and $C_W$ that is our knowledge about P and W 
%	\medskip
%	\begin{itemize}
%		\onslide<2->{\item $C_P \subseteq \mathcal{P}$ constraints on the profile given by the agents}
%		\onslide<3->{\item $C_W \subseteq \mathcal{W}$ constraints on the voting rule given by the chair}
%	\end{itemize}
%
%	\def\Pcircle{(0,0) circle (0.6cm)}
%	\def\CPcircle{(0,0) circle (0.2cm)}
%	\def\Wcircle{(0,-1.6) circle (0.6cm)}
%	\def\CWcircle{(0,-1.6) circle (0.2cm)}
%	\def\Allcircle{(3,-0.8) circle (0.6cm)}
%	\def\Currcircle{(3,-0.8) circle (0.2cm)}
%	\def\wcircle{(3,-0.8) circle (0.05cm)}
%	\begin{center}
%		\begin{tikzpicture}
%			\onslide<2->{\draw \Pcircle node[left] at (0,0.7) {\footnotesize$\mathcal{P}$};
%			\filldraw[fill=gray, draw=black] \CPcircle node[above]  at (0,0.07) {\footnotesize$C_P$};}
%			\onslide<3->{\draw \Wcircle node[left] at (0,-0.9) {\footnotesize$\mathcal{W}$};
%			\filldraw[fill=gray, draw=black] \CWcircle node[above]  at (0,-1.5) {\footnotesize$C_W$};}
%			\onslide<4->{\draw [decorate, decoration = {brace, raise=40pt, amplitude=5pt}] (0,0.7) --  (0,-2.3);}
%			\onslide<5->{\draw \Allcircle node[above] at (3,-0.2) {\footnotesize$\powersetz{\allalts} {\scriptstyle \ = \ 2^{\allalts}\setminus \emptyset}$ };}
%			\onslide<5->{\filldraw[fill=gray, draw=black] \Currcircle node[above]  at (3,-0.7){\footnotesize$C_{\mathscr{P}}$};}
%			\onslide<6->{\filldraw[fill=black, draw=black] \wcircle;}
%		\end{tikzpicture}
%	\end{center}
%\end{frame}

\begin{frame}
	\frametitle{Minimax Regret}
	Given $C_P \subseteq \mathcal{P}$ and $C_W \subseteq \mathcal{W}$:
	
	\onslide<1-> \begin{block}{}
	The \emph{Maximum Regret} MR of an alternative $a$ is the highest possible loss when selecting $a$ as a winner under all possible completions of $C_P$ and $C_W$
	\end{block}
	
	\onslide<2-> \begin{block}{}
		This can be seen as a game in which an adversary selects a completion of the profile and weights in order to maximize the regret of choosing $a$
	\end{block}

%	\begin{block}{}
%		\[\color{darkred}{\PMR^{C_P,C_W}(a,b)}= \max_{P\in C_P, W \in C_W} s^{P,W}(b)-s^{P,W}(a) \]
%		is the maximum difference of score between $a$ and $b$ under all possible realizations of the full profile {\em and} weights
%	\end{block}
%	
%	\onslide<2->  We care about the worst case loss: \emph{maximum regret} between a chosen alternative $a$ and best real alternative $b$
%	\[\MR^{C_P,C_W}(a)= \max_{b\in \allalts} \PMR^{C_P,C_W}(a,b) \]
%	 	
	\onslide<3->
	\[\color{darkred}\MMR^{C_P,C_W} = \min_{a\in \allalts} \MR^{C_P,C_W}(a)\]
%	\onslide<3-> \centerline{\textbf{We select the alternative that \emph{minimizes} the maximum regret}}
\end{frame}

	\begin{frame}
	\frametitle{Minimax Regret}
	
		The alternative that \emph{minimizes} the maximum regret is used:
		\begin{itemize}
			\item \onslide<2-> as winner recommendation when the elicitation process stops
			\item \onslide<3-> to guide elicitation strategies
			%intuitively, we try to reduce this regret by asking for more information about this alternative
		\end{itemize}
		
	\end{frame}


%\begin{frame}
%	\frametitle{Pairwise Max Regret Computation}
%	The computation of $\PMR^{C_P,C_W}( \icimg{assets/d1.png},\icimg{assets/d3.png})$ can be seen as a game in which an adversary both:\begin{itemize}
%		\onslide<2-> \item \textbf{chooses a complete profile $\mathbf{P \in \mathcal{P}}$}\\
%		\begin{columns}
%			\begin{column}{0.4\textwidth}
%				\begin{figure}
%					\begin{tabular}{ccc}
%						\subfloat{\includegraphics[scale=0.12]{assets/v1.png}} &
%						\subfloat{\includegraphics[scale=0.12]{assets/v2.png}} &
%						\subfloat{\includegraphics[scale=0.12]{assets/v3.png}} \\
%						\subfloat{\includegraphics[scale=0.05]{assets/d1.png}} &
%						\subfloat{\includegraphics[scale=0.05]{assets/d1.png}} & \\
%						\subfloat{\includegraphics[scale=0.05]{assets/d2.png}} &
%						\subfloat{\includegraphics[scale=0.05]{assets/d2.png}} & \\
%						\subfloat{\includegraphics[scale=0.05]{assets/d3.png}} & & \\
%					\end{tabular}
%				\end{figure}
%			\end{column}
%			\begin{column}{0.08\textwidth}
%				\icarr{arrow.png}
%			\end{column}
%			\begin{column}{0.4\textwidth}
%				\begin{figure}
%					\begin{tabular}{ccc}
%						\subfloat{\includegraphics[scale=0.12]{assets/v1.png}} &
%						\subfloat{\includegraphics[scale=0.12]{assets/v2.png}} &
%						\subfloat{\includegraphics[scale=0.12]{assets/v3.png}} \\
%						\subfloat{\includegraphics[scale=0.05]{assets/d1.png}} &
%						\subfloat{\includegraphics[scale=0.05]{assets/d3.png}} & 
%						\subfloat{\includegraphics[scale=0.05]{assets/d3.png}} \\
%						\subfloat{\includegraphics[scale=0.05]{assets/d2.png}} &
%						\subfloat{\includegraphics[scale=0.05]{assets/d1.png}} & 
%						\subfloat{\includegraphics[scale=0.05]{assets/d2.png}} \\
%						\subfloat{\includegraphics[scale=0.05]{assets/d3.png}} &
%						\subfloat{\includegraphics[scale=0.05]{assets/d2.png}} & 
%						\subfloat{\includegraphics[scale=0.05]{assets/d1.png}} \\
%					\end{tabular}
%				\end{figure}
%				
%			\end{column}
%		\end{columns}
%		
%		\onslide<3-> \item \textbf{chooses a feasible weight vector $\mathbf{W \in \mathcal{W}}$}\\
%		\medskip
%		\centerline{\color{darkred}$\mathbf{(1,?,0)} \qquad$ \icarr{arrow.png} \color{darkred}$\qquad \mathbf{(1,0,0)}$}
%	\end{itemize}
%	\medskip
%	in order to maximize the difference of scores
%\end{frame}
	
	\begin{frame}
		\frametitle{Elicitation strategies}
		At each step, the strategy selects a question to ask either to one of the agents about her preferences or to the chair about the voting rule \\ \bigskip
		\onslide<1-> The termination condition could be:
		\begin{itemize}
			\item <1-> when the minimax regret is lower than a threshold
			\item <1-> when the minimax regret is zero
		\end{itemize}
		\bigskip
	\end{frame}
	
%	\begin{frame}[t]
%		\frametitle{Elicitation strategies}
%		\framesubtitle{Pessimistic Strategy}
%		It selects $n+(m-2)$ candidate questions:
%		\vspace{1em}
%		\begin{itemize}
%			\item \textbf{One question per agent:} It selects incomparable alternatives that can be used by the "adversary" to increase the PMR
%			\vspace{1em}
%			\item \textbf{One question per rank} (excluding the first and the last one which are known)\textbf{:} It selects a value $\lambda$ for which it is still possible to ask whether the difference of weights between the current rank and the next one is $\lambda$ times better than the next two ranks
%%			For each rank r take the middle of the interval of values for $\lambda$ that are still possible and asks whether \[W_r - W_{r+1} \geq \lambda (W_{r+1} - W_{r+2})\]
%		\end{itemize}
%	\end{frame}
	
	\begin{frame}[t]
		\frametitle{Elicitation strategies}
		\framesubtitle{Pessimistic Strategy}
		\vspace{1em}
		\begin{itemize}
			\item It selects first $n+(m-2)$ candidate questions: one per each agent and one per each rank (excluding the first and the last one which are known)
			\vspace{1em}
			\item It selects the question that leads to minimal regret in the worst case from the set candidate questions
		\end{itemize}
		
%		\onslide<1-> Assume that a question leads to the possible new knowledge states $(C_P^1, C_W^1)$ and $(C_P^2, C_W^2)$ depending on the answer, then its score is:
%		\[\max_{i=1,2} \MMR(C_P^i, C_W^i) \]
%		\onslide<1-> The pessimistic strategy 
		
%		\bigskip
%		{\small \onslide<3-> \begin{block}{Note:}
%			if the maximal MMR of two questions are equal, then prefers the one with the lowest MMR values associated to the opposite answer
%		\end{block}}
	\end{frame}
	
	
	\begin{frame}
		\frametitle{Empirical Evaluation}
		\framesubtitle{Pessimistic for different datasets}
		\begin{figure}
			\centering
			\caption{Average MMR (normalized by $n$) after $k$ questions with Pessimistic strategy for different datasets.}
			\label{fig:linearity}
			\begin{tikzpicture}
				\pgfplotsset{
					every axis legend/.append style={
						at={(0.5,1.1)},
						anchor=south
					},
				}
				\begin{axis}[
					y=80,
					legend columns=3,
					xlabel=Number of Questions,
					ylabel=MMR/n,
					ytick={0,0.5,1},
					xtick distance=100,
					xtick pos=left,
					ymajorgrids=true,
					enlarge x limits=-1, %hack to plot on the full x-axis scale
					width=10cm, %set bigger width
					ytick style={draw=none},
					ymin=0,
					ymax=1,
					xmin=0,
					xmax=1000,
					yticklabels={0,0.5,1},
					legend style={font=\footnotesize}]
					
					\addlegendimage{mark=halfsquare right*,brown,mark size=2}
					\addlegendimage{mark=diamond*,red,mark size=2}
					\addlegendimage{mark=pentagon*,cyan,mark size=2}
					\addlegendimage{mark=halfcircle*,violet,mark size=2}
					\addlegendimage{mark=*,pink,mark size=2}
					\addlegendimage{mark=triangle*,green,mark size=2}
					\addlegendimage{mark=halfsquare left*,blue,mark size=2}
					\addlegendimage{mark=square*,teal,mark size=2}
					\addlegendimage{mark=halfsquare*,magenta,mark size=2}
					
					
					\addplot[thick, mark=halfsquare right*, mark size = {2}, mark indices = {120}, brown] table [x=k, y=5.20]{data/linearity.dat};
					\addlegendentry{m=5, n=20}
					\addplot[thick, mark=diamond*, mark size = {2}, mark indices = {150}, red] table [x=k, y=10.20]{data/linearity.dat};
					\addlegendentry{m=10, n=20}
					\addplot[thick, mark=pentagon*, mark size = {2}, mark indices = {240}, cyan] table [x=k, y=11.30]{data/linearity.dat};
					\addlegendentry{m=11, n=30}
					\addplot[thick, mark=halfcircle*, mark size = {2}, mark indices = {400}, violet] table [x=k, y=tshirts]{data/linearity.dat};
					\addlegendentry{tshirts m11n30}
					\addplot[thick, mark=*, mark size = {2}, mark indices = {400}, pink] table [x=k, y=courses]{data/linearity.dat};
					\addlegendentry{courses m9n146}
					\addplot[thick, mark=triangle*, mark size = {2}, mark indices = {400}, green] table [x=k, y=9.146]{data/linearity.dat};
					\addlegendentry{m=9, n=146}
					\addplot[thick, mark=halfsquare left*, mark size = {2}, mark indices = {200}, blue] table [x=k, y=14.9]{data/linearity.dat};
					\addlegendentry{m=14, n=9}
					\addplot[thick, mark=square*, mark size = {2}, mark indices = {60}, teal] table [x=k, y=skate]{data/linearity.dat};
					\addlegendentry{skate m14n9}
					\addplot[thick, mark=halfsquare*, mark size = {2}, mark indices = {400}, magenta] table [x=k, y=15.30]{data/linearity.dat};
					\addlegendentry{m=15, n=30}
				\end{axis}
			\end{tikzpicture}
		\end{figure}
	\end{frame}

\begin{frame}
	\frametitle{Building concrete questions for the chair}
	Queries relating the difference between the importance of consecutive ranks
%	\only<1>{{\usebeamercolor*[fg]{description item} \[ W_{r} - W_{r+1} \geq \lambda (W_{r+1} - W_{r+2}) \quad ? \] }}
	\onslide<1->{{\usebeamercolor*[fg]{description item} \[ W_{2} - W_{3} \geq 2 \ (W_{3} - W_{4}) \quad ? \] }}
	\onslide<2->{Abstract queries are difficult to answer}
	{\usebeamercolor*[fg]{description item}
		\begin{align*}
			\onslide<4->{W_{2}\ - \ W_{3} &\geq 2 \ W_{3} - 2 \ W_{4} }\\
			\onslide<5->{W_{2} + 2 \ W_{4} &\geq 3 \ W_{3} }\\
			\onslide<6->{s(a) &\geq s(b)}
		\end{align*}
	}
	\only<6>{\begin{figure}
			\begin{center}
				$
				\begin{array}{cc}
					\mathbf{\#1}&\mathbf{\#2} \\
					-&-\\
					\mathbf{a}&-\\
					\mathbf{b}&\mathbf{b}\\
					-&\mathbf{a}\\
				\end{array}
				$
			\end{center}
		\end{figure}
	}
	\onslide<7->{
		\begin{figure}
			\begin{center}
				$
				\begin{array}{cccc}
					\mathbf{\#1}&\mathbf{\#2}&\mathbf{\#3}&\mathbf{\#3} \\
					c&d&a&b\\
					\mathbf{a}&c&b&a\\
					\mathbf{b}&\mathbf{b}&c&d\\
					d&\mathbf{a}&d&c\\
				\end{array}
				$
			\end{center}
		\end{figure}
	}
\end{frame}


\section{Majority Judgment winner determination under incomplete information}

\begin{frame}
	\frametitle{Incomplete knowledge: profile}
	\vspace{-2em}
	\footnotesize{Agents} $=\{$ \includegraphics[scale=0.05]{assets/v1.png}, \includegraphics[scale=0.05]{assets/v2.png}, \includegraphics[scale=0.05]{assets/v3.png} $\}$, $\ $ \footnotesize{Altern.} $=\{$ \includegraphics[scale=0.025]{assets/d1.png}, \includegraphics[scale=0.025]{assets/d2.png}, \includegraphics[scale=0.025]{assets/d3.png} $\}$, $\ $ \footnotesize{Chair} $=$ \includegraphics[scale=0.02]{assets/committee.png} $\Rightarrow$ \footnotesize{Voting Rule}
	\vspace{2em}
	\begin{columns}
		\begin{column}{0.45\textwidth}
			\begin{figure}
				\begin{tabular}{ccc}
				\end{tabular}
			\end{figure}
		\end{column}
		\begin{column}{0.45\textwidth}
			\begin{center}
				\visible<1->{\includegraphics[width=0.5\textwidth]{assets/committee.png}}\\
				\vspace{2em}
				\visible<1->{Majority Judgment}
				%\visible<6->{weight$(1$\textsuperscript{st}$)$ $\geq$ 2 $\cdot$ weight$(2$\textsuperscript{nd}$)$}
			\end{center}
			
		\end{column}
	\end{columns}
\end{frame}

\begin{frame}
	\frametitle{Majority Judgment}
	Voters judges candidates assigning grades from an ordinal scale. The winner is the candidate with the highest median of the grades received. (Balinski and Laraki 2011, \cite{Balinski2011}) \vspace{1cm} \\
	\begin{center}
		\includegraphics[width=0.8\textwidth]{assets/vector.png}		
	\end{center}
\end{frame}

\begin{frame}
	\frametitle{Majority Judgment}
	\footnotesize{Agents} $=\{$ \includegraphics[scale=0.05]{assets/v1.png}, \includegraphics[scale=0.05]{assets/v2.png}, \includegraphics[scale=0.05]{assets/v3.png} $\}$, $\ $ \footnotesize{Altern.} $=\{$ \includegraphics[scale=0.025]{assets/d1.png}, \includegraphics[scale=0.025]{assets/d2.png}, \includegraphics[scale=0.025]{assets/d3.png} $\}$, $\ $ \footnotesize{Chair} $=$ \includegraphics[scale=0.02]{assets/committee.png} $\Rightarrow$ \footnotesize{Majority Judgment}
	\vspace{2em}
	\begin{columns}
		\begin{column}{0.45\textwidth}
			\begin{figure}
				\begin{tabular}{cccc}
					&
					\visible<1->{\subfloat{\includegraphics[scale=0.12]{assets/v1.png}}} &
					\visible<1->{\subfloat{\includegraphics[scale=0.12]{assets/v2.png}}} &
					\visible<1->{\subfloat{\includegraphics[scale=0.12]{assets/v3.png}}} \\
					\visible<1->{\subfloat{\includegraphics[scale=0.05]{assets/d1.png}}} &
					\visible<1->{\subfloat{\includegraphics[scale=0.5]{assets/votes/exc.png}}} & \visible<1->{\subfloat{\includegraphics[scale=0.5]{assets/votes/avg.png}}} & \visible<1->{\subfloat{\includegraphics[scale=0.5]{assets/votes/med.png}}} \\
					\visible<1->{\subfloat{\includegraphics[scale=0.05]{assets/d2.png}}} &\visible<1->{\subfloat{\includegraphics[scale=0.5]{assets/votes/good.png}}} & \visible<1->{\subfloat{\includegraphics[scale=0.5]{assets/votes/med.png}}} & \visible<1->{\subfloat{\includegraphics[scale=0.5]{assets/votes/good.png}}} \\
					\visible<1->{\subfloat{\includegraphics[scale=0.05]{assets/d3.png}}} & \visible<1->{\subfloat{\includegraphics[scale=0.5]{assets/votes/inad.png}}} & 
					\visible<1->{\subfloat{\includegraphics[scale=0.5]{assets/votes/exc.png}}} & 
					\visible<1->{\subfloat{\includegraphics[scale=0.5]{assets/votes/vg.png}}}\\
				\end{tabular}
			\end{figure}
		\end{column}
		\begin{column}{0.45\textwidth}
			\vspace{0.8em}
			\begin{center}
				\visible<2->{\textbf{Median}}
			\end{center}
			\begin{figure}
				\begin{tabular}{cc}
					\visible<2->{\subfloat{\includegraphics[scale=0.05]{assets/d1.png}}} &
					\visible<2->{\subfloat{\includegraphics[scale=0.5]{assets/votes/avg.png}}} \\
					\visible<2->{\subfloat{\includegraphics[scale=0.05]{assets/d2.png}}} &\visible<2->{\subfloat{\includegraphics[scale=0.5]{assets/votes/good.png}}} \\
					\visible<2->{\subfloat{\includegraphics[scale=0.05]{assets/d3.png}}} &
					\visible<2->{\subfloat{\includegraphics[scale=0.5]{assets/votes/vg.png}}}\\
				\end{tabular}
			\end{figure}
			
		\end{column}
	\end{columns}
\begin{center}
	\visible<3->{winner: \includegraphics[scale=0.05]{assets/d3.png}}
\end{center}
\end{frame}

\begin{frame}
	\frametitle{Majority Judgment: Incomplete Knowledge}
	\footnotesize{Agents} $=\{$ \includegraphics[scale=0.05]{assets/v1.png}, \includegraphics[scale=0.05]{assets/v2.png}, \includegraphics[scale=0.05]{assets/v3.png} $\}$, $\ $ \footnotesize{Altern.} $=\{$ \includegraphics[scale=0.025]{assets/d1.png}, \includegraphics[scale=0.025]{assets/d2.png}, \includegraphics[scale=0.025]{assets/d3.png} $\}$, $\ $ \footnotesize{Chair} $=$ \includegraphics[scale=0.02]{assets/committee.png} $\Rightarrow$ \footnotesize{Majority Judgment}
	\vspace{2em}
	\begin{columns}
		\begin{column}{0.45\textwidth}
			\begin{figure}
				\begin{tabular}{cccc}
					&
					\visible<1->{\subfloat{\includegraphics[scale=0.12]{assets/v1.png}}} &
					\visible<1->{\subfloat{\includegraphics[scale=0.12]{assets/v2.png}}} &
					\visible<1->{\subfloat{\includegraphics[scale=0.12]{assets/v3.png}}} \\
					\visible<1->{\subfloat{\includegraphics[scale=0.05]{assets/d1.png}}} &
					\visible<1->{\subfloat{\includegraphics[scale=0.5]{assets/votes/exc.png}}} & \visible<1->{\subfloat{\includegraphics[scale=0.5]{assets/votes/avg.png}}} & \visible<1->{\subfloat{\includegraphics[scale=0.5]{assets/votes/med.png}}} \\
					\visible<1->{\subfloat{\includegraphics[scale=0.05]{assets/d2.png}}} &\visible<1->{\subfloat{\includegraphics[scale=0.5]{assets/votes/good.png}}} & \visible<1->{\subfloat{\includegraphics[scale=0.5]{assets/votes/med.png}}} &  \\
					\visible<1->{\subfloat{\includegraphics[scale=0.05]{assets/d3.png}}} & \visible<1->{\subfloat{\includegraphics[scale=0.5]{assets/votes/inad.png}}} & 
					 & \visible<1->{\subfloat{\includegraphics[scale=0.5]{assets/votes/vg.png}}} \\
				\end{tabular}
			\end{figure}
		\end{column}
		\begin{column}{0.45\textwidth}
			\vspace{0.8em}
			\begin{center}
				\visible<2->{\textbf{Median}}
			\end{center}
			\begin{figure}
				\begin{tabular}{cc}
					\visible<2->{\subfloat{\includegraphics[scale=0.05]{assets/d1.png}}} &
					\visible<2->{\subfloat{\includegraphics[scale=0.5]{assets/votes/avg.png}}} \\
					\visible<2->{\subfloat{\includegraphics[scale=0.05]{assets/d2.png}}} &\visible<2->{\subfloat{\includegraphics[scale=0.5]{assets/votes/med.png}}} \\
					\visible<2->{\subfloat{\includegraphics[scale=0.05]{assets/d3.png}}} &
					\visible<2->{\subfloat{\includegraphics[scale=0.5]{assets/votes/inad.png}}}\\
				\end{tabular}
			\end{figure}
			
		\end{column}
	\end{columns}
\begin{center}
	\visible<3->{winner: \includegraphics[scale=0.05]{assets/d1.png}}
\end{center}
\end{frame}

\begin{frame}
	\frametitle{Majority Judgment}
	\framesubtitle{Uses}
	\onslide<1-> In the last few years MJ has being adopted by a progressively larger number of french political parties including: Le Parti Pirate, Génération(s), LaPrimaire.org, France Insoumise and La République en Marche. \cite{MV} \vspace{1cm}
	
	\onslide<1-> LaPrimaire.org is a french political initiative whose goal is to select an independent candidate for the french presidential election using MJ as voting rule.
\end{frame}

\begin{frame}
	\frametitle{Majority Judgment}
	\framesubtitle{Generalizing LaPrimaire.org}
	The procedure consists of two rounds:
	\vspace{1em}
	\only<1>{\begin{itemize}
		\item[1:] each voter expresses her judgment on five random candidates. The five ones with the highest medians qualify for the second round 
		\vspace{1em}
		\item[2:] each voter expresses her judgment on all the five finalists. The one with the best median is the winner
	\end{itemize}}
	\only<2->{
	\begin{itemize}
		\item[1:] each voter expresses her judgment on {\color{red}k} random candidates. The {\color{red}k} ones with the highest medians qualify for the second round 
		\vspace{1em}
		\item[2:] each voter expresses her judgment on all the {\color{red}k} finalists. The one with the best median is the winner
	\end{itemize}}
\end{frame}

\begin{frame}
	\frametitle{Incomplete Knowledge}
	\begin{remark}
		If a winner of the complete profile is among the k finalists then it will also be a winner of the incomplete profile
	\end{remark}
	\begin{theorem}
		There exist an incomplete profile and one of its completion that do not share the same sets of winners
	\end{theorem}
\vspace{-1em}
	\begin{columns}
		\begin{column}{0.45\textwidth}
			\begin{figure}
				\begin{tabular}{ccccc}
					&
					\visible<2->{\subfloat{\includegraphics[scale=0.07]{assets/v1.png}}} &
					\visible<2->{\subfloat{\includegraphics[scale=0.07]{assets/v2.png}}} &
					\visible<2->{$\dots$} &
					\visible<2->{\subfloat{\includegraphics[scale=0.07]{assets/v3.png}}} \\
					\visible<2->{\subfloat{\includegraphics[scale=0.04]{assets/d1.png}}} &
					\visible<2>{\subfloat{\includegraphics[scale=0.3]{assets/votes/exc.png}}} & \visible<2>{\subfloat{\includegraphics[scale=0.3]{assets/votes/exc.png}}} & \visible<2>{\subfloat{\includegraphics[scale=0.3]{assets/votes/exc.png}}} & \visible<2>{\subfloat{\includegraphics[scale=0.3]{assets/votes/exc.png}}} \\
					\visible<2->{\subfloat{\includegraphics[scale=0.04]{assets/d2.png}}} &\visible<2->{\subfloat{\includegraphics[scale=0.3]{assets/votes/med.png}}} & \visible<2->{\subfloat{\includegraphics[scale=0.3]{assets/votes/med.png}}} & \visible<2->{\subfloat{\includegraphics[scale=0.3]{assets/votes/med.png}}} & \visible<2->{\subfloat{\includegraphics[scale=0.3]{assets/votes/med.png}}} \\
					\visible<2->{$\dots$} & \visible<2->{$\dots$} & \visible<2->{$\dots$} & \visible<2->{$\dots$} \\
					\visible<2->{\subfloat{\includegraphics[scale=0.04]{assets/d3.png}}} &
					\visible<2->{\subfloat{\includegraphics[scale=0.3]{assets/votes/med.png}}} & \visible<2->{\subfloat{\includegraphics[scale=0.3]{assets/votes/med.png}}} & \visible<2->{\subfloat{\includegraphics[scale=0.3]{assets/votes/med.png}}} & \visible<2->{\subfloat{\includegraphics[scale=0.3]{assets/votes/med.png}}} \\
				\end{tabular}
			\end{figure}
		\end{column}
		\begin{column}{0.45\textwidth}
			\vspace{0.8em}
			\begin{center}
				\visible<2->{\textbf{Median}}
			\end{center}
			\begin{figure}
				\begin{tabular}{cc}
					\visible<2->{\subfloat{\includegraphics[scale=0.04]{assets/d1.png}}} &
					\visible<2>{\subfloat{\includegraphics[scale=0.3]{assets/votes/exc.png}}} \\
					\visible<2->{\subfloat{\includegraphics[scale=0.04]{assets/d2.png}}} &\visible<2->{\subfloat{\includegraphics[scale=0.3]{assets/votes/med.png}}} \\
					 \visible<2->{$\dots$} & \visible<2->{$\dots$} \\
					\visible<2->{\subfloat{\includegraphics[scale=0.04]{assets/d3.png}}} &
					\visible<2->{\subfloat{\includegraphics[scale=0.3]{assets/votes/med.png}}}\\
				\end{tabular}
			\end{figure}
			
		\end{column}
	\end{columns}
	%\visible<3->{winner: \includegraphics[scale=0.05]{assets/d1.png}}
	\visible<3->{}
	%simmilar results yealds for profiles where we ask each agent k questions
\end{frame}

\begin{frame}
	\frametitle{Probability of missing the winner}
	\begin{theorem}
		By asking each voter to evaluate k equiprobably picked alternatives, the probability that an alternative $j$ is never asked about is $(1-\frac{k}{m})^n$
	\end{theorem}
	\visible<2>{\begin{center}
			For $k=3$, $n=m=10$ this is $0.7^{10} \approx 0.0282\%$
		\end{center} }
	\visible<3->{
		\begin{center}
			\textbf{What about real scenarios?}
	\end{center}}
\end{frame}

\begin{frame}
	\frametitle{Experimental results}
	\begin{figure}
		\centering
		\begin{subfigure}[b]{0.45\textwidth}
			\centering
			\caption{\scriptsize{Avg prob. of missing the winner under uniform distribution of preferences, for $n=100, m=50$ and $k\in \intvl{1,25}$}\vspace{2.6em}}
			\label{fig:MJelicitationIC}
			\begin{tikzpicture}[scale=0.65]
				\begin{axis}[
					%	y=8,
					xlabel=$k/m$,
					ylabel=Avg. Prob. of Miss.,
					table/col sep=comma,
					%	ytick={0,2,4,6,8,10},
					%	xtick distance=10,
					%	ytick distance=2,
					%	xtick pos=left,
					ymajorgrids=true,
					ytick style={draw=none},
					ymin=0,
					ymax=0.64,
					xmin=0.02,
					xmax=0.5,
					%	yticklabels={0,2,4,6,8,10},
					%	legend style={font=\scriptsize}
					]
					
					\addplot[thick, red] table [x=kom, y=ProbOfMiss]{data/table.dat};
					
				\end{axis}
			\end{tikzpicture}
		\end{subfigure}
		\hfill
		\begin{subfigure}[b]{0.45\textwidth}
			\centering
			\caption{\scriptsize{Avg prob. of missing the winner using a real case distribution of preferences, given $m=12$ several $n$ and $k\in \intvl{1,5}$}}
			\label{fig:MJelicitationUAV}
			\begin{tikzpicture}[scale=0.65]
				\pgfplotsset{
					every axis legend/.append style={
						at={(0.5,1.1)},
						anchor=south
					},
				}
				\begin{axis}[
					%	y=8,
					xlabel=$k$,
					ylabel=Avg. Prob. of Miss.,
					table/col sep=comma,
					legend columns=3,
					%	ytick={0,2,4,6,8,10},
					%	xtick distance=10,
					%	ytick distance=2,
					%	xtick pos=left,
					ymajorgrids=true,
					ytick style={draw=none},
					ymin=0,
					ymax=0.8,
					xmin=1,
					xmax=5,
					yticklabels={0,2,4,6,8,10},
					legend style={font=\scriptsize}
					]
					
					\addlegendimage{mark=halfsquare right*,brown,mark size=2}
					\addlegendimage{mark=diamond*,red,mark size=2}
					\addlegendimage{mark=pentagon*,cyan,mark size=2}
					\addlegendimage{mark=halfcircle*,violet,mark size=2}
					\addlegendimage{mark=*,pink,mark size=2}
					\addlegendimage{mark=triangle*,green,mark size=2}
					\addlegendimage{mark=halfsquare left*,blue,mark size=2}
					\addlegendimage{mark=square*,teal,mark size=2}
					\addlegendimage{mark=halfsquare*,magenta,mark size=2}
					
					
					\addplot[thick, mark=halfsquare right*, mark size = {2}, mark indices = {3}, brown] table [x=k, y=p10]{data/electiontableFig.dat};
					\addlegendentry{n=10}
					\addplot[thick, mark=diamond*, mark size = {2}, mark indices = {3}, red] table [x=k, y=p25]{data/electiontableFig.dat};
					\addlegendentry{n=25}
					\addplot[thick, mark=pentagon*, mark size = {2}, mark indices = {2}, cyan] table [x=k, y=p50]{data/electiontableFig.dat};
					\addlegendentry{n=50}
					\addplot[thick, mark=halfcircle*, mark size = {2}, mark indices = {2}, violet] table [x=k, y=p100]{data/electiontableFig.dat};
					\addlegendentry{n=100}
					\addplot[thick, mark=*, mark size = {2}, mark indices = {1}, pink] table [x=k, y=p250]{data/electiontableFig.dat};
					\addlegendentry{n=250}
					\addplot[thick, mark=triangle*, mark size = {2}, mark indices = {1}, green] table [x=k, y=p500]{data/electiontableFig.dat};
					\addlegendentry{n=500}
					%				\addplot[thick, mark=halfsquare left*, mark size = {2}, mark indices = {1}, blue] table [x=k, y=p1000]{data/electiontableFig.dat};
					%				\addlegendentry{n=1000}
					%				\addplot[thick, mark=square*, mark size = {2}, mark indices = {1}, teal] table [x=k, y=p1500]{data/electiontableFig.dat};
					%				\addlegendentry{n=1500}
					
				\end{axis}
			\end{tikzpicture}
		\end{subfigure}
	\end{figure}
	
\end{frame}



%\begin{frame}
%	\frametitle{Majority Judgment}
%	\framesubtitle{LaPrimaire.org}
%	\footnotesize{Agents} $=\{$ \includegraphics[scale=0.05]{assets/v1.png}, \includegraphics[scale=0.05]{assets/v2.png}, \includegraphics[scale=0.05]{assets/v3.png} $\}$, $\ $ \footnotesize{Altern.} $=\{$ \includegraphics[scale=0.025]{assets/d1.png}, \includegraphics[scale=0.025]{assets/d2.png}, \includegraphics[scale=0.025]{assets/d3.png} $\}$, $\ $ \footnotesize{Chair} $=$ \includegraphics[scale=0.02]{assets/committee.png} $\Rightarrow$ \footnotesize{Majority Judgment}
%	\vspace{2em}
%	\begin{columns}
%		\begin{column}{0.45\textwidth}
%			\begin{figure}
%				\begin{tabular}{cccc}
%					&
%					\visible<1->{\subfloat{\includegraphics[scale=0.12]{assets/v1.png}}} &
%					\visible<1->{\subfloat{\includegraphics[scale=0.12]{assets/v2.png}}} &
%					\visible<1->{\subfloat{\includegraphics[scale=0.12]{assets/v3.png}}} \\
%					\visible<1->{\subfloat{\includegraphics[scale=0.05]{assets/d1.png}}} &
%					\visible<1->{\subfloat{\includegraphics[scale=0.5]{assets/votes/exc.png}}} & \visible<1->{\subfloat{\includegraphics[scale=0.5]{assets/votes/avg.png}}} & \visible<1->{\subfloat{\includegraphics[scale=0.5]{assets/votes/med.png}}} \\
%					\visible<1->{\subfloat{\includegraphics[scale=0.05]{assets/d2.png}}} &\visible<1->{\subfloat{\includegraphics[scale=0.5]{assets/votes/good.png}}} & \visible<1->{\subfloat{\includegraphics[scale=0.5]{assets/votes/med.png}}} & \visible<3->{\subfloat{\includegraphics[scale=0.5]{assets/votes/good.png}}} \\
%					\visible<1>{\subfloat{\includegraphics[scale=0.05]{assets/d3.png}}} & \visible<1>{\subfloat{\includegraphics[scale=0.5]{assets/votes/inad.png}}} & 
%					& \visible<1>{\subfloat{\includegraphics[scale=0.5]{assets/votes/vg.png}}} \\
%				\end{tabular}
%			\end{figure}
%		\end{column}
%		\begin{column}{0.45\textwidth}
%			\vspace{0.8em}
%			\begin{center}
%				\visible<1->{\textbf{Median}}
%			\end{center}
%			\begin{figure}
%				\begin{tabular}{cc}
%					\visible<1->{\subfloat{\includegraphics[scale=0.05]{assets/d1.png}}} &
%					\visible<1->{\subfloat{\includegraphics[scale=0.5]{assets/votes/avg.png}}} \\
%					\visible<1->{\subfloat{\includegraphics[scale=0.05]{assets/d2.png}}} &\only<1-2>{\subfloat{\includegraphics[scale=0.5]{assets/votes/med.png}}} \only<3->{\subfloat{\includegraphics[scale=0.5]{assets/votes/good.png}}} \\
%					\visible<1>{\subfloat{\includegraphics[scale=0.05]{assets/d3.png}}} &
%					\visible<1>{\subfloat{\includegraphics[scale=0.5]{assets/votes/inad.png}}}\\
%				\end{tabular}
%			\end{figure}
%			
%		\end{column}
%	\end{columns}
%\begin{center}
%	\visible<4->{winner: \includegraphics[scale=0.05]{assets/d2.png}}
%\end{center}
%\end{frame}
%
%\begin{frame}
%	\frametitle{Research Questions}
%	\begin{itemize}
%		\item What is the probability of selecting a winner different from the one selected in case of complete knowledge?
%		\item Can we elicit voters preferences using a minimax regret notion?
%		\item What is the best trade-off between communication cost and optimal result?
%		\item What is the voting rule applied on the resulting incomplete profile? What are its properties?
%	\end{itemize}
%\end{frame}


\section{Compromising as an equal loss principle}

\begin{frame}
	\frametitle{Compromising as an equal loss principle}
	
	\textbf{Setting}: Several voters express their preferences over a set of alternatives \\ 
	
	\vspace{1em}

	\onslide<2->\textbf{Goal}:Find a procedure determining a collective choice that promotes a notion of compromise \\
	\vspace{1em}
	\onslide<3-> \textbf{Approach}:
	\vspace{-0.5em}
	%	\newline
	%	\setbeamertemplate{bibliography item}[article]
	%	\usebeamertemplate{bibliography item}
	%	\bibentry{Napolitano2021}
	\begin{block}{}
		\includegraphics[scale=0.035]{assets/article.png} {\footnotesize \bibentry{Cailloux2022}}
	\end{block}
	\begin{itemize}
		\item Define a compromise from an equal loss perspective %, favoring an outcome where every voter concedes as equally as possible
		\item Propose classes of rules reflecting this concept
	\end{itemize}
\end{frame}

\begin{frame}
	\frametitle{Related Works}
	\begin{itemize}
		%\item<1-> \textbf{Plurality}: selects the alternatives considered as best by the highest number of voters 
		%	In other words, it insists on a support of first and highest quality, disregarding the quantity of support this may lead to.
	%	\item<2-> \textbf{Median Voting Rule}: picks all alternatives receiving a majority of support at the highest possible quality (Bassett and Persky, 1999 \cite{Bassett1999})
		\item<1-> \textbf{Majoritarian Compromise}: picks the alternatives that receive the support of the majority of voters at the highest possible quality, breaking ties according to the quantity of support (Sertel, 1986 \cite{Sertel1986})
		%gives up from the quality of support, in order to ensure a majority support behind the selected alternatives.
		\item<2-> \textbf{q-approval FB}: picks the alternatives that receive the support of q voters at the highest possible quality, no tie-breaking
		\item<3-> \textbf{Fallback Bargaining}: q-approval with $q=n$ (Brams and Kilgour, 2001 \cite{Brams2001})
		%bargainers fall back to less and less preferred alternatives until they reach a unanimous agreement
		%	Note that MC and FB winners are	particular cases of q-approval compromises, for q being respectively equal to majority and unanimity. Moreover for q = 1, q-approval compromises	coincide with the plurality rule (PR) winners
	\end{itemize}
	\vspace{1em}
	\onslide<4->\begin{block}{}
		Note: q-approval with $q=1$ corresponds to plurality
	\end{block}
\end{frame}

\begin{frame}
	\frametitle{Motivation}
	$n=100, \allalts=\{a,b,c\}$
	\begin{center}
		$
		\begin{array}{cccccc}
			\mathbf{51} \quad &a&\succ & b & \succ&c\\
			\mathbf{49} \quad &c&\succ & b & \succ&a\\
		\end{array}
		$
	\end{center}
\only<1-4>{
	\begin{itemize}
		\item<2-4> Plurality: $\{a\}$
		\item<3-4> MC: $\{a\}$
		\item<4-4> FB$_q$
			\begin{itemize}
				\item $q\in \left\{ 1, \dots, \frac{n}{2} - 1\right\} $: $\{a,c\}$
				\item $q\in \left\{ \frac{n}{2} , \frac{n}{2} + 1\right\} $: $\{a\}$
				\item $q\in \left\{ \frac{n}{2} + 2, \dots, n \right\} $: $\{b\}$
			\end{itemize}	
	%	\item<5-> FB: $\{b\}$
	\end{itemize}}

	\only<5->{\begin{block}{Observations}
		\begin{itemize}
			\item $b$ receives unanimous support when each voter falls back one step from his ideal point
			\item almost all these SCRs impose a willingness to compromise, but do not ensure a compromise
		\end{itemize}
	\end{block}}
	\only<6>{\begin{block}{Thesis}
		$b$ is a better compromise when egalitarianism is a major concern
	\end{block}}
%	\visible<5->{\centering Does $b$ seem a better compromise?}
	
\end{frame}

%The difference between the two is that one,called egalitarian compromise, insists on equality at the expense of Pareto efficiency while the other, called Paretian compromise, is constrained to pick among the Pareto efficient alternatives.

\begin{frame}
	\frametitle{Equal loss}
	%Our compromises attempt to equalize “losses”
		\begin{description}[$\lambda_{\prof}: \allalts \rightarrow \intvl{0, m - 1}^\voters$]
			\item<1-> [$\lambda_{\prof}: \allalts \rightarrow \intvl{0, m - 1}^\voters$] a loss vector
			\item<3-> [$\sigma: \intvl{0, m - 1}^N \rightarrow \R^+$] a spread measure
		\end{description}
	\onslide<2>{
	\begin{center}
		$
		\begin{array}{cccccc}
			v_1: \quad &a&\succ & b & \succ&c\\
			v_2: \quad &c&\succ & b & \succ&a\\
		\end{array}
		 \qquad
		\begin{array}{c}
			\lambda_{\prof}(a)=(0,2)\\
			\lambda_{\prof}(b)=(1,1) \\
			\lambda_{\prof}(a)=(2,0)
		\end{array}
	$
	\end{center}
	}
	\onslide<4-> \begin{block}{}
		$\Sigma$ is the set of spread measures $\sigma$ such that 
		\[ \sigma(l)=0 \iff l_{i}=l_{j}, \ \forall i,j\in N, \quad \forall l\in\alllosses \].
	\end{block}
	%Thus, the spread of $l$ gets its lowest value $0$ in case of perfect equality and only in this case.
\end{frame}

\begin{frame}
	\frametitle{Equal loss}
	\begin{block}{}
		\[ \argmin_{\allalts}(\sigma~\circ~\lambda_P)=\set{a \in \allalts \suchthat \forall b \in \allalts: \sigma(\lambda_P(a)) \leq \sigma(\lambda_P(b))}\]
	\end{block}
	\vspace{0.5cm}
	$\argmin_{\allalts}(\sigma\circ\lambda_{P})$ denotes the alternatives in $\allalts$ whose loss vectors are the most equally distributed according to $\sigma$
\end{frame}

\begin{frame}
	\frametitle{Equal loss}
	\begin{center}
		\begin{tikzpicture}
			\path node (P) {$\prof$};
			\path (P.south) node[anchor=north] (profile) {$%
				\begin{array}{r@{\hspace{1mm} : \hspace{1mm}}l}
					v_1 & a \pref b \pref c\\%
					v_2 & c \pref b \pref a\\%
				\end{array}%
				$};
			\path (P.center) ++ (3cm, 0) node (L) {$\lambda_{\prof}$};
			\path (L.south) node[anchor=north] (losses) {$%
				\begin{array}{r@{\hspace{1mm} : \hspace{1mm}}l}
					a & (0, 2)\\%
					b & (1, 1)\\%
					c & (2, 0)\\%
				\end{array}%
				$};
		\end{tikzpicture}
	\end{center}
	\bigskip
	
	\[\argmin_{\allalts}(\sigma~\circ~\lambda_P)=\{b\} \quad \forall \sigma \in \Sigma\]
\end{frame}


\begin{frame}
	\frametitle{Egalitarian compromises}
	An SCR is Egalitarian Compromise Compatible iff at each profile, it selects some “less unequal” alternatives
	\begin{block}{Egalitarian compromise compatibility}
		An SCR $f$ is ECC iff 
		\[
		\exists \sigma \in \Sigma \suchthat \forall \prof \in \linors^N: f(\prof) \cap \argmin_{\color{red}{\allalts}} (\sigma \circ \lambda_{\prof}) \neq \emptyset
		\]
	\end{block}
\end{frame}

\begin{frame}
	\frametitle{Egalitarian compromises and Pareto dominance}
	ECC rules are \emph{very} egalitarian
	\begin{center}
		\begin{tikzpicture}
			\path node (P) {$\prof$};
			\path (P.south) node[anchor=north] (profile) {$%
				\begin{array}{r@{\hspace{1mm} : \hspace{1mm}}l}
					v_1 & a \pref c \pref b\\%
					v_2 & c \pref a \pref b\\%
				\end{array}%
				$};
			\path (P.center) ++ (3cm, 0) node (L) {$\lambda_{\prof}$};
			\path (L.south) node[anchor=north] (losses) {$%
				\begin{array}{r@{\hspace{1mm} : \hspace{1mm}}l}
					a & (0, 1)\\%
					b & (2, 2)\\%
					c & (1, 0)\\%
				\end{array}%
				$};
		\end{tikzpicture}
	\end{center}
	\[\argmin_{\allalts} (\sigma \circ \lambda_{\prof}) = \{b\} \quad \forall \sigma \in \Sigma\]
	%	\onslide<3>{Egalitarian compromises favor equality over Pareto efficiency}
	\onslide<2->\begin{theorem}
		ECC $\cap$ Paretian = $\emptyset$ \hfill {\small (for $n, m \geq 2$)}
	\end{theorem}
	\onslide<3-> $f \in \text{ECC} \Rightarrow b \in f(\prof)$, $f \in \text{Paretian} \Rightarrow b \notin f(\prof)$
\end{frame}

%\begin{frame}
%	\frametitle{ECC $\cap$ Paretian = $\emptyset$}
%	\begin{theorem}
%		No Paretian SCR is ECC \hfill {\small (for $n, m \geq 2$)}
%	\end{theorem}
%	\begin{proof}[Proof (for $m$ = 4; adaptations for any $m \geq 2$ are straightforward)]
%		$\begin{array}{r@{}l@{\hspace{1mm} \mapsto \hspace{1mm}}l}
%			1 &\text{ voter} & a_1 \pref a_2 \pref a_3 \pref x\\%
%			n - 1 &\text{ voters} & a_2 \pref a_3 \pref a_1 \pref x\\%
%		\end{array}$%
%		\begin{itemize}
%			\item $\forall \sigma \in \Sigma: \argmin_\allalts (\sigma \circ \lambda_{\prof}) = \set{x}$
%			\item $f \in \text{ECC} \Rightarrow x \in f(\prof)$
%			\item $f \in \text{Paretian} \Rightarrow x \notin f(\prof)$ \qedhere
%		\end{itemize}
%	\end{proof}
%\end{frame}


\begin{frame}
	\frametitle{Paretian compromises}
	An SCR is Paretian Compromise Compatible iff at each profile, it selects some “less unequal” alternatives \emph{among the Pareto optimal ones}
	\begin{block}{Paretian compromise compatibility}
		An SCR $f$ is PCC iff 
		\[\exists \sigma \in \Sigma \suchthat \forall \prof \in \linors^N: f(\prof) \cap \argmin_{\color{red}{\PO(\prof)}} (\sigma \circ \lambda_{\prof}) \neq \emptyset \]
	\end{block}
%	\onslide<2->{\small Recall:
%		\begin{block}{Egalitarian compromise compatibility}
%			An SCR $f$ is ECC iff 
%			\[\exists \sigma \in \Sigma \suchthat \forall \prof \in \linors^N: f(\prof) \cap \argmin_\allalts (\sigma \circ \lambda_{\prof}) \neq \emptyset\]
%	\end{block}}
\end{frame}

\begin{frame}
	\frametitle{Results}
	\only<1>{For at least three voters and no restrictions on $\Sigma$:
		\vspace{2em} 
		\begin{center}
			\begin{tabular}{| c | C{0.15\textwidth} | C{0.15\textwidth} |}
				\hline
				& ECC & PCC \\
				\hline
				Condorcet procedures & {\color{red}No} & {\color{red}No} \\
				\hline
				Scoring rules & {\color{red}No} & {\color{red}No} \\
				\hline
				Antiplurality & {\color{red}No} & {\color{mygreen}Yes} \\
				\hline
				BK compromises & {\color{red}No} & {\color{red}No} \\
				\hline
				Fallback bargaining &  {\color{red}No} & {\color{mygreen}Yes} \\
				\hline
			\end{tabular}
		\end{center}
	}
\end{frame}


\begin{frame}
	\frametitle{Restricting $\Sigma$}
	We consider a restriction $\bar{\Sigma}\subset\Sigma$ such that, for each $\bar{\sigma}\in \bar{\Sigma}$ if
%	\begin{definition}[Condition $C_{m, n}$]
%		Given $m \geq 4, n \geq \max\set{4, m - 1}$, $\sigma$ satisfies condition $C_{m, n}$ iff $\sigma(m - 3, m - 1, m - 2, …, m - 2) < \sigma(m - 2, m - 3, …, 1, 0, …, 0)$.
%	\end{definition}
	\begin{center}
		$\begin{array}{r@{\hspace{1mm} : \hspace{1mm}}lllll}
			v_1 &	&	&\mathbf{a}	&\mathbf{b}	&x_1\\%
			v_2 &	&	&\mathbf{b}	&	&\mathbf{a}\\%
			v_3 &	&\mathbf{b}	&	&\mathbf{a}	&x_2\\%
			v_4 &\mathbf{b}	&	&	&\mathbf{a}	&x_3\\%
		\end{array}$%
	\end{center}
	then: $(\bar{\sigma} \circ \lambda_{\prof})(a) < (\bar{\sigma} \circ \lambda_{\prof})(b)$
\end{frame}

\begin{frame}
	\frametitle{Restricting $\Sigma$}
	\begin{theorem}
		Under $\bar{\Sigma}$, AP and FB are not PCC.
	\end{theorem}
	\begin{proof}[Proof for $m = 5, n = 4$]
		$\begin{array}{r@{\hspace{1mm} : \hspace{1mm}}lllll}
			v_1 &	&	&a	&b	&x_1\\%
			v_2 &	&	&b	&	&a\\%
			v_3 &	&b	&	&a	&x_2\\%
			v_4 &b	&	&	&a	&x_3\\%
		\end{array}$%
		\vspace{1em}
		\begin{itemize}
			\item $b$ is the only alternative never last, thus for both rules: $f(\prof) = \set{b}$
			\item $(\bar{\sigma}\circ \lambda_{\prof})(a) < (\bar{\sigma} \circ \lambda_{\prof})(b)$
			\item and $a \in \PO(\prof)$, thus $b \notin \argmin_{\PO(\prof)}(\bar{\sigma} \circ \lambda_{\prof})$ \qedhere
		\end{itemize}
	\end{proof}
\end{frame}


\begin{frame}
	\frametitle{Results}
	\only<1>{For at least three voters and \textbf{no} restrictions on $\Sigma$:
		\vspace{2em} 
		\begin{center}
			\begin{tabular}{| c | C{0.15\textwidth} | C{0.15\textwidth} |}
				\hline
				& ECC & PCC \\
				\hline
				Condorcet procedures & {\color{red}No} & {\color{red}No} \\
				\hline
				Scoring rules & {\color{red}No} & {\color{red}No} \\
				\hline
				Antiplurality & {\color{red}No} & {\color{mygreen}Yes} \\
				\hline
				BK compromises & {\color{red}No} & {\color{red}No} \\
				\hline
				Fallback bargaining &  {\color{red}No} & {\color{mygreen}Yes} \\
				\hline
			\end{tabular}
		\end{center}
	}
	\only<2->{For at least three voters \textbf{with} restrictions $\bar{\Sigma}$ on $\Sigma$:
		\vspace{2em} 
		\begin{center}
			\begin{tabular}{| c | C{0.15\textwidth} | C{0.15\textwidth} |}
				\hline
				& ECC & PCC \\
				\hline
				Condorcet procedures & {\color{red}No} & {\color{red}No} \\
				\hline
				Scoring rules & {\color{red}No} & {\color{red}No} \\
				\hline
				Antiplurality & {\color{red}No} & {\color{red}No} \\
				\hline
				BK compromises & {\color{red}No} & {\color{red}No} \\
				\hline
				Fallback bargaining &  {\color{red}No} & {\color{red}No} \\
				\hline
			\end{tabular}
		\end{center}
	}
	\only<3>{\begin{block}{}
			Similar results for two voters
	\end{block}}
\end{frame}

%\begin{frame}
%	\frametitle{Other results}
%	\begin{theorem}
%		Condorcet consistent rules are neither ECC nor PCC \hfill {\small (for $m, n \geq 3$)}
%	\end{theorem}
%	\begin{theorem}
%		Scoring rules,except AP, are neither ECC nor PCC \hfill {\small (for $m \geq 3$ and large enough $n$)}
%	\end{theorem}
%	\begin{theorem}
%		FB$_q$ rules with $q \in \intvl{1, n - 1}$ are neither ECC nor PCC \hfill {\small (for $m, n \geq 3$)}
%	\end{theorem}
%\end{frame}

\section{Conclusions}
\begin{frame}
	\frametitle{Conclusions}
\begin{block}{Considering a classical setting, we:}
	\begin{itemize}
		\item revised the concept of compromise on an equal loss perspective
		\item proved that almost all SCRs fail to ensure a compromise
		\item defined new classes of voting rules reflection this notion
	\end{itemize}
%	Considering a classical model in which the preferences of a set of voters over a set of alternatives are known, we defined two classes of voting rules able to reflect a notion of compromise in which egalitarianism, in the sense of conceding equally, is a major concern.
\end{block}
\onslide<2->\begin{block}{Considering incomplete knowledge, we:}
	\begin{itemize}
	\item analyzed the elicitation strategy used in a real voting scenario using MJ
	\item introduced a simultaneous and incremental elicitation approach for agents and chair preferences
	\item developed several strategies and released our framework for further experiments and improvements
	\end{itemize}
%	Moreover, we stepped back from this standard perspective in	which preferences are assumed to be known from the beginning and investigated the problem of preference elicitation in different settings.
\end{block}
\end{frame}

\begin{frame}
	\frametitle{Future work}
	\begin{block}{Considering a classical setting:}
		\begin{itemize}
			\item the cardinal setting can be analyzed including intensity of preferences
			\item new definitions of compromise can be conceived
			\item the trade-off between equity and efficiency can be explored
		\end{itemize}
		%	Considering a classical model in which the preferences of a set of voters over a set of alternatives are known, we defined two classes of voting rules able to reflect a notion of compromise in which egalitarianism, in the sense of conceding equally, is a major concern.
	\end{block}
	\onslide<2->\begin{block}{Considering incomplete knowledge using MJ:}
		\begin{itemize}
			\item the manipulability of the elicitation process using MJ can be explored
			\item steps toward explicability and axiomatization can be taken
		\end{itemize}
	\end{block}
	\onslide<3->\begin{block}{Considering incomplete knowledge of agents and chair preferences:}
		\begin{itemize}
			\item more strategies with different heuristics can be implemented
			\item the elicitation of the rule can be expanded to more than scoring rules and the convexity constraint can be relaxed
			\item the conversion of questions into profiles can be used in other settings
		\end{itemize}	
		%	Moreover, we stepped back from this standard perspective in	which preferences are assumed to be known from the beginning and investigated the problem of preference elicitation in different settings.
	\end{block}
\end{frame}

\addtocounter{framenumber}{-1}
\begin{frame}[plain]
	\centering {\color{darkred}\LARGE Thank You!}
	{\let\thefootnote\relax\footnote{\tiny Icon made by Eucalyp from www.flaticon.com}}
\end{frame}


\begin{frame}[plain,allowframebreaks,noframenumbering]
	\scriptsize \bibliography{biblio}
\end{frame}

\begin{frame}[plain,noframenumbering]
	\frametitle{Empirical Evaluation}
	\framesubtitle{Pessimistic reaching "low enough" regret}
	\sisetup{table-number-alignment = center, table-figures-integer=2, table-figures-decimal=1, table-auto-round}
	\begin{minipage}{\textwidth}
		\captionof{table}{Questions asked by Pessimistic strategy on several datasets to reach $\frac{n}{10}$ regret, columns 4 and 5, and zero regret, last two columns.}
		\label{tab:questions}
		\begin{center}
			\scalebox{0.8}{
				\begin{tabular}{cccS[table-number-alignment = center, table-figures-integer=2] S[table-figures-integer=3, table-figures-decimal=1]@{ $\vert$ }S[table-figures-integer=2, table-figures-decimal=1]@{ $\vert$ }S[table-figures-integer=2, table-figures-decimal=1]@{ ]} S[table-number-alignment = center, table-figures-integer=4]S[table-figures-integer=3, table-figures-decimal=1]@{ $\vert$ }S[table-figures-integer=2, table-figures-decimal=1]@{ $\vert$ }S[table-figures-integer=2, table-figures-decimal=1]@{ ]}}
					\toprule
					{dataset} & m & n &{$q_{c}^{\scriptscriptstyle{MMR} \leq n/10}$} & \multicolumn{3}{c}{$q_{a}^{\scriptscriptstyle{MMR} \leq n/10}$} & {$q_{c}^{ \scriptscriptstyle{MMR} = 0}$} & \multicolumn{3}{c}{$q_{a}^{ \scriptscriptstyle{MMR} = 0}$} \\
					\midrule
					m5n20 & 5&20&0.0&[4.3 &4.95 & 5.84 &5.25&[ \ 5.36 & 6.15 & 7.21\\
					m10n20&10&20&0.0&[13.85 & 16.1& 18.41&31.95&[19.66 & 21.78 & 24.7\\
					m11n30&11&30&0.0&[16.55&19.0&22.26&45.15&[23.07&25.7&28.89\\
					tshirts&11&30&0.0&[13.08&16.6&19.58& 43.15 &[28.22&31.98 &35.62\\
					courses&9&146&0.0&[6.03 &7.0 &7.0&0.0&[6.81 & 7.0 &7.0\\
					%m9n146&9&146&0&0&[1.94 &8 &9.25&999.3&0.47&0&0&[1.94&8&9.25&999.3&0.47\\
					m14n9&14&9&5.4&[30.3&33.45&36.65&64.05&[37.55&40.5&44.3\\
					skate&14&9&0.0&[11.35&11.6&12.3&0.0&[11.5&11.8&12.75 \\
					m15n30&15&30&0.0&[24.95&29.5&33.68 \\
					\bottomrule
			\end{tabular}}
		\end{center}
	\end{minipage}
\end{frame}


\begin{frame}[plain,noframenumbering]
	\frametitle{Empirical Evaluation}
	\framesubtitle{Pessimistic chair first and then agents (and vice-versa)}
	\sisetup{table-number-alignment = center, table-figures-integer=2, table-figures-decimal=1, table-auto-round}
	\begin{minipage}{\textwidth}
		\centering
		\captionof{table}{Average MMR in problems of size $(10, 20)$ after $500$ questions, among which $q_c$ to the chair.}
		\label{tab:twoP500}
		\scalebox{0.85}{
			\begin{tabular}{S[table-figures-integer=3, table-figures-decimal=0]S[table-number-alignment = right]@{ ± }S[table-number-alignment = left, table-figures-integer=1]S[table-number-alignment = right]@{ ± }S[table-number-alignment = left, table-figures-integer=1]}
				\toprule
				{$q_c$} & {ca} & {sd} & {ac} & {sd} \\
				\midrule		
				
				0	&	0.62	&	0.52	&	0.62	&	0.52	\\
				15	&	0.515	&	0.48	&	0.54	&	0.46	\\
				30	&	0.345	&	0.47	&	0.325	&	0.425	\\
				50	&	0.045	&	0.09	&	0.03	&	0.065	\\
				100	&	0.14	&	0.23	&	0.075	&	0.135	\\
				200	&	2.305	&	1.36	&	2.145	&	1.845	\\
				300	&	5.15	&	2.38	&	6.83	&	0.625	\\
				400	&	10.905	&	0.89	&	12.245	&	0.99	\\
				500	&	20.0	&	0.0	&	20.0	&	0.0	\\
				
				\bottomrule
			\end{tabular}
		}
	\end{minipage}
\end{frame}

\begin{frame}
	\frametitle{FB and AP are PCC}
	\begin{theorem}
		FB and Antiplurality are PCC. \hfill {\small (for $n, m \geq 3$)}
	\end{theorem}
	\begin{proof}[Proof sketch]
		Define $\sigma^\text{discrete}(l) = 1 \iff l$ is not constant.
		
		If some $a \in \PO(\prof)$ has a constant loss vector, e.\ g.\ 
		\\$\begin{array}{l}
			a_1 \pref a_2 \pref a_3 \pref a_4\\
			a_3 \pref a_2 \pref a_1 \pref a_4 \\
		\end{array}$
		\\there is exactly one such alternative, $FB(\prof) = \set{a}$ and it is never last so $a \in AP(\prof)$.
		
		Otherwise, $\sigma$ does not discriminate among $\PO(\prof)$, thus Paretianism suffices.
	\end{proof}
\end{frame}

\end{document}