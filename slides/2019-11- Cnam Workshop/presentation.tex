\documentclass{beamer}
\usetheme{CambridgeUS}
\usecolortheme{beaver}
\usepackage{amsmath,amssymb,enumerate,amsthm}
\usepackage{bm}
\usepackage{graphicx}

\usepackage{algorithm, algpseudocode}
\usepackage{amsmath,amssymb,enumerate,amsthm}
\usepackage[strict]{siunitx}
\usepackage{hyperref}
\usepackage{mathrsfs}

\usepackage{bm}
\usepackage{empheq}
\usepackage{cleveref}
\usepackage{xcolor}
\usepackage{textcomp}


\newcommand{\profile}{\bm{v}}%(complete) profile
\newcommand{\pprofile}{{\bm{p}}}%partial profile
\newcommand{\w}{\bm{w}}
\newcommand{\W}{\mathcal{W}}
\newcommand{\Co}{\mathcal{C}}
\newcommand{\pw}{W}%our knowledge about the weights
\newcommand{\strat}[1]{\emph{#1}}
\newcommand{\ppref}{\succ^\text{p}}%partial pref
\newcommand{\pprefeq}{\succeq^\text{p}}%partial pref
\newcommand{\pref}{\succ}% pref
\DeclareMathOperator{\Regret}{Regret}
\DeclareMathOperator{\SCORE}{Score}
\DeclareMathOperator{\PMR}{PMR}
\DeclareMathOperator{\MR}{MR}
\DeclareMathOperator{\MMR}{MMR}

\newcommand*{\icimg}[1]{%
	\raisebox{-.3\baselineskip}{%
		\includegraphics[
		height=\baselineskip,
		width=\baselineskip,
		keepaspectratio,
		]{#1}%
	}%
}

\newcommand*{\icarr}[1]{%
	\raisebox{-0.4\baselineskip}{%
		\includegraphics[
		height=2.5\baselineskip,
		width=3\baselineskip,
		keepaspectratio,
		]{#1}%
	}%
}

\makeatletter
\defbeamertemplate*{title page}{mydefault}[1][]
{
	\vbox{}
	\vfill
	\begin{centering}

%{\usebeamercolor[fg]{titlegraphic}\inserttitlegraphic\par}
		\begin{beamercolorbox}{titlegraphic}
				\usebeamerfont{titlegraphic}\inserttitlegraphic
		\end{beamercolorbox}%
			\vskip1em\par	
		\begin{beamercolorbox}[rounded=true, center, shadow=true, sep=8pt,#1]{title}
			\usebeamerfont{title}\inserttitle\par%
			\ifx\insertsubtitle\@empty%
			\else%
			\vskip0.5em%
			{\usebeamerfont{subtitle}\usebeamercolor[fg]{subtitle}\insertsubtitle\par}%
			\fi%     
		\end{beamercolorbox}%
		\vskip1em\par
		\begin{beamercolorbox}[sep=8pt,center,#1]{author}
			\usebeamerfont{author}\insertauthor
		\end{beamercolorbox}
		\begin{beamercolorbox}[sep=8pt,center,#1]{institute}
			\usebeamerfont{institute}\insertinstitute
		\end{beamercolorbox}
		\begin{beamercolorbox}[sep=8pt,center,#1]{date}
			\usebeamerfont{date}\insertdate
		\end{beamercolorbox}\vskip0.5em
		\begin{beamercolorbox}[sep=8pt,center,#1]{logo}
			\usebeamerfont{titlegraphic}\insertlogo
		\end{beamercolorbox}%
	\end{centering}
	\vfill
}
\setbeamertemplate{title page}[mydefault]
\makeatother



\titlegraphic{\includegraphics[width=50mm]{logo_dauphine} \hspace*{5.5cm} \includegraphics[width=7mm]{cnrs}}
\title[Elicitation of Incomplete Preferences]{Simultaneous Elicitation of Committee and \\ Voters' Preferences}
\institute[]{$^1$ LAMSADE, Université Paris-Dauphine, Paris, France \\ $^2$ LIP6, Sorbonne Universit\'e, Paris, France}
\author[B. Napolitano, O. Cailloux, P. Viappiani]{B. Napolitano$^1$, O. Cailloux$^1$ and P. Viappiani$^2$}
%
%\title[Elicitation and Explanation in Social Choice]{Elicitation and Explanation in Social Choice Theory}
%%\subtitle{Proposal: ``Elicitation and Explanation for Voting Rules''}
%\author[Beatrice Napolitano]{\textbf{Beatrice Napolitano} \\
%	Supervisors: Remzi Sanver, Olivier Cailloux}
\date[28 November 2019]{{\small Advances in Economic Design : Games,
		voting, information and measurement} \\ 28 November 2019 \\ \includegraphics[width=35mm]{LOGO_LAMSADE} }

\usepackage{tikz}
\usepackage{amsmath}
\usepackage{graphicx}

\definecolor{darkred}{rgb}{0.8,0,0}

\begin{document}

\beamertemplatenavigationsymbolsempty

\begin{frame}[plain]
\maketitle
\end{frame}

\addtocounter{framenumber}{-1}


\section{Setting}
\subsection{Scenario}

\begin{frame}[t]
	\frametitle{Classical Scenario}
	\textbf{Setting}: Voters specify preferences over alternatives and a committee defines the social choice rule to aggregate them
	\begin{figure}
		\includegraphics[scale=0.35]{classset.png}
		%		\caption{.}
		%		\label{fig:b1}
	\end{figure}
	\onslide<2-> \textbf{Goal}: Find a consensus choice 
\end{frame}

\begin{frame}[t]
\frametitle{Our Scenario}
%\framesubtitle{Robust Winner Determination}
%	\textbf{Setting}: Two kind of players
\textbf{Setting}: Incompletely specified preferences and social choice rule \bigskip
\begin{figure}
	\includegraphics[scale=0.35]{ourset.png}
	%		\caption{.}
	%		\label{fig:b1}
\end{figure}
\onslide<2-> \textbf{Goal}: Develop an incremental elicitation strategy to acquire the most relevant information 
\end{frame}
%\addtocounter{framenumber}{-1}
%\begin{frame}
%	\frametitle{Scenario}
%	\textbf{Setting}: Incompletely specified profile and positional scoring rule
%	\begin{figure}
%		\includegraphics[scale=0.35]{set2.png}
%%		\caption{.}
%%		\label{fig:b1}
%	\end{figure}
%	\textbf{Goal}: Development of an incremental elicitation protocol based on minimax regret 
%\end{frame}

\subsection{Motivation and approach}
\begin{frame}
	\frametitle{Motivation and approach}
	\textbf{Who?}
	\begin{itemize}
		\item Imagine to be an \emph{external observer} helping with the voting procedure
	\end{itemize}
	\onslide<2-> \textbf{Why?}
	\begin{itemize}
		\item Voters: difficult or costly to order \emph{all} alternatives
		\item Committee: difficult to \emph{specify} a voting rule precisely and abstractly
	\end{itemize}
	\onslide<3-> \textbf{What?}
	\begin{itemize}
		\item We want to reduce uncertainty, inferring (\textit{eliciting}) the true preferences of voters and committee, in order to quickly converge to an optimal
		or a near-optimal alternative
	\end{itemize}		
\end{frame}

\begin{frame}
	\frametitle{Motivation and approach}
	\onslide<1-> \textbf{Approach}
	\begin{itemize}
		\item Develop query strategies that interleave questions to the committee and questions to the voters
		\item Use \emph{Minimax regret} to measure the quality of those strategies
	\end{itemize}
	\onslide<2-> \textbf{Assumptions}
	\begin{itemize}
		\item We consider \textit{positional scoring rules}, which attach weights to positions according to a scoring vector $w$
		\item We assume $w$ to be \textit{convex}
		\[ w_r - w_{r+1} \geq w_{r+1}-w_{r+2} \qquad \forall r\]
		and that $w_1=1$ and $w_m=0$
	\end{itemize}	
\end{frame}

\subsection{Related Works}
\begin{frame}
	\frametitle{Related Works}
	\textbf{Incomplete profile}  
	\begin{itemize}
		\item and known weights: Minimax regret to produce a robust winner approximation (\textit{Lu and Boutilier 2011}, \cite{Lu2011}; \textit{Boutilier et al. 2006}, \cite{Boutilier2006})
	\end{itemize}~\\
	\textbf{Uncertain weights} 
	\begin{itemize}
		\item and complete profile: dominance relations derived to eliminate alternatives always less preferred than others (\textit{Stein et al. 1994}, \cite{Stein1994})
		\item in positional scoring rules (\textit{Viappiani 2018}, \cite{Viappiani2018})
	\end{itemize}
\end{frame}

\subsection{Context}

\begin{frame}
	\frametitle{Context}

		\begin{description}[$W=(\w_k,\ 1\leq k \leq m), \ W \in \mathcal{W}$]
			\item [$A$] alternatives, $|A|=m$
			\item [$N$] voters 
			\item [$P = (\pref_j, \ j \in N ), \ P \in \mathcal{P} $] complete preferences profile 
			\item [$W=(\w_r,\ 1\leq r \leq m), \ W \in \mathcal{W}$] (convex) scoring vector that the committee has in mind
		\end{description}
		\bigskip
		\onslide<2-> \begin{block}{}
			$W$ defines a Positional Scoring Rule $f_W(P)\subseteq A$ using scores \color{blue}$s^{W,P}(a), \ \forall \ a \in A$
		\end{block}
		\onslide<3-> \begin{block}{}
			$P$ and $W$ exist in the minds of voters and committee but unknown to us
		\end{block}
		
\end{frame}

\subsection{Questions}
\begin{frame}
	\frametitle{Questions}
	\onslide<2-> \textbf{Questions to the voters}
	\begin{itemize}
		\item[] Comparison queries that ask a particular voter to compare two alternatives $a, b \in A$
		\color{blue}\[a \pref_j b \quad ?\]
	\end{itemize}
	\onslide<3->  \textbf{Questions to the committee}
	\begin{itemize}
		\item[] Queries relating the difference between the importance of consecutive ranks from $r$ to $r+2$
		\color{blue} \[ w_{r} - w_{r+1} \geq \lambda (w_{r+1} - w_{r+2}) \quad ? \] 
	\end{itemize}
\end{frame}

\subsection{Our Knowledge}
\begin{frame}
	\frametitle{Our Knowledge}
	The answers to these questions define $C_P$ and $C_W$ that is our knowledge about P and W 
	\medskip
	\begin{itemize}
		\onslide<2-> \item $C_P \subseteq \mathcal{P}$ constraints on the profile given by the voters
		\onslide<3-> \item $C_W \subseteq \mathcal{W}$ constraints on the voting rule given by the committee
	\end{itemize}
\end{frame}

\section{Minimax Regret}
\subsection{Definition}
\begin{frame}
	\frametitle{Minimax Regret}
	Given $C_P \subseteq \mathcal{P}$ and $C_W \subseteq \mathcal{W}$:
	
	\begin{block}{}
		\[\color{red}{\PMR^{C_P,C_W}(a,b)}= \max_{P\in C_P, W \in C_W} s^{P,W}(b)-s^{P,W}(a) \]
		is the maximum difference of score between $a$ and $b$ under all possible realizations of the full profile {\em and} weights
	\end{block}
	
	\onslide<2->  We care about the worst case loss: \emph{maximal regret} between a chosen alternative $a$ and best real alternative $b$
	\[\MR^{C_P,C_W}(a)= \max_{b\in A} \PMR^{C_P,C_W}(a,b) \]
	 	
	\onslide<3-> \centerline{\textbf{We select the alternative which \emph{minimizes} the maximal regret}}
	\[\MMR^{C_P,C_W}= \min_{a\in A} \MR^{C_P,C_W}(a)\]
\end{frame}

\subsection{Pairwise Max Regret Computation}
\begin{frame}
	\frametitle{Pairwise Max Regret Computation}
	The computation of $\PMR^{C_P,C_W}( \icimg{salad.png},\icimg{aubergine.png})$ can be seen as a game in which an adversary both:
	\begin{itemize}
		\onslide<2-> \item \textbf{chooses a complete profile $\mathbf{P \in \mathcal{P}}$}\\
		\medskip
		\begin{center}
			\includegraphics[scale=0.35]{completion4.png}
		\end{center}
		
		\onslide<3-> \item \textbf{chooses a feasible weight vector $\mathbf{W \in \mathcal{W}}$}\\
		\medskip
		\centerline{\color{red}$\mathbf{(1,?,0)}$ \icarr{arrow.png} \color{red}$\mathbf{(1,0,0)}$}
	\end{itemize}
	\medskip
	in order to maximize the difference of scores
\end{frame}

\subsection{Computing Minimax Regret}
\begin{frame}[t]
	\frametitle{Computing Minimax Regret: Example}
	\textbf{Profile completion}\\
	Consider the following partial profile
	\begin{center}
		\includegraphics[scale=0.33]{compl.png}
	\end{center}
\end{frame}
\begin{frame}[t]
	\frametitle{Computing Minimax Regret: Example}
	\textbf{Weight selection} \\ \bigskip
	Consider the following constraint on the scoring vector given by the committee
	\[w_1 \geq 2 \cdot w_2\]
	and the convex assumption
	\[w_1 - w_2 \geq w_2 - w_3 \]

\end{frame}
\begin{frame}[t]
	\frametitle{Computing Minimax Regret: Example}
	\textbf{Minimax computing}\\
	\begin{center}
		\includegraphics[scale=0.35]{minmax.png}
	\end{center}
\end{frame}

\section{Elicitation}
	
	\subsection{Elicitation strategies}
	\begin{frame}
		\frametitle{Elicitation strategies}
		At each step, the strategy selects a question to ask either to one of the voters about her preferences or to the committee about the voting rule \\ \bigskip
		\onslide<2-> The termination condition could be:
		\begin{itemize}
			\item <3-> when the minimax regret is lower than a threshold
			\item <4-> when the minimax regret is zero
		\end{itemize}
		\bigskip
	\end{frame}

	\begin{frame}[t]
		\frametitle{Elicitation strategies}
		\textbf{Random Strategy} \\
		\bigskip
		\onslide<2-> Decides with a probability of $\frac{1}{2}$ each whether to ask a question about
		\bigskip
			\begin{itemize}
				\item<3-> \textbf{weights}: it draws a rank $2 \leq r \leq m-2$ equiprobably, takes $\lambda$ as the middle of the interval of values we are still uncertain about, and asks whether $w_{r}-w_{r+1} \geq \lambda (w_{r+1} - w_{r+2})$
				\item<4-> \textbf{a preference ordering}: it draws equiprobably a voter whose preference is not known entirely and two alternatives that are incomparable in our current knowledge
			\end{itemize}
	\end{frame}	
	
	\begin{frame}[t]
		\frametitle{Elicitation strategies}
		\textbf{Pessimistic Strategy} \\
		\bigskip
		\onslide<2-> Assume that a question leads to the possible new knowledge states $(C_P^1, C_W^1)$ and $(C_P^2, C_W^2)$ depending on the answer, then the badness of the question in the worst case is:
		\[\max_{i=1,2} \MMR(C_P^i, C_W^i) \]
		\onslide<3-> The pessimistic strategy selects the question that leads to minimal regret in the worst case \\
		\bigskip
		\onslide<4-> \begin{block}{Note:}
			if the maximal MMR of two questions are equal, then prefers the one with the lowest MMR values associated to the opposite answer
		\end{block}
	\end{frame}	
	
	\begin{frame}[t]
		\frametitle{Elicitation strategies}
		\textbf{Two Phase Strategy} \\
		\bigskip
		\begin{itemize}
			\item <2-> \textbf{phase one:} asks a predefined sequence of $m - 2$ questions to the committee in order to gather informations about the weights (one per rank except the extremes)
			\item <3-> \textbf{phase two:} asks only questions to the voters, using the mechanism defined in the Pessimistic strategy
		\end{itemize}
	\end{frame}
	
	\begin{frame}[t]
		\frametitle{Elicitation strategies}
		\textbf{Extreme Completions Strategy} \\
		\medskip
		\onslide<2-> Let $a^{*}$ be the minimax regret optimal alternative, and $\bar{b}$, $\bar{P}$ and $\bar{W}$ the instantiations that maximize the regret when $a^*$ is chosen. \onslide<3-> Define: 
		\begin{align*}
		\tau_{W} & = \min_{W \in C_W} s^{\bar{P},W}(\bar{b}) - s^{\bar{P},W}(a^{*}) \\
		\tau_{P_i} & = \min_{\hat{P_i} \in C_P} s^{\hat{P_i},\bar{W}}(\bar{b}) -  s^{\hat{P_i},\bar{W}}(a^{*})
		\end{align*}
		where $\hat{P_i}$ is defined as $\bar{P}$ except for $i$ \\
		\bigskip
		\onslide<4->
		$\MMR - \tau_{W}$ estimates the contribution to the regret of our uncertainty about the weights; $\MMR - \tau_{P_i}$ estimates the uncertainty about the profile
		\\ \bigskip
		\onslide<5-> The extreme completions strategy asks a question to whoever minimizes $\tau$
	\end{frame}
	
	\section{Results}
	\subsection{Preliminary Results}
	\begin{frame}
		\frametitle{Preliminary Results}
		\sisetup{round-mode=places, round-precision=1, table-figures-integer=1, table-figures-decimal=1}
		\begin{table}
			\begin{center}
				\begin{tabular}{S[table-figures-integer=2, table-number-alignment = right, table-figures-decimal=0]S[table-number-alignment = right]@{$\pm$}S[table-number-alignment = left]S[table-number-alignment = right]@{$\pm$}S[table-number-alignment = left]S[table-number-alignment = right]@{$\pm$}S[table-number-alignment = left]S[table-number-alignment = right]@{$\pm$}S[table-number-alignment = left]}
					\hline
					{k} & {Rnd} & {sd} & {T. ph.} & {sd} & {Pes.} & {sd} \\
					\hline
					0 & 5.0 & 0 & 5.0 & 0 & 5.0 & 0\\
					5 & 4.93 & 0.16 & 4.98 & 0.01 & 4.15 & 0.32\\
					10 & 4.84 & 0.22 & 4.35 & 0.28 & 3.44 & 0.46\\
					15 & 4.28 & 0.58 & 3.70 & 0.25 & 2.66 &0.54 \\
					20 & 3.85 & 0.33 & 2.67 & 0.51 & 1.62 & 0.82\\
					25 & 3.50 & 0.78 & 2.17 & 0.66 & 0.84 &  0.57\\
					30 & 2.96 & 0.80 & 1.47 & 1.10 & 0.53 & 0.70\\
					\hline
				\end{tabular}
			\end{center}
			\caption{Minimax regret in problems of size $(5, 5)$ after $k$ questions.}
			\label{fig:xp2}
		\end{table}
	\end{frame}

\addtocounter{framenumber}{-1}
\begin{frame}[plain]
	\centering \color{darkred}\LARGE Thank You!
\end{frame}




\bibliographystyle{plain}
\bibliography{biblio} 
%given a combination of axioms we want to find an outcome that doesn't satisfy them, and we would do that for several reasons:
%-querying the user, depending of her answer we might infer her preferences over the set of axioms;
%-proving that a set of axioms is not valid giving a counter-example;


%A method for automatically proving impossibility theorems in the area of ranking sets of objects has already been implemented (Geist \& Endriss, 2011). It:



\end{document}